\documentclass[table]{beamer}
% \documentclass[table,handout]{beamer}
% \setbeameroption{show notes}
% \setbeameroption{hide notes}
% \setbeameroption{show only notes}
\usepackage{varwidth}

\newif\ifhide
\newif\ifpost
\newif\ifhideclicker

\hidetrue
% \hideclickertrue
% \posttrue

\newcommand{\whiteout}[1]{\textcolor{white}{#1}}
\newcommand{\whiteoutbox}[1]{\fcolorbox{white}{white}{\parbox{\dimexpr \linewidth-2\fboxsep-2\fboxrule}{\whiteout{#1}}}}
\newcommand{\notebox}[1]{\fcolorbox{blue}{white}{\parbox{\dimexpr \linewidth-2\fboxsep-2\fboxrule}{#1}}}

\ifhide%
    \newcommand{\hmask}[1]{\phantom{\varwidth{\linewidth}#1\endvarwidth}}%
\else%
    \newcommand{\hmask}[1]{#1}%
\fi

\ifhide%
    \newcommand{\hignore}[1]{}%
\else%
    \newcommand{\hignore}[1]{#1}%
\fi

\ifpost%
    \newcommand{\nopost}[1]{}%
\else%
    \newcommand{\nopost}[1]{#1}%
\fi

\ifhide%
    \newcommand{\hidebox}[1]{\phantom{\varwidth{\linewidth}#1\endvarwidth}}%
\else%
    \newcommand{\hidebox}[1]{\fbox{\parbox{\linewidth}{#1}}}%
\fi

\ifhide%
    \newcommand{\wbox}[1]{\whiteoutbox{#1}}%
\else%
    \newcommand{\wbox}[1]{\notebox{#1}}%
\fi

% \ifhide%
%     \newcommand{\clickeranswer}[1]{#1}%
% \else%
%     \newcommand{\clickeranswer}[1]{\textbf{\textcolor{blue}{#1}}}%
% \fi

\ifhideclicker%
    \newcommand{\clickeranswer}[1]{#1}%
\else%
    \ifhide%
        \newcommand{\clickeranswer}[1]{#1}%
    \else%
        \newcommand{\clickeranswer}[1]{\textbf{\textcolor{blue}{#1}}}%
    \fi
\fi

\input{../utils/slide-preamble2.tex}
\newcommand{\highlight}[1]{\textcolor{violet}{\textit{\textbf{#1}}}}
\newcommand{\super}[1]{\ensuremath{^{\textrm{#1}}}}
\newcommand{\sub}[1]{\ensuremath{_{\textrm{#1}}}}
\newcommand{\dC}{\ensuremath{^\circ{\textrm{C}}}}
\newcommand{\tb}{\hspace{2em}}
\providecommand{\e}[1]{\ensuremath{\times 10^{#1}}}
\newcommand{\myHangIndent}{\hangindent=5mm}

\makeatletter
\newcommand*{\rom}[1]{\expandafter\@slowromancap\romannumeral #1@}
\makeatother

\newcommand{\blankslide}{{\setbeamercolor{background canvas}{bg=black}
\setbeamercolor{whitetext}{fg=white}
\begin{frame}<handout:0>[plain]
\end{frame}}}

\newcommand{\whiteslide}{
\begin{frame}<handout:0>[plain]
\end{frame}}

\newcommand{\f}[1]{\ensuremath{F_{#1}}}

\bibliography{../bib/references}
\input{../utils/title-info.tex}

\title{Ecosystems I: Energy \& Nutrients}
% \date{\today}
\date{June 1, 2015}

% \setbeamertemplate{section in toc}[sections numbered]
% \setbeamertemplate{subsection in toc}[subsections numbered]

\begin{document}

\begin{noheadline}
\maketitle
\end{noheadline}

\nopost{
\begin{noheadline}
\begin{frame}[c]
    \vspace{-6mm}
    \begin{center} 
        \includegraphics[height=1.2\textheight]{../images/seating-chart-2.pdf}
    \end{center}
\end{frame}
\end{noheadline}
}

\clickerslide{
\begin{noheadline}
\begin{frame}
    \begin{adjustwidth}{-2em}{-1.5em}

        % \vspace{-3mm}

        \centerline{
            \includegraphics[height=0.65\textheight]{diversity-productivity-pathogen-clicker.png}}

        \vspace{-2mm}
        \begin{clickerquestion}
            \item What is the take-home message from comparing these graphs? 

            \begin{clickeroptions}
                \item \clickeranswer{Species-poor plots have low NPP due to
                        disease.}
                \item The sterilization process itself is responsible for the
                    differences observed.
                \item In disease-free environments, NPP does not change with
                    increasing species richness.
                \item In disease-free environments, NPP plateaus with
                    increasing species richness.
            \end{clickeroptions}
        \end{clickerquestion}
    \end{adjustwidth}
    \note[item]{What's the mechanism? More species, lower probability they will
        all be hit hard by disease (diseases are mostly species specific). In a
        monoculture = higher population density = higher transmission rates}
\end{frame}
\end{noheadline}
}

\begin{noheadline}
\begin{frame}
\frametitle{Today's issues:}
\vspace{5mm}
% \tableofcontents[subsectionstyle=hide]
\tableofcontents
\end{frame}
\end{noheadline}

\section{How does energy flow through ecosystems?}

\begin{frame}[t]
    \begin{adjustwidth}{-2em}{-1.5em}
        \vspace{-3mm}
        How does energy flow through ecosystems?

        \vspace{2mm}
        Patterns of net primary productivity I

        \includegraphics[width=\linewidth]{productivity-by-ecosystems.png}

    \end{adjustwidth}
\end{frame}

\begin{frame}[t]
    \begin{adjustwidth}{-2em}{-1.5em}
        \vspace{-3mm}
        \includegraphics[width=\linewidth]{productivity-by-ecosystems.png}

        \begin{itemize}
            \item Where is NPP/km\super{2} relatively high?

                \nbox{\scriptsize Tropical rainforests, algal beds/coral reefs,
                    wetlands}

            \item Why is NPP/km\super{2} in the open ocean so low, if lots of
                light is available and the water is warm?

                \nbox{No nutrients (non-living biomass sinks to bottom, where
                    it's very cold and there's no light)}

            \item Why does so much of the total NPP come from open ocean?

                \nbox{It's HUGE!}
        \end{itemize}

    \end{adjustwidth}
\end{frame}

\begin{frame}[t]
    \begin{adjustwidth}{-2em}{-1.5em}
        \vspace{-3mm}
        Patterns of net primary productivity II

        \includegraphics[width=\linewidth]{productivity-heat-map.png}

        \begin{itemize}
            \item Compare and contrast where terrestrial vs.\ marine
                productivity is high.

                \nbox{\tiny Terrestrial is highest near equator; marine is
                    highest along shorelines and high latitudes}

            \item Generally, what limits productivity in terrestrial habitats?

                \nbox{\tiny Temperature and water}

        \end{itemize}

    \end{adjustwidth}
\end{frame}

\clickerslide{
\begin{frame}
    \begin{clickerquestion}
        \item  Per m\super{2}, how does productivity compare in terrestrial
            vs.\ marine environments?

        \begin{clickeroptions}
            \item \clickeranswer{Terrestrial is much higher}
            \item Terrestrial is slightly higher
            \item About the same
            \item Marine is slightly higher
            \item Marine is much higher
        \end{clickeroptions}
    \end{clickerquestion}
\end{frame}
}

\clickerslide{
\begin{frame}
    \begin{clickerquestion}
        \item On a global scale, what is the most important limitation on
            productivity in marine habitats? 
 
        \begin{clickeroptions}
            \item \clickeranswer{Nutrients}
            \item Amount of incident radiation (sunlight)
            \item Water quality (pollution)
            \item Water temperature
        \end{clickeroptions}
    \end{clickerquestion}
\end{frame}
}

\begin{frame}[t]
    \begin{adjustwidth}{-2em}{-1.5em}

        \vspace{-3mm}
        \begin{itemize}
            \item In the open ocean, what happens to nutrients available in
                organisms living at the surface?

                \nbox{\scriptsize Lots of sunlight at surface; primary
                    producers (PP) convert sunlight to biomass; uneaten PPs
                    sink to benthos when they die}

            \item In the open ocean, why aren't nutrients recycled from the
                benthos?

                \nbox{\scriptsize It's cold and dark on the bottom; no primary
                    productivity. Nutrients can't get to the surface where the
                    PPs are}

            \item Why are coastlines so productive?

                \nbox{\scriptsize When deep currents hit the continental shelf,
                    upwelling brings nutrients from the benthos to the
                    surface---available for PPs}
        \end{itemize}

    \end{adjustwidth}
\end{frame}

\begin{frame}[t]
    \begin{adjustwidth}{-2em}{-1.5em}
        What happens to energy (NPP) on land? (Data from Hubbard Brook)

        \includegraphics[width=\linewidth]{energy-transfer-cascade.png}

        In this ecosystem, what percentage of NPP gets eaten as living tissue?
        
        \nbox{\tiny $(1103/4514)100 = 24\%$}

    \end{adjustwidth}
    \note[item]{What number is GPP? What number is NPP?}
\end{frame}

\begin{frame}[t]
    \begin{adjustwidth}{-2em}{-1.5em}
        Trophic levels in a Puget Sound food chain

        \uncover<2->{
        \begin{table}%[htbp]
            \centering
            \begin{tabular}{ c c c }
                \multicolumn{1}{p{12mm}}{Trophic level} &
                \multicolumn{1}{c}{Feeding strategy} &
                \multicolumn{1}{c}{Grazing food chain} \\
                \hline
                5 & 4\degree consumer & Orcas \\[2ex]
                4 & 3\degree consumer & Salmon \\[2ex]
                3 & 2\degree consumer & Herring \\[2ex]
                2 & 1\degree consumer & Copepods \\[2ex]
                1 & Autotroph (PP) & Algae \& bacteria \\ 
            \end{tabular}
        \end{table}
        }

        \uncover<3->{
        \begin{itemize}
            \item What about orcas that eat seals (seals eat salmon)?
            \item How would a Puget Sound food WEB be different?
        \end{itemize}
        }
    \end{adjustwidth}
    \note[item]{What about orcas that eat seals (seals eat salmon)}
    \note[item]{Many consumers eat at multiple trophic levels}
    \note[item]{How would a Puget Sound food web be different?}
\end{frame}

\begin{frame}[t]
    \begin{adjustwidth}{-2em}{-1.5em}
        \vspace{-3mm}
        Efficiency of energy transfer across trophic levels at Hubbard Brook

        \begin{center}
        \begin{tikzpicture}[>=latex]%,xscale=0.3,yscale=0.3]
        
            \tikzstyle{state} = [draw, fill=structure!20!, rectangle,
                minimum height=2em,
                minimum width=4em,
                node distance=4em,
                font={\sffamily\bfseries}]
            \tikzstyle{stateEdgePortion} = [black,ultra thick];
            \tikzstyle{stateEdge} = [stateEdgePortion,->];
            \tikzstyle{edgeLabel} = [pos=0.5, text centered, font={\sffamily\small}];
        
            % Reservoirs
            \node[visible on=<6->, state, name=c3] {3 g/m\super{2}/year of
                tertiary consumer biomass (10\% efficiency)};
            \node[visible on=<5->, state, name=c2, below of=c3] {30 g/m\super{2}/year of
                secondary consumer biomass (15\% efficiency)};
            \node[visible on=<4->, state, name=c1, below of=c2] {200 g/m\super{2}/year of
                primary consumer biomass (20\% efficiency)};
            \node[visible on=<3->, state, name=pp, below of=c1] {1000 g/m\super{2}/year of
                primary producer biomass (NPP)};
            \node[visible on=<2->, state, name=sun, below of=pp] {$\approx$1\% of sunlight to
                GPP};
        
            % Connect States via edges
            \begin{uncoverenv}<3->
            \draw ($(sun.north)
                    + (0,0)
                $) 
                edge[stateEdge] node[edgeLabel,
                    % xshift=-3em
                ]{} 
                ($(pp.south)
                    + (0,0)
                $); 
            \end{uncoverenv}
            \begin{uncoverenv}<4->
            \draw ($(pp.north)
                    + (0,0)
                $) 
                edge[stateEdge] node[edgeLabel,
                    % xshift=-3em
                ]{} 
                ($(c1.south)
                    + (0,0)
                $); 
            \end{uncoverenv}
            \begin{uncoverenv}<5->
            \draw ($(c1.north)
                    + (0,0)
                $) 
                edge[stateEdge] node[edgeLabel,
                    % xshift=-3em
                ]{} 
                ($(c2.south)
                    + (0,0)
                $); 
            \end{uncoverenv}
            \begin{uncoverenv}<6->
            \draw ($(c2.north)
                    + (0,0)
                $) 
                edge[stateEdge] node[edgeLabel,
                    % xshift=-3em
                ]{} 
                ($(c3.south)
                    + (0,0)
                $); 
            \end{uncoverenv}
        \end{tikzpicture}
        \end{center}

    \end{adjustwidth}
\end{frame}

\end{document}

\clickerslide{
\begin{frame}
    \begin{clickerquestion}
        \item 
        \begin{clickeroptions}
            \item 
            \item 
            \item 
            \item 
        \end{clickeroptions}
    \end{clickerquestion}
\end{frame}
}

\clickerpost{
{
\usebackgroundtemplate{\includegraphics[page=17,width=\paperwidth]{./24-Radiation-extinction.pdf}}
\begin{frame}[t,plain]
    \begin{adjustwidth}{-2em}{-1.5em}
        \cmask{Answer: 3}
    \end{adjustwidth}
\end{frame}
}
}

