\documentclass[table]{beamer}
% \documentclass[table,handout]{beamer}
% \setbeameroption{show notes}
% \setbeameroption{hide notes}
% \setbeameroption{show only notes}
\usepackage{varwidth}

\newif\ifhide
\newif\ifpost
\newif\ifhideclicker

\hidetrue
% \hideclickertrue
% \posttrue

\newcommand{\whiteout}[1]{\textcolor{white}{#1}}
\newcommand{\whiteoutbox}[1]{\fcolorbox{white}{white}{\parbox{\dimexpr \linewidth-2\fboxsep-2\fboxrule}{\whiteout{#1}}}}
\newcommand{\notebox}[1]{\fcolorbox{blue}{white}{\parbox{\dimexpr \linewidth-2\fboxsep-2\fboxrule}{#1}}}

\ifhide%
    \newcommand{\hmask}[1]{\phantom{\varwidth{\linewidth}#1\endvarwidth}}%
\else%
    \newcommand{\hmask}[1]{#1}%
\fi

\ifhide%
    \newcommand{\hignore}[1]{}%
\else%
    \newcommand{\hignore}[1]{#1}%
\fi

\ifpost%
    \newcommand{\nopost}[1]{}%
\else%
    \newcommand{\nopost}[1]{#1}%
\fi

\ifhide%
    \newcommand{\hidebox}[1]{\phantom{\varwidth{\linewidth}#1\endvarwidth}}%
\else%
    \newcommand{\hidebox}[1]{\fbox{\parbox{\linewidth}{#1}}}%
\fi

\ifhide%
    \newcommand{\wbox}[1]{\whiteoutbox{#1}}%
\else%
    \newcommand{\wbox}[1]{\notebox{#1}}%
\fi

% \ifhide%
%     \newcommand{\clickeranswer}[1]{#1}%
% \else%
%     \newcommand{\clickeranswer}[1]{\textbf{\textcolor{blue}{#1}}}%
% \fi

\ifhideclicker%
    \newcommand{\clickeranswer}[1]{#1}%
\else%
    \ifhide%
        \newcommand{\clickeranswer}[1]{#1}%
    \else%
        \newcommand{\clickeranswer}[1]{\textbf{\textcolor{blue}{#1}}}%
    \fi
\fi

\input{../utils/slide-preamble2.tex}
\newcommand{\highlight}[1]{\textcolor{violet}{\textit{\textbf{#1}}}}
\newcommand{\super}[1]{\ensuremath{^{\textrm{#1}}}}
\newcommand{\sub}[1]{\ensuremath{_{\textrm{#1}}}}
\newcommand{\dC}{\ensuremath{^\circ{\textrm{C}}}}
\newcommand{\tb}{\hspace{2em}}
\providecommand{\e}[1]{\ensuremath{\times 10^{#1}}}
\newcommand{\myHangIndent}{\hangindent=5mm}

\makeatletter
\newcommand*{\rom}[1]{\expandafter\@slowromancap\romannumeral #1@}
\makeatother

\newcommand{\blankslide}{{\setbeamercolor{background canvas}{bg=black}
\setbeamercolor{whitetext}{fg=white}
\begin{frame}<handout:0>[plain]
\end{frame}}}

\newcommand{\whiteslide}{
\begin{frame}<handout:0>[plain]
\end{frame}}

\newcommand{\f}[1]{\ensuremath{F_{#1}}}

\bibliography{../bib/references}
\input{../utils/title-info.tex}

\title[Behavioral ecology \& sexual selection]{Behavioral ecology \& sexual
    selection}
% \date{\today}
\date{May 12, 2015}


% \setbeamertemplate{section in toc}[sections numbered]
% \setbeamertemplate{subsection in toc}[subsections numbered]

\begin{document}

\begin{noheadline}
\maketitle
\end{noheadline}


\nopost{
\begin{noheadline}
\begin{frame}[c]
    \vspace{-6mm}
    \begin{center} 
        \includegraphics[height=1.2\textheight]{../images/seating-chart-2.pdf}
    \end{center}
\end{frame}
\end{noheadline}
}

\clickerslide{
\begin{noheadline}
\begin{frame}
    \begin{clickerquestion}
        \item Recent data suggest that some interbreeding occurred between
            \textit{H.\ sapiens} and \textit{H.\ neanderthalensis}. Does this
            mean that they are actually the same species? 

        \begin{clickeroptions}
            \item Yes---they were not reproductively isolated so they were not
                evolutionary independent units in nature. 
            \item Yes---both \textit{H.\ sapiens} and Neanderthals had
                language, buried their dead, used tools, and had other aspects
                of advanced culture.
            \item \clickeranswer{No---some interbreeding is common, and they
                    are separate phylogenetic and morphological species.}
            \item No---they've always been considered separate species.
        \end{clickeroptions}
    \end{clickerquestion}
\end{frame}
\end{noheadline}
}

\begin{noheadline}
\begin{frame}
\frametitle{Today's issues:}
\vspace{5mm}
% \tableofcontents[subsectionstyle=hide]
\tableofcontents
\end{frame}
\end{noheadline}

\section[What are ecology and behavioral ecology?]{What are ecology and
    behavioral ecology?}

\begin{noheadline}
\begin{frame}[t]
    \frametitle{What are ecology and behavioral ecology?}
    \begin{adjustwidth}{-2em}{-1.5em}

        \begin{itemize}
            \item \textbf{Ecology:} The study of how organisms interact with
                their environment.

                \vspace{4mm}
            \item \textbf{Behavior:} The study of what organisms do, how they
                do it (in terms of genetic/neuronal/hormonal mechanisms,
                and why (in terms of fitness).

                \vspace{4mm}
            \item \textbf{Behavioral ecology:} The study of how organisms
                make decisions when they interact with various aspects of
                their environment.
        \end{itemize}
    \end{adjustwidth}
\end{frame}
\end{noheadline}

\begin{frame}[t]
    \begin{adjustwidth}{-2em}{-1.5em}
        Questions in behavioral ecology:

        \begin{itemize}
            \item What should I eat?

                \vspace{4mm}
            \item Where should I live?

                \vspace{4mm}
            \item How should I communicate?

                \vspace{4mm}
            \item Who should I mate with?

                \vspace{4mm}
            \item When should I cooperate?
        \end{itemize}
    \end{adjustwidth}
\end{frame}

\section[What is sexual selection?]{What is sexual selection?}

\begin{noheadline}
\begin{frame}[t]
    \frametitle{What is sexual selection?}
    \begin{adjustwidth}{-2em}{-1.5em}
        (background for addressing ``who should I mate with?'')

        \vspace{3mm}
        Darwin wanted to explain why, in some species, males look different than
        females.
        \begin{enumerate}
            \item There is heritable variation in appearance and/or courtship
                behavior.

                \vspace{5mm}
            \item Individuals experience differential success in obtaining
                mates---individuals with certain traits do better.
        \end{enumerate}
    \end{adjustwidth}
\end{frame}
\end{noheadline}

\clickerslide{
\begin{frame}
    \begin{clickerquestion}
        \item What is the essence of evolution by sexual selection? 

        \begin{clickeroptions}
            \item Eggs are expensive; sperm are cheap. 
            \item Males fight; females choose. 
            \item \clickeranswer{Alleles that lead to increased success in
                    mating increase in frequency.}
            \item In most environments, sexually reproduced offspring have
                higher fitness than asexually reproduced offspring.
        \end{clickeroptions}
    \end{clickerquestion}
\end{frame}
}

\begin{frame}[t]
    \begin{adjustwidth}{-2em}{-1.5em}
        In many species, females invest MUCH more in offspring than males do.
        For example:

        \vspace{4mm}
        Female red deer average 145 kg.\, and are pregnant for 6 months over
        the winter. Calves average 10 kg at birth and nurse for two months;
        they weigh about 55 kg at weaning.

        \vspace{8mm}
        Males average 200 kg, they contribute

    \end{adjustwidth}
\end{frame}

\clickerslide{
\begin{frame}
    \begin{clickerquestion}
        \item Which of the following statements is correct? 

        \begin{clickeroptions}
            \item In most species, males are larger than females. 
            \item \clickeranswer{In most species, female RS is limited by
                    access to the resources required to breed.}
            \item In most species, male RS is limited by the ability to produce
                large amounts of sperm. 
            \item In most species, female RS is limited by the ability to
                obtain mates. 
        \end{clickeroptions}
    \end{clickerquestion}
\end{frame}
}

\section[How does sexual selection act when males compete for mates?]{How does
    sexual selection act when males compete for mates?}

\begin{noheadline}
\begin{frame}[t]
    \frametitle{How does sexual selection act when males compete for mates?}
    \begin{adjustwidth}{-2em}{-1.5em}
        \vspace{-2mm}
        In red deer, intense male-male competition occurs.

        \begin{columns}[t]

        \column{0.44\linewidth}

        \begin{enumerate}
            \item Which sex has higher variation in RS?

                \nbox{\scriptsize Males---note that modes are about the same,
                    but that males have a much wider distribution (= greater
                    variance).}

            \item In which sex would alleles associated with increased mating
                success increase faster?

                \nbox{\scriptsize Males---successful alleles can be passed on
                    to 20 or more offspring}
        \end{enumerate}

        \column{0.55\linewidth}

        \includegraphics[width=\columnwidth]{deer-rs.jpg}

        \end{columns}
    \end{adjustwidth}
\end{frame}
\end{noheadline}

\begin{frame}[t]
    \begin{adjustwidth}{-2em}{-1.5em}
        Observation: In some human cultures, men control access to jobs and
        other resources required to start a family, and marriages are arranged.
        As a result, females have little or no ability to choose their mates. 

        \vspace{5mm}
        Question: In these cultures, how might men react to efforts to (1)
        increase the education level of women, or (2) help women start
        businesses through micro-lending or other programs? 

        \nbox{This is a discussion/opinion question \ldots but, answers should
            acknowledge that access to resources should increase the power that
            women have to choose their own males.}

    \end{adjustwidth}
\end{frame}

\begin{frame}[t]
    \begin{adjustwidth}{-2em}{-1.5em}
        Observations:

        \begin{itemize}
            \item In the U.S., there is a strong correlation between
                level of educational attainment and lifetime income.
            
            \item Traditionally, men have been the family ``breadwinner.''

            \item Currently, 57\% of U.S.\ college students are female.
        \end{itemize}

        Question: In this culture, how might men and women react to the new
        asymmetry in level of education?

        \nbox{This is a discussion/opinion question \ldots but, answers should
            acknowledge that traditional roles may reverse or the correlation
            between education and income may break down.}

    \end{adjustwidth}
\end{frame}

\begin{frame}[t]
    \begin{adjustwidth}{-2em}{-1.5em}
        Observation: In China in 1996, there were 121 boys aged 1--4 for every
        100 girls aged 1--4. This cohort is almost of marriageable age.

        \begin{itemize}
            \item How likely is it that all men in this cohort will
                be able to marry and have a family?
            
                \vspace{4mm}
            \item Will this situation make men happy or crabby?

                \vspace{4mm}
            \item Are they likely to remain celibate all their lives?
        \end{itemize}

        \nbox{This is a discussion/opinion question \ldots but, answers should
            acknowledge that unbalanced sex ratios may have far-ranging social
            consequences.}

    \end{adjustwidth}
\end{frame}

\section[How does sexual selection act when females choose mates?]{How does
    sexual selection act when females choose mates?}

\begin{frame}[t]
    \begin{adjustwidth}{-2em}{-1.5em}
        \vspace{-3mm}
        How does sexual selection act when females choose mates?

        \vspace{2mm}
        What do females choose?

        \begin{enumerate}
            \item ``Good alleles''
        \end{enumerate}

        \begin{columns}[t]

        \column{0.39\linewidth}

        Why is this male's plumage and display an ``honest'' advertisement of
        genetic quality?

        \nbox{\small If the male is sick or weakened by poor nutrition, his
            feathers would not be as bright, and he would not be able to
            display very well.}
        \nbox{\small NOTE: male birds of paradise do not help rear the young}

        \column{0.6\linewidth}
        
        \includegraphics[width=\columnwidth]{bird-of-paradise.jpg}

        \end{columns}

    \end{adjustwidth}
\end{frame}

\begin{frame}[t]
    \begin{adjustwidth}{-2em}{-1.5em}
        \vspace{-3mm}
        How does sexual selection act when females choose mates?

        \vspace{2mm}
        What do females choose?

        \begin{enumerate}
            \addtocounter{enumi}{1}
            \item Resources required to produce offspring

                \vspace{1mm}
            E.g., Sexual selection in Australian redback spiders.
        \end{enumerate}

        \begin{columns}[t]

        \column{0.42\linewidth}

        \vspace{-5mm}
        \begin{itemize}
                \small
            \item During copulation, male redbacks insert a sperm transfer
                organ into the female.

            \item Then they do a somersault, which places their abdomen in
                front of the female's mouthparts.

            \item In many cases, the female proceeds to eat the male.
        \end{itemize}

        \column{0.57\linewidth}
        
        \includegraphics[width=\columnwidth]{spider.jpg}

        \end{columns}

    \end{adjustwidth}
\end{frame}

\begin{frame}[t]
    \begin{adjustwidth}{-2em}{-1.5em}
        Hypothesis: Copulation lasts longer and more sperm are transferred if
        males are eaten. 

        \vspace{4mm}
        Prediction: RS of cannibalized males is higher than RS of
        non-cannibalized males.

        \vspace{4mm}
        A test: Allow female redbacks to mate with two males.  Document the
        proportion of young sired by each male. 
    \end{adjustwidth}
\end{frame}

\end{document}

\clickerslide{
\begin{frame}
    \begin{clickerquestion}
        \item 
        \begin{clickeroptions}
            \item 
            \item 
            \item 
            \item 
        \end{clickeroptions}
    \end{clickerquestion}
\end{frame}
}

\clickerpost{
{
\usebackgroundtemplate{\includegraphics[page=17,width=\paperwidth]{./24-Radiation-extinction.pdf}}
\begin{frame}[t,plain]
    \begin{adjustwidth}{-2em}{-1.5em}
        \cmask{Answer: 3}
    \end{adjustwidth}
\end{frame}
}
}

