\documentclass[table]{beamer}
% \documentclass[table,handout]{beamer}
% \setbeameroption{show notes}
% \setbeameroption{hide notes}
% \setbeameroption{show only notes}
\usepackage{varwidth}

\newif\ifhide
\newif\ifpost
\newif\ifhideclicker

\hidetrue
% \hideclickertrue
% \posttrue

\newcommand{\whiteout}[1]{\textcolor{white}{#1}}
\newcommand{\whiteoutbox}[1]{\fcolorbox{white}{white}{\parbox{\dimexpr \linewidth-2\fboxsep-2\fboxrule}{\whiteout{#1}}}}
\newcommand{\notebox}[1]{\fcolorbox{blue}{white}{\parbox{\dimexpr \linewidth-2\fboxsep-2\fboxrule}{#1}}}

\ifhide%
    \newcommand{\hmask}[1]{\phantom{\varwidth{\linewidth}#1\endvarwidth}}%
\else%
    \newcommand{\hmask}[1]{#1}%
\fi

\ifhide%
    \newcommand{\hignore}[1]{}%
\else%
    \newcommand{\hignore}[1]{#1}%
\fi

\ifpost%
    \newcommand{\nopost}[1]{}%
\else%
    \newcommand{\nopost}[1]{#1}%
\fi

\ifhide%
    \newcommand{\hidebox}[1]{\phantom{\varwidth{\linewidth}#1\endvarwidth}}%
\else%
    \newcommand{\hidebox}[1]{\fbox{\parbox{\linewidth}{#1}}}%
\fi

\ifhide%
    \newcommand{\wbox}[1]{\whiteoutbox{#1}}%
\else%
    \newcommand{\wbox}[1]{\notebox{#1}}%
\fi

% \ifhide%
%     \newcommand{\clickeranswer}[1]{#1}%
% \else%
%     \newcommand{\clickeranswer}[1]{\textbf{\textcolor{blue}{#1}}}%
% \fi

\ifhideclicker%
    \newcommand{\clickeranswer}[1]{#1}%
\else%
    \ifhide%
        \newcommand{\clickeranswer}[1]{#1}%
    \else%
        \newcommand{\clickeranswer}[1]{\textbf{\textcolor{blue}{#1}}}%
    \fi
\fi

\input{../utils/slide-preamble2.tex}
\newcommand{\highlight}[1]{\textcolor{violet}{\textit{\textbf{#1}}}}
\newcommand{\super}[1]{\ensuremath{^{\textrm{#1}}}}
\newcommand{\sub}[1]{\ensuremath{_{\textrm{#1}}}}
\newcommand{\dC}{\ensuremath{^\circ{\textrm{C}}}}
\newcommand{\tb}{\hspace{2em}}
\providecommand{\e}[1]{\ensuremath{\times 10^{#1}}}
\newcommand{\myHangIndent}{\hangindent=5mm}

\makeatletter
\newcommand*{\rom}[1]{\expandafter\@slowromancap\romannumeral #1@}
\makeatother

\newcommand{\blankslide}{{\setbeamercolor{background canvas}{bg=black}
\setbeamercolor{whitetext}{fg=white}
\begin{frame}<handout:0>[plain]
\end{frame}}}

\newcommand{\whiteslide}{
\begin{frame}<handout:0>[plain]
\end{frame}}

\newcommand{\f}[1]{\ensuremath{F_{#1}}}

\bibliography{../bib/references}
\input{../utils/title-info.tex}

\title[History of life: Radiations \& extinctions]{History of life:
    Radiations \& extinctions}
% \date{\today}
\date{May 5, 2015}

\begin{document}

\begin{noheadline}
\maketitle
\end{noheadline}

\nopost{
\begin{noheadline}
\begin{frame}[c]
    \vspace{-6mm}
    \begin{center} 
        \includegraphics[height=1.2\textheight]{../images/seating-chart-2.pdf}
    \end{center}
\end{frame}
\end{noheadline}
}

\clickerslide{
\begin{frame}
    \begin{clickerquestion}
        \item Why is it correct to claim that data from relative dating,
            absolute dating, structural homology, phylogenies, and development
            of living species represent independent tests of evolutionary
            hypotheses? 

        \begin{clickeroptions}
            \item Different teams of biologists work independently on each
                type of data.
            \item The data are graphed on the x-axis as the independent
                variable; the evolutionary response is on the y-axis. 
            \item \clickeranswer{The data are from different sources; each
                    source is produced by different processes and can conflict
                    with the hypothesis.}
            \item The analyses are done independently---often separated by
                years---and published in different journals.
        \end{clickeroptions}
    \end{clickerquestion}
\end{frame}
}

\begin{noheadline}
\begin{frame}
\frametitle{Today's issues:}
\vspace{5mm}
\tableofcontents[subsectionstyle=hide]
% \tableofcontents
\end{frame}
\end{noheadline}

\section[Why do adaptive radiations occur?]{Why do adaptive radiations occur?}

\begin{noheadline}
\begin{frame}
    \frametitle{Why do adaptive radiations occur?}
    \begin{adjustwidth}{-2em}{-1.5em}
        \begin{columns}
            \column{0.55\linewidth}

            \begin{description}
                \item[Adaptive radiation]
                    The \highlight{rapid} diversification of a single lineage
                    into an \highlight{array} of species that fill a
                    \highlight{wide variety} of ecological niches.

                \vspace{5mm}
                \item[Niche]
                    The range of habitats or resources used by a species (a way
                    of making a living).
            \end{description}

            \column{0.45\linewidth}
            
            \begin{center}
                \includegraphics[width=\columnwidth]{radiation-tree.png}
            \end{center}

        \end{columns}
    \end{adjustwidth}
\end{frame}
\end{noheadline}

{
\usebackgroundtemplate{\includegraphics[page=4,width=\paperwidth]{./24-Radiation-extinction.pdf}}
\begin{frame}[t,plain]
    \begin{adjustwidth}{-2em}{-1.5em}
        % \cmask{Answer: 5}
    \end{adjustwidth}
\end{frame}
}

\begin{frame}[t]
    \frametitle{Two classical explanations (hypotheses)}
    \begin{adjustwidth}{-2em}{-1.5em}

        \begin{enumerate}
            \item Ecological opportunity (resource availability; ``empty
                niches'')
            \uncover<2->{
            \begin{enumerate}
                \item Competitors wiped out

                    \nbox{Resources that were formerly being used by competing
                        species are now available to exploit. E.g., mammal
                        radiation after dinosaur extinction.}

                \vspace{1cm}
                \item Colonize a new region

                    \nbox{No other species around to use resources (no
                        competition). E.g., Hawaiian honeycreepers}
            \end{enumerate}
            }
        \end{enumerate}
    \end{adjustwidth}
\end{frame}

\begin{frame}[t]
    \frametitle{Two classical explanations (hypotheses)}
    \begin{adjustwidth}{-2em}{-1.5em}

        \begin{enumerate}
            \addtocounter{enumi}{1}

            \item Morphological innovation

            \begin{enumerate}

                \item An ``adaptive breakthrough''---a morphological trait that
                    makes it possible to exploit resources in a new way or
                    occupy new niches. E.g., adhesive toepads in geckos
                    ($\approx$2000 species!).

            \end{enumerate}
        \end{enumerate}
    \end{adjustwidth}
\end{frame}

\begin{frame}[t]
    \begin{adjustwidth}{-2em}{-1.5em}
        Issue: These hypotheses can be difficult to test---correlation or
        causation?

        \vspace{3mm}
        E.g., Cambrian explosion:

        \uncover<2->{
        \begin{enumerate}
            \item High oxygen hypothesis (using oxygen allows cells to extract
                much more energy from a molecule of sugar)

                \nbox{More oxygen would make it possible for species to grow
                    larger and have higher rates of activity. But, did higher
                    oxygen availability actually cause the radiation, or was it
                    something else and oxygen just happened to increase at the
                    same time?}

            \item New genes, new bodies hypothesis (genome duplications; more
                genetic information available)

                \nbox{The availability of additional genetic material could
                    allow some duplicated gene copies to evolve new functions,
                    but again, it's not clear that the gene duplications
                    actually caused the radiation.}
        \end{enumerate}
        }
    \end{adjustwidth}
    \note[item]{Ask to think about the logic behind these hypotheses, AND also
        why is it difficult to prove causation}
\end{frame}

\clickerslide{
\begin{frame}
    \begin{clickerquestion}
        \item Which of the following statements is correct? 

        \begin{clickeroptions}
            \item Causative variables produce positive correlations (variables
                increase together).
            \item Causative variables produce negative correlations (causative
                variable increases; response variable decreases)
            \item Variables plotted on the x-axis (explanatory variables) cause
                the observed changes in the response variable. 
            \item \clickeranswer{Variables may be correlated but have no
                    cause-effect relationship.}
        \end{clickeroptions}
    \end{clickerquestion}
    \note[item]{Does plastic cause obesity and type 2 diabetes? Endocrine disruptors leaching from plastics and mimicking hormones.}
\end{frame}
}

{
\usebackgroundtemplate{\includegraphics[page=9,width=\paperwidth]{./24-Radiation-extinction.pdf}}
\begin{frame}[t,plain]
    \begin{adjustwidth}{-2em}{-1.5em}
        % \cmask{Answer: 5}
    \end{adjustwidth}
    \note[item]{Icteridae: $\approx$100 species. Many different niches and
        mating strategies---e.g., brood parasites (cowbird)}
\end{frame}
}

{
\usebackgroundtemplate{\includegraphics[page=10,width=\paperwidth]{./24-Radiation-extinction.pdf}}
\begin{frame}[t,plain]
    \begin{adjustwidth}{-2em}{-1.5em}
        % \cmask{Answer: 5}
    \end{adjustwidth}
    \note[item]{Long processus retroarticulatus allows more muscle
        attachment---blackbirds can use bill as lever to break stuff open.}
\end{frame}
}

{
\usebackgroundtemplate{\includegraphics[page=11,width=\paperwidth]{./24-Radiation-extinction.pdf}}
\begin{frame}[t,plain]
    \begin{adjustwidth}{-2em}{-1.5em}
        % \cmask{Answer: 5}
    \end{adjustwidth}
    \note[item]{Gaping (a long processus retroarticulatus) should have evolved
        at the base of the radiation---map this by adding an outgroup that
        doesn't gape.}
\end{frame}
}

\section[Why do mass extinctions occur?]{Why do mass extinctions occur?}

\begin{noheadline}
\begin{frame}[t]
    \frametitle{Why do mass extinctions occur?}
    \begin{adjustwidth}{-2em}{-1.5em}

        \begin{description}
            \item<1->[Background extinction]
                Normal extinction rates, due to normal rates of
                environmental change (environment includes competitors)

            \vspace{15mm}
            \item<2->[Mass extinction]
                At least 60\% of species present go extinct in less than 1
                million years; due to extreme rates of environmental
                change.
        \end{description}

        \vspace{15mm}
        \uncover<3->{
        Mass extinctions function like a ``species-level bottleneck event''
        (another reason evolution is NOT progressive).
        }

    \end{adjustwidth}
\end{frame}
\end{noheadline}

\begin{frame}[t]
    \begin{adjustwidth}{-2em}{-1.5em}
        During the Permian-Triassic event, 250 million years ago, over 90\% of
        existing species were wiped out.

        \uncover<2->{
        \vspace{4mm}
        \textbf{Impact hypothesis:} An asteroid hit the Earth and initiated
        environmental changes that extinguished most species.

        \vspace{4mm}
        \textbf{Killing mechanisms:}
            
            \nbox{Wildfires, global cooling due to soot/ash/dust, tsunamis,
                changes in ocean currents and chemistry.}
        }
    \end{adjustwidth}
\end{frame}

\begin{frame}[t]
    \begin{adjustwidth}{-2em}{-1.5em}
        During the Permian-Triassic event, 250 million years ago, over 90\% of
        existing species were wiped out.

        \vspace{4mm}
        \textbf{World-went-to-hell hypothesis:}

        \uncover<2->{
        \begin{enumerate}
            \item Global warming (high $CO_2$, high $CH_4$)

            \item Anoxic conditions in the ocean and atmosphere
        \end{enumerate}
        \vspace{4mm}
        Key observation: Siberian traps date to 250 mya

        % \nbox{Go Dawgs! Lack of oxygen would make only low elevation habitats
        %     reasonable for oxygen breathing species.}
        \vspace{5mm}

        \textbf{Killing mechanisms:}

            \nbox{1. If change in temperatures is rapid enough, then there
                would not be enough time for populations to adapt.  2.
                Oxygen-requiring organisms are out of luck.}
        }
    \end{adjustwidth}
    \note[item]{Siberian traps---huge basalt flow---one of largest volcanic
        events in Earth's history coincided with this mass extinction.}
\end{frame}

\begin{frame}[t]
    \begin{adjustwidth}{-2em}{-1.5em}
        Predictions

        \vspace{2mm}
        \textbf{Impact hypothesis:}

            \nbox{Should see microtektites, shocked quartz, crater, soot/ash,
                iridium, etc. \ldots Not a lot of evidence of these.}

        \vspace{13mm}
        \textbf{World-went-to-hell hypothesis:}

            \nbox{High $CO_2$/$CH_4$ and low $O_2$ signatures in rocks}

        \uncover<2->{
        \vspace{14mm}
        We know that anoxia and $CO_2$ occurred, but what \highlight{CAUSED}
        changes in atmospheric and oceanic chemistry? And, why did they
        reverse?
        \nbox{These are still open questions}
        }
    \end{adjustwidth}
\end{frame}

\clickerpost{
{
\usebackgroundtemplate{\includegraphics[page=17,width=\paperwidth]{./24-Radiation-extinction.pdf}}
\begin{frame}[t,plain]
    \begin{adjustwidth}{-2em}{-1.5em}
        \cmask{Answer: 3}
    \end{adjustwidth}
\end{frame}
}
}

\clickerslide{
\begin{frame}
    \begin{clickerquestion}
        \item In terms of its impact on the species present before and after, a
            mass extinction event is most similar to \ldots
 
        \begin{clickeroptions}
            \item Directional selection
            \item Stabilizing selection
            \item Disruptive selection
            \item Mutation
            \item \clickeranswer{Genetic drift}
            \item Gene flow
        \end{clickeroptions}
    \end{clickerquestion}
\end{frame}
}

\end{document}

\clickerslide{
\begin{frame}
    \begin{clickerquestion}
        \item 
        \begin{clickeroptions}
            \item 
            \item 
            \item 
            \item 
        \end{clickeroptions}
    \end{clickerquestion}
\end{frame}
}
