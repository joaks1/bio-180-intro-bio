\documentclass[table]{beamer}
% \documentclass[table,handout]{beamer}
% \setbeameroption{show notes}
% \setbeameroption{hide notes}
% \setbeameroption{show only notes}
\usepackage{varwidth}

\newif\ifhide
\newif\ifpost
\newif\ifhideclicker

\hidetrue
% \hideclickertrue
% \posttrue

\newcommand{\whiteout}[1]{\textcolor{white}{#1}}
\newcommand{\whiteoutbox}[1]{\fcolorbox{white}{white}{\parbox{\dimexpr \linewidth-2\fboxsep-2\fboxrule}{\whiteout{#1}}}}
\newcommand{\notebox}[1]{\fcolorbox{blue}{white}{\parbox{\dimexpr \linewidth-2\fboxsep-2\fboxrule}{#1}}}

\ifhide%
    \newcommand{\hmask}[1]{\phantom{\varwidth{\linewidth}#1\endvarwidth}}%
\else%
    \newcommand{\hmask}[1]{#1}%
\fi

\ifhide%
    \newcommand{\hignore}[1]{}%
\else%
    \newcommand{\hignore}[1]{#1}%
\fi

\ifpost%
    \newcommand{\nopost}[1]{}%
\else%
    \newcommand{\nopost}[1]{#1}%
\fi

\ifhide%
    \newcommand{\hidebox}[1]{\phantom{\varwidth{\linewidth}#1\endvarwidth}}%
\else%
    \newcommand{\hidebox}[1]{\fbox{\parbox{\linewidth}{#1}}}%
\fi

\ifhide%
    \newcommand{\wbox}[1]{\whiteoutbox{#1}}%
\else%
    \newcommand{\wbox}[1]{\notebox{#1}}%
\fi

% \ifhide%
%     \newcommand{\clickeranswer}[1]{#1}%
% \else%
%     \newcommand{\clickeranswer}[1]{\textbf{\textcolor{blue}{#1}}}%
% \fi

\ifhideclicker%
    \newcommand{\clickeranswer}[1]{#1}%
\else%
    \ifhide%
        \newcommand{\clickeranswer}[1]{#1}%
    \else%
        \newcommand{\clickeranswer}[1]{\textbf{\textcolor{blue}{#1}}}%
    \fi
\fi

\input{../utils/slide-preamble2.tex}
\newcommand{\highlight}[1]{\textcolor{violet}{\textit{\textbf{#1}}}}
\newcommand{\super}[1]{\ensuremath{^{\textrm{#1}}}}
\newcommand{\sub}[1]{\ensuremath{_{\textrm{#1}}}}
\newcommand{\dC}{\ensuremath{^\circ{\textrm{C}}}}
\newcommand{\tb}{\hspace{2em}}
\providecommand{\e}[1]{\ensuremath{\times 10^{#1}}}
\newcommand{\myHangIndent}{\hangindent=5mm}

\makeatletter
\newcommand*{\rom}[1]{\expandafter\@slowromancap\romannumeral #1@}
\makeatother

\newcommand{\blankslide}{{\setbeamercolor{background canvas}{bg=black}
\setbeamercolor{whitetext}{fg=white}
\begin{frame}<handout:0>[plain]
\end{frame}}}

\newcommand{\whiteslide}{
\begin{frame}<handout:0>[plain]
\end{frame}}

\newcommand{\f}[1]{\ensuremath{F_{#1}}}

\bibliography{../bib/references}
\input{../utils/title-info.tex}

\title{Professional development III: Research}

% \date{\today}
\date{June 4, 2015}

% \setbeamertemplate{section in toc}[sections numbered]
% \setbeamertemplate{subsection in toc}[subsections numbered]

\begin{document}

\begin{noheadline}
\maketitle
\end{noheadline}

\nopost{
\begin{noheadline}
\begin{frame}[c]
    \vspace{-6mm}
    \begin{center} 
        \includegraphics[height=1.2\textheight]{../images/seating-chart-2.pdf}
    \end{center}
\end{frame}
\end{noheadline}
}

\clickerslide{
\begin{frame}
    \begin{clickerquestion}
        \item How does the divorce rate impact an ecological footprint? 

        \begin{clickeroptions}
            \item It doesn't---this is a social and ethical issue.
            \item It lowers it, because divorce reduces spending power and thus
                consumption levels.
            \item \clickeranswer{It increases it, because now there are two
                    households instead of one.}
            \item Minimal impact---the biggest factor for footprints in
                developed countries is use of oil.
        \end{clickeroptions}
    \end{clickerquestion}
\end{frame}
}

\clickerslide{
\begin{frame}
    \begin{clickerquestion}
        \item If humans are responsible for a 6\super{th} mass extinction, why
            should we care? 

        \begin{clickeroptions}
            \item As oil and other non-renewable resources are used up, human
                quality of life is likely to decrease.      
            \item We have an ethical obligation.
            \item We don't need to---technological innovations insulate humans
                from the rest of the biota. 
            \item If species richness continues to decline, productivity,
                resistance, resilience, and other ecosystem services will
                decline.
            \item Most people don't want to live in a world where biodiversity
                consists of rats, cockroaches, and dandelions.
        \end{clickeroptions}
    \end{clickerquestion}
\end{frame}
}


\begin{noheadline}
\begin{frame}
\frametitle{Today's issues:}
% \vspace{5mm}
% \tableofcontents[subsectionstyle=hide]
\tableofcontents
\end{frame}
\end{noheadline}

\section{Why do undergraduate research?}

\begin{frame}[t]
    \begin{adjustwidth}{-2em}{-1.5em}
        Why do undergraduate research?

        \vspace{2mm}
        Research experience is strongly recommended for professional school,
        and essentially required for graduate school. Why?

        \vspace{2mm}
        \nbox{\scriptsize For professions that depend on new knowledge, you
            will understand where that knowledge comes from (the research
            process).}
        \nbox{\scriptsize Professional and graduate students do research (and
            all faculty do research); you need to show that you know what
            you're getting into.}
        \nbox{\scriptsize You refine your higher order cognitive skills
            (Blooms!): learn to read/interpret/analyze papers, collaborate with
            research groups, collect/organize/analyze/interpret data, solve
            problems.}
        \nbox{\scriptsize It shows that you are independent and intellectually
            curious enough to develop, and follow through on, your own research
            project.}

        \barefootnote{\footnotesize Research shows that undergraduate research
            experience leads to better retention in STEM majors and higher
            likelihood of entering graduate/professional school.}
    \end{adjustwidth}
\end{frame}

\section{How are labs (research groups) organized?}

\begin{frame}[t]
    \begin{adjustwidth}{-2em}{-1.5em}
        % \vspace{-3mm}
        How are labs (research groups) organized?

        \vspace{2mm}
        Jobs and roles in academic labs

        \begin{itemize}
            \item PI:

                \nbox{\tiny Principal investigator (``the boss''). Writes
                    research grants to (hopefully) make sure there is money to
                    do research and pay everyone listed below.}

            % \vspace{4mm}
            \item Lab manager/tech:

                \nbox{\tiny Oversee day-to-day operations of lab. Make
                    sure lab consumables are in stock, equipment is clean and
                    working correctly, and lab reagents/buffers/solutions are
                    available.}

            % \vspace{2mm}
            \item Post-docs:

                \nbox{\tiny Have a Ph.D. Short-term (1--3 year) research
                    positions.}

            % \vspace{4mm}
            \item Graduate students:
                
                \nbox{\tiny Working on Master's or Ph.D. As undergrad,
                    you will probably work with them (often will be your
                    primary mentor).}

            % \vspace{4mm}
            \item Undergrad researchers:

                \nbox{\tiny Develop and work on your own project (perhaps
                    get undergraduate research grant).}

            % \vspace{4mm}
            \item Undergrad/hourly helpers:

                \nbox{\tiny You work under a mentor; learn basic
                    techniques in the field of study. Often assigned to
                    relatively simple, low-risk tasks; mentor can determine if
                    you are reliable and serious.}
        \end{itemize}
    \end{adjustwidth}
\end{frame}

\clickerslide{
\begin{frame}
    How is research (and university growth) funded?

    \vspace{2mm}
    \begin{clickerquestion}
        \item Last year UW researchers won \$1.24 billion grants. Of this,
            \$816 million came from the federal government. On federal grants,
            the UW charges 54\% as ``indirect cost recovery'' (IDCR). The \$816
            million represents money used directly for research ($X$)
            \textbf{AND} the IDCR.  How much IDCR did UW receive last year?

        \begin{clickeroptions}
            \item IDCR total = 1,240,000,000 - 0.54(816,000,000) = 799,360,000
            \item \clickeranswer{1.54$X$ = 816,000,000; $X$ = 529,870,130; IDCR
                    total = 816,000,000 - 529,870,130 = 286,129,870}
            \item IDCR total = 0.54(816,000,000) = 440,640,000
            \item IDCR total = 1,240,000,000 - 816,000,000 = 424,000,000
        \end{clickeroptions}
    \end{clickerquestion}
\end{frame}
}


\section{How should you look for a position?}

% \subsection{Resources for you}

\begin{frame}
    \footnotesize
    \begin{adjustwidth}{0em}{0em}
        \vspace{-1mm}
        Finding opportunities:
        \begin{itemize}
            \item \href{http://www.biology.washington.edu/research/undergraduate}{Helpful advice from Bio Department}
            \item \href{http://www.washington.edu/undergradresearch/students/find/}{UW Undergraduate Research Program}
            \item \href{http://www.biology.washington.edu/research}{Biology faculty interests}
            \item \href{http://mailman12.u.washington.edu/mailman/listinfo/biostudent}{Biology student e-mail list serve}
            \item \href{http://depts.washington.edu/uwmcnair/}{UW McNair Achievement Program}
            \item \href{http://depts.washington.edu/fhl/ResInts2013.html}{Friday Harbor Lab faculty interests}
            \item \href{http://depts.washington.edu/fhl/stu_index.html}{Friday Harbor Lab Research Apprenticeships}
            \item \href{http://www.washington.edu/undergradresearch/symposium/}{UW Undergraduate Research Symposium}
        \end{itemize}

        Finding funding:
        \begin{itemize}
            \item \href{http://www.washington.edu/undergradresearch/students/funding/}{Undergraduate research funding}
            \item \href{http://expd.washington.edu/mge/apply/research/index.htm}{Mary Gates Research Scholarship}
        \end{itemize}

        Opportunites beyond UW:
        \begin{itemize}
            \item \href{http://www.fredhutch.org/en/education-training.html}{Fred Hutch Cancer Research Center}
            \item \href{http://www.seattlecca.org/}{Seattle Cancer Care Alliance}
            \item \href{http://www.seattlechildrens.org/}{Seattle Childrens}
            \item \href{http://www.waspacegrant.org/for_students/student_internships/wsgc_internships/SURP_for_students.html}{Washington NASA Space Grant Summer Research Program}
            \item \href{http://www.systemsbiology.org/}{Institute for Systems Biology (Seattle)}
            \item \href{http://alleninstitute.org/}{Allen Institute for Brain Science}
            \item \href{http://www.seattlegenetics.com/}{Seattle Genetics}
            \item \href{https://www.benaroyaresearch.org/}{Benaroya Research Institute}
        \end{itemize}
    \end{adjustwidth}
\end{frame}

\begin{frame}[t]
    \begin{adjustwidth}{-2em}{-1.5em}
        How should you look for a position?

        \begin{itemize}
            \item Identify a question or area of interest

            \vspace{4mm}
            \item Do your research on labs that match your interests (web,
                networking)
            
            \vspace{4mm}
            \item READ PAPERS!

            \vspace{4mm}
            \item Write to the PI

            \vspace{4mm}
            \item If no response: Take a class from PI (if you can), research
                other labs/PIs, write to post-docs or grad students, try again
                when you have more experience.

            \vspace{4mm}
            \item Keep working at it! (Be a Dawg!)
        \end{itemize}
    \end{adjustwidth}
\end{frame}

\section{What are your goals once you have a position?}

\begin{frame}[t]
    \begin{adjustwidth}{-2em}{-1.5em}
        What are your goals once you have a position?

        \nbox{Attend a scientific conference and present your research as a
            talk or poster.}
        
        \nbox{Publish your research as a paper(s) in a scientific journal.}

    \end{adjustwidth}
\end{frame}

\whiteslide

\end{document}

\clickerslide{
\begin{frame}
    \begin{clickerquestion}
        \item 
        \begin{clickeroptions}
            \item 
            \item 
            \item 
            \item 
        \end{clickeroptions}
    \end{clickerquestion}
\end{frame}
}

\clickerpost{
{
\usebackgroundtemplate{\includegraphics[page=17,width=\paperwidth]{./24-Radiation-extinction.pdf}}
\begin{frame}[t,plain]
    \begin{adjustwidth}{-2em}{-1.5em}
        \cmask{Answer: 3}
    \end{adjustwidth}
\end{frame}
}
}

