\documentclass[table]{beamer}
% \documentclass[table,handout]{beamer}
% \setbeameroption{show notes}
% \setbeameroption{hide notes}
% \setbeameroption{show only notes}
\usepackage{varwidth}

\newif\ifhide
\newif\ifpost
\newif\ifhideclicker

\hidetrue
% \hideclickertrue
% \posttrue

\newcommand{\whiteout}[1]{\textcolor{white}{#1}}
\newcommand{\whiteoutbox}[1]{\fcolorbox{white}{white}{\parbox{\dimexpr \linewidth-2\fboxsep-2\fboxrule}{\whiteout{#1}}}}
\newcommand{\notebox}[1]{\fcolorbox{blue}{white}{\parbox{\dimexpr \linewidth-2\fboxsep-2\fboxrule}{#1}}}

\ifhide%
    \newcommand{\hmask}[1]{\phantom{\varwidth{\linewidth}#1\endvarwidth}}%
\else%
    \newcommand{\hmask}[1]{#1}%
\fi

\ifhide%
    \newcommand{\hignore}[1]{}%
\else%
    \newcommand{\hignore}[1]{#1}%
\fi

\ifpost%
    \newcommand{\nopost}[1]{}%
\else%
    \newcommand{\nopost}[1]{#1}%
\fi

\ifhide%
    \newcommand{\hidebox}[1]{\phantom{\varwidth{\linewidth}#1\endvarwidth}}%
\else%
    \newcommand{\hidebox}[1]{\fbox{\parbox{\linewidth}{#1}}}%
\fi

\ifhide%
    \newcommand{\wbox}[1]{\whiteoutbox{#1}}%
\else%
    \newcommand{\wbox}[1]{\notebox{#1}}%
\fi

% \ifhide%
%     \newcommand{\clickeranswer}[1]{#1}%
% \else%
%     \newcommand{\clickeranswer}[1]{\textbf{\textcolor{blue}{#1}}}%
% \fi

\ifhideclicker%
    \newcommand{\clickeranswer}[1]{#1}%
\else%
    \ifhide%
        \newcommand{\clickeranswer}[1]{#1}%
    \else%
        \newcommand{\clickeranswer}[1]{\textbf{\textcolor{blue}{#1}}}%
    \fi
\fi

\input{../utils/slide-preamble2.tex}
\newcommand{\highlight}[1]{\textcolor{violet}{\textit{\textbf{#1}}}}
\newcommand{\super}[1]{\ensuremath{^{\textrm{#1}}}}
\newcommand{\sub}[1]{\ensuremath{_{\textrm{#1}}}}
\newcommand{\dC}{\ensuremath{^\circ{\textrm{C}}}}
\newcommand{\tb}{\hspace{2em}}
\providecommand{\e}[1]{\ensuremath{\times 10^{#1}}}
\newcommand{\myHangIndent}{\hangindent=5mm}

\makeatletter
\newcommand*{\rom}[1]{\expandafter\@slowromancap\romannumeral #1@}
\makeatother

\newcommand{\blankslide}{{\setbeamercolor{background canvas}{bg=black}
\setbeamercolor{whitetext}{fg=white}
\begin{frame}<handout:0>[plain]
\end{frame}}}

\newcommand{\whiteslide}{
\begin{frame}<handout:0>[plain]
\end{frame}}

\newcommand{\f}[1]{\ensuremath{F_{#1}}}

% \bibliography{../bib/references}
\bibliography{references}
\input{../utils/title-info.tex}

\title[Inferring Phylogenies]{Inferring Phylogenies}
% \date{\today}
\date{April 29, 2014}

\begin{document}

% \maketitle
\begin{noheadline}
\begin{frame}
    \begin{columns}[c]
        \column{.5\textwidth}
            \maketitle
        \column{.5\textwidth}
            \begin{figure}
                \begin{center}
                \includegraphics[width=\textwidth]{../images/darwin-tol-copyright-boris-kulikov-2007.jpg}
                \caption{\tiny \copyright~2007 Boris Kulikov \href{http://boris-kulikov.blogspot.com/}{boris-kulikov.blogspot.com}}
                \end{center}
            \end{figure}
    \end{columns}
\end{frame}
\end{noheadline}

\nopost{
\begin{noheadline}
\begin{frame}[c]
    \vspace{-6mm}
    \begin{center} 
        \includegraphics[height=1.2\textheight]{../images/seating-chart.pdf}
    \end{center}
\end{frame}
\end{noheadline}
}

\begin{noheadline}
\begin{frame}
    \begin{columns}[c]
        \column{.5\textwidth}
            Today's issues:
            \tableofcontents[subsectionstyle=hide]
        \column{.5\textwidth}
            \begin{figure}
                \begin{center}
                \includegraphics[width=\textwidth]{../images/darwin-tol-copyright-boris-kulikov-2007.jpg}
                \caption{\tiny \copyright~2007 Boris Kulikov \href{http://boris-kulikov.blogspot.com/}{boris-kulikov.blogspot.com}}
                \end{center}
            \end{figure}
    \end{columns}
\end{frame}
\end{noheadline}

% \begin{frame}
%     \frametitle{Acknowledgements}
%     Some of the slides and images were stolen (with permission) from Dr.\ Mark Holder.
% \end{frame}

% %%%%%%%%%%%%%%%%%%%%%%%%%%%%%%%%%%%%%%%%%
% \section{What is phylogenetics?}
% %%%%%%%%%%%%%%%%%%%%%%%%%%%%%%%%%%%%%%%%%
% \begin{frame}
%     \frametitle{What is phylogenetics?}
%     \begin{description}
%         \item[Systematics] The science devoted to the study of the diversity of
%             organisms, and the relationships among them.
%         \item[Classification] The ordering of organisms into named groups on
%             the basis of their relationships.
%         \item[Phylogenetics] The science of inferring the genealogical
%             relationships between species.
%     \end{description}
% \end{frame}
% Classification before phylogeny -- been classifying stuff since hg days -> tends to form hiearchies
% Linneaus formalized this
% Evolution gave the framework behind why hiearchies were intuitive

%%%%%%%%%%%%%%%%%%%%%%%%%%%%%%%%%%%%%%%%%
\section{Why is phylogenetics important?}
%%%%%%%%%%%%%%%%%%%%%%%%%%%%%%%%%%%%%%%%%

% \begin{frame}
%     % \frametitle{Why is phylogenetics important?}
%     \begin{quote}
%     Seen in the light of evolution, biology is, perhaps, intellectually the
%     most satisfying and inspiring science. Without that light it becomes a pile
%     of sundry facts some of them interesting or curious but making no
%     meaningful picture as a whole.
%     \end{quote}

%     \myHangIndent
%     - Dobzhansky, T. (1973). Nothing in biology makes sense except in the light
%     of evolution. The American Biology Teacher 35:125--129.

%     \bigskip

%     \begin{quote}
%     \ldots nothing in evolution makes sense except in the light of phylogeny \ldots
%     \end{quote}

%     \myHangIndent
%     - Society of Systematic Biologists
% \end{frame}

\begin{frame}
    % \frametitle{Why is phylogenetics important?}
    \begin{adjustwidth}{-1.5em}{-1.5em}
    We cannot understand biodiversity without its blueprint: the tree of life.
    \begin{columns}

        \column{0.5\linewidth}

        \begin{itemize}
            \item Every field of biology studies organisms.
            \item These organisms are \textbf{not} independent.
            \item To analyze biological data correctly we need to account for
                shared history among organisms.
        \end{itemize}

        \column{0.5\linewidth}
        \begin{figure}
            \begin{center}
            \includegraphics[width=0.55\textwidth]{../images/treeoflife.jpg}
            \caption{\tiny \copyright~2007 Tree of Life Web Project \href{http://tolweb.org/tree/}{tolweb.org}}
            \end{center}
        \end{figure}
    
    \end{columns}
    \end{adjustwidth}
\end{frame}

% \begin{frame}
%     \frametitle{West Nile Virus}
%     \begin{figure}
%         \begin{center}
%         \includegraphics[width=0.5\textwidth]{../images/pybus-fig2.jpg}
%         \caption{\tiny \shortfullcite{Pybus2012}}
%         \end{center}
%     \end{figure}
% \end{frame}

\begin{frame}
    \frametitle{Ebola}
    \begin{figure}
        \begin{center}
        \includegraphics[width=\textwidth]{../images/gire-et-al-2014-ebola-fig3.jpg}
        \caption{\tiny \shortfullcite{Gire2014}}
        \end{center}
    \end{figure}
\end{frame}



% \begin{frame}
% \frametitle{Outline}
% \tableofcontents[currentsection]
% \end{frame}

%%%%%%%%%%%%%%%%%%%%%%%%%%%%%%%%%%%%%%%%%
\subsection{Tree basics}
%%%%%%%%%%%%%%%%%%%%%%%%%%%%%%%%%%%%%%%%%

\begin{frame}
    \frametitle{Tree terminology}
    \begin{center}
        \vspace{-0.5cm}
        \begin{tikzpicture}
        [xscale=0.75,yscale=0.55,auto=left,every node/.style={circle}]%,fill=blue!20}]
            \node [inode, color=black, label={[label distance=-5pt]90:\small\sffamily\bfseries A}](a) at (1, 7) {};
            \node [inode, color=black, label={[label distance=-5pt]90:\small\sffamily\bfseries B}](b) at (3, 7) {};
            \node [inode, color=black, label={[label distance=-5pt]90:\small\sffamily\bfseries C}](c) at (5, 7) {};
            \node [inode, color=black, label={[label distance=-5pt]90:\small\sffamily\bfseries D}](d) at (7, 7) {};
            \node [inode, color=black, label={[label distance=-5pt]90:\small\sffamily\bfseries E}](e) at (9, 7) {};
            \node [inode, color=red, label={[label distance=-5pt]90:\small\sffamily\bfseries F}](f) at (11, 7) {};
            \node [inode](bc) at (4, 5)  {};
            \node [inode](ef) at (10, 5)  {};
            \node [inode, color=red](abc) at (3, 3)  {};
            \node [inode] (abcd) at (4, 1)  {};
            \node [inode, color=red](r) at (6, -3)  {};
          
            \foreach \from/\to in {bc/b,bc/c,abc/bc,abc/a,abcd/abc,abcd/d,r/abcd,ef/e,ef/f}
                \draw [very thick] (\from) -- (\to);
            \draw[very thick, color=red] (r) -- (ef);
            
            \draw [thick, <-, color=red] (2.9, 2.9) -- (2.4, 2.4) node[align=right, left]{};%{\sffamily internal node};
            \draw [thick, <-, color=red] (8.1, 0.9) -- (8.6, 0.4) node[align=left, right]{};%{\sffamily branch};
            \draw [thick, <-, color=red] (11.1, 6.9) -- (11.6, 6.4) node[align=left, right]{};%{\sffamily terminal node \\(or leaf, tip)};
            \draw [thick, <-, color=red] (5.9, -3.1) -- (5.4, -3.6) node[align=right, left]{};%{\sffamily root node};
        
        \end{tikzpicture}
    \end{center}
\end{frame}

\tikzset{node lower left/.style={font=\scriptsize,anchor=north east,text height=0.192cm,text depth=0.055cm,inner sep=0.03cm},
leaf/.style={font=\small,anchor=west,text height=0.240cm,text depth=0.068cm},
node upper left/.style={font=\scriptsize,anchor=south east,text height=0.192cm,text depth=0.055cm,inner sep=0.03cm},
bracket label/.style={font=\small,anchor=west,text height=0.240cm,text depth=0.068cm,inner sep=0.1cm},
node upper right/.style={font=\scriptsize,anchor=south west,text height=0.192cm,text depth=0.055cm,inner sep=0.03cm},
node right/.style={font=\scriptsize,anchor=west,text height=0.192cm,text depth=0.055cm,inner sep=0.03cm},
branch/.style={font=\tiny,text height=0.144cm,text depth=0.041cm,inner sep=0.025cm},
root/.style={font=\small,anchor=east,text height=0.240cm,text depth=0.068cm},
node lower right/.style={font=\scriptsize,anchor=north west,text height=0.192cm,text depth=0.055cm,inner sep=0.03cm}}

\clickerslide{
\begin{noheadline}
\begin{frame}
    \begin{adjustwidth}{-1.5em}{-1.5em}
    \begin{clickerquestion}
        \item Three of these trees describe the same relationships. Which one
            is different?
    \end{clickerquestion}
    % \vspace{-0.5cm}
    \begin{columns}

        \column{0.5\linewidth}

        %% This is a tikz file
\begin{tikzpicture}[very thick,inner sep=0.1cm]
%        +---3:\hspace{11mm}\includegraphics[height=6.5mm,resolution=150]{../images/Alces.png}
%     +--2
%     |  | +--5:\hspace{11mm}\includegraphics[width=12mm,resolution=150]{../images/Rattus.png}
%     |  +-4
%   +-1    +--6:\hspace{11mm}\includegraphics[height=6.5mm,resolution=150]{../images/Homo-sapiens.png}
%   | |
% 1:0 +------7:\hspace{11mm}\includegraphics[height=6.5mm,resolution=150]{../images/Loxodonta.png}
%   |
%   |      +-9:\hspace{11mm}\includegraphics[height=6.5mm,resolution=150]{../images/Phascolarctos.png}
%   +------8
%          +-10:\hspace{11mm}\includegraphics[height=6.5mm,resolution=150]{../images/Macropus.png}

% The scale is 0.960000, and the yScale is 0.800000

%% Coordinates of nodes.
\coordinate (nr) at (-0.4,1.450);
\coordinate (n0) at (0.000,1.450);
\coordinate (n1) at (0.960,2.500);
\coordinate (n1p) at (0.000,2.500);
\coordinate (n2) at (1.920,3.400);
\coordinate (n2p) at (0.960,3.400);
\coordinate (n3) at (3.840,4.000);
\coordinate (n3p) at (1.920,4.000);
\coordinate (n4) at (2.880,2.800);
\coordinate (n4p) at (1.920,2.800);
\coordinate (n5) at (3.840,3.200);
\coordinate (n5p) at (2.880,3.200);
\coordinate (n6) at (3.840,2.400);
\coordinate (n6p) at (2.880,2.400);
\coordinate (n7) at (3.840,1.600);
\coordinate (n7p) at (0.960,1.600);
\coordinate (n8) at (2.880,0.400);
\coordinate (n8p) at (0.000,0.400);
\coordinate (n9) at (3.840,0.800);
\coordinate (n9p) at (2.880,0.800);
\coordinate (n10) at (3.840,0.000);
\coordinate (n10p) at (2.880,0.000);

%% horizontal lines
\draw (n1p) -- (n1);
\draw (n2p) -- (n2);
\draw (n3p) -- (n3);
\draw (n4p) -- (n4);
\draw (n5p) -- (n5);
\draw (n6p) -- (n6);
\draw (n7p) -- (n7);
\draw (n8p) -- (n8);
\draw (n9p) -- (n9);
\draw (n10p) -- (n10);
\draw (n0) -- (nr);

%% vertical lines
\draw [line cap=rect] (n1p) -- (n8p);
\draw [line cap=rect] (n2p) -- (n7p);
\draw [line cap=rect] (n3p) -- (n4p);
\draw [line cap=rect] (n5p) -- (n6p);
\draw [line cap=rect] (n9p) -- (n10p);

%% leaf labels
\node at (n3) {\hspace{11mm}\includegraphics[height=6.5mm,resolution=150]{../images/Alces.png}};
\node at (n5) {\hspace{11mm}\includegraphics[width=12mm,resolution=150]{../images/Rattus.png}};
\node at (n6) {\hspace{11mm}\includegraphics[height=6.5mm,resolution=150]{../images/Homo-sapiens.png}};
\node at (n7) {\hspace{11mm}\includegraphics[height=6.5mm,resolution=150]{../images/Loxodonta.png}};
\node at (n9) {\hspace{11mm}\includegraphics[height=6.5mm,resolution=150]{../images/Phascolarctos.png}};
\node at (n10) {\hspace{11mm}\includegraphics[height=6.5mm,resolution=150]{../images/Macropus.png}};

%% root label
\node [root] at (nr) {\textcolor{red}{\sffamily\LARGE\bf 1)}};

% internal node labels (doSmartLabels is True)

\end{tikzpicture}


        % \vspace{-1cm}

        %% This is a tikz file
\begin{tikzpicture}[very thick,inner sep=0.1cm]
%     +-----2:\hspace{11mm}\includegraphics[height=6.5mm,resolution=150]{../images/Loxodonta.png}
%     |
%   +-1    +-5:\hspace{11mm}\includegraphics[width=12mm,resolution=150]{../images/Rattus.png}
%   | | +--4
%   | +-3  +-6:\hspace{11mm}\includegraphics[height=6.5mm,resolution=150]{../images/Homo-sapiens.png}
%   |   |
% 3:0   +---7:\hspace{11mm}\includegraphics[height=6.5mm,resolution=150]{../images/Alces.png}
%   |
%   |     +-9:\hspace{11mm}\includegraphics[height=6.5mm,resolution=150]{../images/Macropus.png}
%   +-----8
%         +-10:\hspace{11mm}\includegraphics[height=6.5mm,resolution=150]{../images/Phascolarctos.png}

% The scale is 0.960000, and the yScale is 0.800000

%% Coordinates of nodes.
\coordinate (nr) at (-0.4,1.750);
\coordinate (n0) at (0.000,1.750);
\coordinate (n1) at (0.960,3.100);
\coordinate (n1p) at (0.000,3.100);
\coordinate (n2) at (3.840,4.000);
\coordinate (n2p) at (0.960,4.000);
\coordinate (n3) at (1.920,2.200);
\coordinate (n3p) at (0.960,2.200);
\coordinate (n4) at (2.880,2.800);
\coordinate (n4p) at (1.920,2.800);
\coordinate (n5) at (3.840,3.200);
\coordinate (n5p) at (2.880,3.200);
\coordinate (n6) at (3.840,2.400);
\coordinate (n6p) at (2.880,2.400);
\coordinate (n7) at (3.840,1.600);
\coordinate (n7p) at (1.920,1.600);
\coordinate (n8) at (2.880,0.400);
\coordinate (n8p) at (0.000,0.400);
\coordinate (n9) at (3.840,0.800);
\coordinate (n9p) at (2.880,0.800);
\coordinate (n10) at (3.840,0.000);
\coordinate (n10p) at (2.880,0.000);

%% horizontal lines
\draw (n1p) -- (n1);
\draw (n2p) -- (n2);
\draw (n3p) -- (n3);
\draw (n4p) -- (n4);
\draw (n5p) -- (n5);
\draw (n6p) -- (n6);
\draw (n7p) -- (n7);
\draw (n8p) -- (n8);
\draw (n9p) -- (n9);
\draw (n10p) -- (n10);
\draw (n0) -- (nr);

%% vertical lines
\draw [line cap=rect] (n1p) -- (n8p);
\draw [line cap=rect] (n2p) -- (n3p);
\draw [line cap=rect] (n4p) -- (n7p);
\draw [line cap=rect] (n5p) -- (n6p);
\draw [line cap=rect] (n9p) -- (n10p);

%% leaf labels
\node at (n2) {\hspace{11mm}\includegraphics[height=6.5mm,resolution=150]{../images/Loxodonta.png}};
\node at (n5) {\hspace{11mm}\includegraphics[width=12mm,resolution=150]{../images/Rattus.png}};
\node at (n6) {\hspace{11mm}\includegraphics[height=6.5mm,resolution=150]{../images/Homo-sapiens.png}};
\node at (n7) {\hspace{11mm}\includegraphics[height=6.5mm,resolution=150]{../images/Alces.png}};
\node at (n9) {\hspace{11mm}\includegraphics[height=6.5mm,resolution=150]{../images/Macropus.png}};
\node at (n10) {\hspace{11mm}\includegraphics[height=6.5mm,resolution=150]{../images/Phascolarctos.png}};

%% root label
\node [root] at (nr) {\textcolor{red}{\sffamily\LARGE\bf 3)}};

% internal node labels (doSmartLabels is True)

\end{tikzpicture}


        \column{0.5\linewidth}

        \input{mammal-tree-2.tikz.tex}

        % \vspace{-1cm}

        \input{mammal-tree-4.tikz.tex}
        
    \end{columns}
    \end{adjustwidth}
\end{frame}
\end{noheadline}
}




\begin{frame}
    \begin{clickerquestion}
        \item Three of these trees describe the same relationships. Which one
            is dfferent?
    \end{clickerquestion}
    \vspace{-0.5cm}
    \begin{columns}
        \column{0.5\textwidth}
        \begin{center}
            \begin{tikzpicture}
            [xscale=0.35,yscale=0.2,auto=left,every node/.style={circle}]%,fill=blue!20}]
              \node [inode, color=black, label={[label distance=-5pt]90:\small\sffamily\bfseries A}](a) at (1, 7) {};
              \node [inode, color=black, label={[label distance=-5pt]90:\small\sffamily\bfseries B}](b) at (3, 7) {};
              \node [inode, color=black, label={[label distance=-5pt]90:\small\sffamily\bfseries C}](c) at (5, 7) {};
              \node [inode, color=black, label={[label distance=-5pt]90:\small\sffamily\bfseries D}](d) at (7, 7) {};
              \node [inode, color=black, label={[label distance=-5pt]90:\small\sffamily\bfseries E}](e) at (9, 7) {};
              \node [inode, color=black, label={[label distance=-5pt]90:\small\sffamily\bfseries F}](f) at (11, 7) {};
              \node [inode](bc) at (4, 5)  {};
              \node [inode](ef) at (10, 5)  {};
              \node [inode](abc) at (3, 3)  {};
              \node [inode] (abcd) at (4, 1)  {};
              \node [inode, label=below: {\textcolor{red}{1}}](r) at (6, -3)  {};
            
              \foreach \from/\to in {bc/b,bc/c,abc/bc,abc/a,abcd/abc,abcd/d,r/abcd,r/ef,ef/e,ef/f}
                \draw [ultra thick] (\from) -- (\to);
              
            \end{tikzpicture}
        \end{center}
        \vspace{-1cm}
        \begin{center}
            \begin{tikzpicture}
            [xscale=0.35,yscale=0.2,auto=left,every node/.style={circle}]%,fill=blue!20}]
                \node [inode, color=black, label={[label distance=-5pt]90:\small\sffamily\bfseries A}](a) at (7, 7) {};
                \node [inode, color=black, label={[label distance=-5pt]90:\small\sffamily\bfseries B}](b) at (3, 7) {};
                \node [inode, color=black, label={[label distance=-5pt]90:\small\sffamily\bfseries C}](c) at (5, 7) {};
                \node [inode, color=black, label={[label distance=-5pt]90:\small\sffamily\bfseries D}](d) at (1, 7) {};
                \node [inode, color=black, label={[label distance=-5pt]90:\small\sffamily\bfseries F}](e) at (9, 7) {};
                \node [inode, color=black, label={[label distance=-5pt]90:\small\sffamily\bfseries E}](f) at (11, 7) {};
                \node [inode](bc) at (4, 5)  {};
                \node [inode](ef) at (10, 5)  {};
                \node [inode](abc) at (5, 3)  {};
                \node [inode] (abcd) at (4, 1)  {};
                \node [inode, label=below: {\textcolor{red}{3}}](r) at (6, -3)  {};
              
                \foreach \from/\to in {bc/b,bc/c,abc/bc,abc/a,abcd/abc,abcd/d,r/abcd,r/ef,ef/e,ef/f}
                    \draw [ultra thick] (\from) -- (\to);
              
            \end{tikzpicture}
        \end{center}
        \column{0.5\textwidth}
        \begin{center}
            \begin{tikzpicture}
            [xscale=0.35,yscale=0.2,auto=left,every node/.style={circle}]%,fill=blue!20}]
              \node [inode, color=black, label={[label distance=-5pt]90:\small\sffamily\bfseries A}](a) at (7, 7) {};
              \node [inode, color=black, label={[label distance=-5pt]90:\small\sffamily\bfseries B}](b) at (11, 7) {};
              \node [inode, color=black, label={[label distance=-5pt]90:\small\sffamily\bfseries C}](c) at (9, 7) {};
              \node [inode, color=black, label={[label distance=-5pt]90:\small\sffamily\bfseries D}](d) at (5, 7) {};
              \node [inode, color=black, label={[label distance=-5pt]90:\small\sffamily\bfseries E}](e) at (1, 7) {};
              \node [inode, color=black, label={[label distance=-5pt]90:\small\sffamily\bfseries F}](f) at (3, 7) {};
              \node [inode](bc) at (10, 5)  {};
              \node [inode](ef) at (2, 5)  {};
              \node [inode](abc) at (9, 3)  {};
              \node [inode] (abcd) at (8, 1)  {};
              \node [inode, label=below: {\textcolor{red}{2}}](r) at (6, -3)  {};
            
              \foreach \from/\to in {bc/b,bc/c,abc/bc,abc/a,abcd/abc,abcd/d,r/abcd,r/ef,ef/e,ef/f}
                \draw [ultra thick] (\from) -- (\to);
              
            \end{tikzpicture}
        \end{center}
        \vspace{-1cm}
        \begin{center}
            \begin{tikzpicture}
            [xscale=0.35,yscale=0.2,auto=left,every node/.style={circle}]%,fill=blue!20}]
              \node [inode, color=black, label={[label distance=-5pt]90:\small\sffamily\bfseries A}](a) at (1, 7) {};
              \node [inode, color=black, label={[label distance=-5pt]90:\small\sffamily\bfseries B}](b) at (3, 7) {};
              \node [inode, color=black, label={[label distance=-5pt]90:\small\sffamily\bfseries C}](c) at (5, 7) {};
              \node [inode, color=black, label={[label distance=-5pt]90:\small\sffamily\bfseries D}](d) at (7, 7) {};
              \node [inode, color=black, label={[label distance=-5pt]90:\small\sffamily\bfseries E}](e) at (9, 7) {};
              \node [inode, color=black, label={[label distance=-5pt]90:\small\sffamily\bfseries F}](f) at (11, 7) {};
              \node [inode](bc) at (4, 5)  {};
              \node [inode](ef) at (10, 5)  {};
              \node [inode](abc) at (5, 3)  {};
              \node [inode] (abcd) at (4, 1)  {};
              \node [inode, label=below: {\textcolor{red}{4}}](r) at (6, -3)  {};
            
              \foreach \from/\to in {bc/b,bc/c,abc/bc,abcd/a,abcd/abc,abcd/d,r/abcd,r/ef,ef/e,ef/f}
                \draw [ultra thick] (\from) -- (\to);
              
            \end{tikzpicture}
        \end{center}
    \end{columns}
    \hidebox{Correct answer = 4}
\end{frame}


% \begin{frame}
%     \frametitle{Style of presentation varies a lot}
%     \begin{center}
%         \includegraphics[width=0.8\textwidth]{../images/crocodylia-species-tree-cladogram.pdf}
%     \end{center}
% \end{frame}

\begin{frame}
    \frametitle{Style of presentation varies a lot}
    \begin{center}
        \includegraphics[width=0.3\textwidth]{../images/crocodylia-species-tree-cladogram.pdf}
        \hspace{2mm}
        \includegraphics[width=0.3\textwidth]{../images/crocodylia-ml.pdf}
        \hspace{2mm}
        \includegraphics[width=0.3\textwidth]{../images/crocodylia-species-tree-square.pdf}

        \vspace{0.5cm}
        \includegraphics[width=0.3\textwidth]{../images/crocodylia-species-tree-round.pdf}
        \hspace{2mm}
        \includegraphics[width=0.3\textwidth]{../images/crocodylia-species-tree-triangle.pdf}
        \hspace{2mm}
        \includegraphics[width=0.3\textwidth]{../images/crocodylia-species-tree-circle.pdf}
    \end{center}
\end{frame}

% \begin{frame}
%     \frametitle{Interpreting rooted vs unrooted trees}
%     \vspace{-0.5cm}
%     \begin{columns}[c]
%         \column{.4\textwidth}
%         \begin{center}
%             \begin{tikzpicture}
%             [xscale=0.35,yscale=0.4,auto=left,every node/.style={circle}]%,fill=blue!20}]
%                 \node [inode, color=black, label={[label distance=-5pt]90:\small\sffamily\bfseries A}](a) at (1, 7) {};
%                 \node [inode, color=black, label={[label distance=-5pt]90:\small\sffamily\bfseries B}](b) at (3, 7) {};
%                 \node [inode, color=black, label={[label distance=-5pt]90:\small\sffamily\bfseries C}](c) at (5, 7) {};
%                 \node [inode, color=black, label={[label distance=-5pt]90:\small\sffamily\bfseries D}](d) at (7, 7) {};
%                 \node [inode, color=black, label={[label distance=-5pt]90:\small\sffamily\bfseries E}](e) at (9, 7) {};
%                 \node [inode, color=black, label={[label distance=-5pt]90:\small\sffamily\bfseries F}](f) at (11, 7) {};
%                 \node [inode](bc) at (4, 5)  {};
%                 \node [inode](ef) at (10, 5)  {};
%                 \node [inode](abc) at (3, 3)  {};
%                 \node [inode] (abcd) at (4, 1)  {};
%                 \node [inode](r) at (6, -3)  {};
              
%                 \foreach \from/\to in {bc/b,bc/c,abc/bc,abc/a,abcd/abc,abcd/d,r/abcd,r/ef,ef/e,ef/f}
%                     \draw [very thick] (\from) -- (\to);

              
%             \end{tikzpicture}
%         \end{center}
%         \column{.6\textwidth}
%         \begin{center}
%             \uncover<2->{Where should the root go?}
%             \begin{tikzpicture}
%             [xscale=0.35, yscale=0.6, auto=left,every node/.style={circle}]%,fill=blue!20}]
%                 \node [tnode](d) at (1,10.5) {C};
%                 \node [tnode](a) at (3,11) {B};
%                 \node [tnode](c) at (3,6) {A};
%                 \node [tnode](e) at (13,10) {E};
%                 \node [tnode](f) at (16,7) {F};
%                 \node [tnode](b) at (14,2) {D};
%                 \node [inode](ef) at (13,9) {};
%                 \node [inode](efb) at (12, 8) {};
%                 \node [inode](adc) at (4,7) {};
%                 \node [inode](ad) at (2, 10) {};
              
%                 \foreach \from/\to in {efb/ef,efb/b,adc/c,adc/efb,ad/a,ad/d,ad/adc,ef/e,ef/f}
%                     \draw [very thick] (\from) -- (\to);
%                 \uncover<2->{
%                 \draw [very thick, <-] (2.7,8.5) -- (0.5, 8.5) node[align=right, left]{\textcolor{red}{1}};
%                 \draw [very thick, <-] (8,7.2) -- (8, 5.5) node[align=right, below]{\textcolor{red}{2}};
%                 \draw [very thick, <-] (12.2, 8.7) -- (10.0, 9.6) node[align=left, left]{\textcolor{red}{3}};
%                 \draw [very thick, <-] (12.7,5) -- (11.5, 3.5) node[align=left, below]{\textcolor{red}{4}};
%                 \draw [very thick, <-] (14.5,8.2) -- (14.8, 9.5) node[align=left, above]{\textcolor{red}{5}};
%             }

%             \end{tikzpicture}
%         \end{center}
%     \end{columns}
%     \hidebox{Correct answer = 3}
% \end{frame}

% \begin{frame}
%     \begin{clickerquestion}
%         \item
%             In reference to the unrooted phylogeny, which of the
%             following statements are always correct?
%         \begin{clickeroptions}
%             \item A and D are sister species.
%             \item Species A, D, and C are monophyletic.
%             \item Species D is more closely related to A than  B.
%             \item All of the above.
%             \item \clickeranswer{None of the above.}
%         \end{clickeroptions}
%     \end{clickerquestion}

%     \begin{center}
%     \begin{tikzpicture}
%     [xscale=0.55,yscale=0.45,auto=left,every node/.style={circle}]%,fill=blue!20}]
%       \node [tnode](d) at (1,10.5) {D};
%       \node [tnode](a) at (3,11) {A};
%       \node [tnode](c) at (3,6) {C};
%       \node [tnode](e) at (13,10) {E};
%       \node [tnode](f) at (16,7) {F};
%       \node [tnode](b) at (14,2) {B};
%       \node [inode](ef) at (13,9) {};
%       \node [inode](efb) at (12, 8) {};
%       \node [inode](adc) at (4,7) {};
%       \node [inode](ad) at (2, 10) {};
%       \node [empty](l) at (8,6) {};
    
%       \foreach \from/\to in {efb/ef,efb/b,adc/c,adc/efb,ad/a,ad/d,ad/adc,ef/e,ef/f}
%         \draw[very thick] (\from) -- (\to);
    
%     \end{tikzpicture}
%     \end{center}
% \end{frame}

                
\begin{frame}[b]
    \frametitle{Classification---the good, the bad, \& the ugly}
    \begin{description}
        \item<2->[Monophyletic group] A group that consists of an ancestor and all
            of its descendants. Also called a clade or ``natural'' group. The
            basis of phylogenetic classification. Good!
    \end{description}
    \vspace{-0.45cm}
    \begin{center}
        \begin{onlyenv}<1-2>
        \begin{tikzpicture}
        [xscale=0.75,yscale=0.45,auto=left,every node/.style={circle}]%,fill=blue!20}]
            \node [inode, color=green, label={[label distance=-5pt]90:\sffamily\bfseries \textcolor{green}{A}}](a) at (1, 7) {};
            \node [inode, color=green, label={[label distance=-5pt]90:\sffamily\bfseries \textcolor{green}{B}}](b) at (3, 7) {};
            \node [inode, color=green, label={[label distance=-5pt]90:\sffamily\bfseries \textcolor{green}{C}}](c) at (5, 7) {};
            \node [inode, color=black, label={[label distance=-5pt]90:\sffamily\bfseries D}](d) at (7, 7) {};
            \node [inode, color=black, label={[label distance=-5pt]90:\sffamily\bfseries E}](e) at (9, 7) {};
            \node [inode, color=black, label={[label distance=-5pt]90:\sffamily\bfseries F}](f) at (11, 7) {};
            \node [inode, color=green](bc) at (4, 5)  {};
            \node [inode](ef) at (10, 5)  {};
            \node [inode, color=green](abc) at (3, 3)  {};
            \node [inode] (abcd) at (4, 1)  {};
            \node [inode](r) at (6, -3)  {};
          
            \foreach \from/\to in {bc/b,bc/c,abc/bc,abc/a}
                \draw [ultra thick, color=green] (\from) -- (\to);
            \foreach \from/\to in {abcd/abc,abcd/d,r/abcd,r/ef,ef/e,ef/f}
                \draw [ultra thick] (\from) -- (\to);
            
        \end{tikzpicture}
        \end{onlyenv}
    \end{center}

    \begin{center}
        \begin{onlyenv}<3>
        \begin{tikzpicture}
        [xscale=0.75,yscale=0.45,auto=left,every node/.style={circle}]%,fill=blue!20}]
            \node [inode, color=green, label={[label distance=-5pt]90:\sffamily\bfseries \textcolor{green}{A}}](a) at (1, 7) {};
            \node [inode, color=green, label={[label distance=-5pt]90:\sffamily\bfseries \textcolor{green}{B}}](b) at (3, 7) {};
            \node [inode, color=green, label={[label distance=-5pt]90:\sffamily\bfseries \textcolor{green}{C}}](c) at (5, 7) {};
            \node [inode, color=green, label={[label distance=-5pt]90:\sffamily\bfseries \textcolor{green}{D}}](d) at (7, 7) {};
            \node [inode, color=black, label={[label distance=-5pt]90:\sffamily\bfseries E}](e) at (9, 7) {};
            \node [inode, color=black, label={[label distance=-5pt]90:\sffamily\bfseries F}](f) at (11, 7) {};
            \node [inode, color=green](bc) at (4, 5)  {};
            \node [inode](ef) at (10, 5)  {};
            \node [inode, color=green](abc) at (3, 3)  {};
            \node [inode, color=green] (abcd) at (4, 1)  {};
            \node [inode](r) at (6, -3)  {};
          
            \foreach \from/\to in {bc/b,bc/c,abc/bc,abc/a,abcd/abc,abcd/d}
                \draw [ultra thick, color=green] (\from) -- (\to);
            \foreach \from/\to in {r/abcd,r/ef,ef/e,ef/f}
                \draw [ultra thick] (\from) -- (\to);
            
        \end{tikzpicture}
        \end{onlyenv}
    \end{center}

    \begin{center}
        \begin{onlyenv}<4>
        \begin{tikzpicture}
        [xscale=0.75,yscale=0.45,auto=left,every node/.style={circle}]%,fill=blue!20}]
            \node [inode, color=black, label={[label distance=-5pt]90:\sffamily\bfseries A}](a) at (1, 7) {};
            \node [inode, color=black, label={[label distance=-5pt]90:\sffamily\bfseries B}](b) at (3, 7) {};
            \node [inode, color=black, label={[label distance=-5pt]90:\sffamily\bfseries C}](c) at (5, 7) {};
            \node [inode, color=black, label={[label distance=-5pt]90:\sffamily\bfseries D}](d) at (7, 7) {};
            \node [inode, color=green, label={[label distance=-5pt]90:\sffamily\bfseries \textcolor{green}{E}}](e) at (9, 7) {};
            \node [inode, color=green, label={[label distance=-5pt]90:\sffamily\bfseries \textcolor{green}{F}}](f) at (11, 7) {};
            \node [inode](bc) at (4, 5)  {};
            \node [inode, color=green](ef) at (10, 5)  {};
            \node [inode](abc) at (3, 3)  {};
            \node [inode] (abcd) at (4, 1)  {};
            \node [inode](r) at (6, -3)  {};
          
            \foreach \from/\to in {bc/b,bc/c,abc/bc,abc/a,abcd/abc,abcd/d,r/abcd,r/ef}
                \draw [ultra thick] (\from) -- (\to);
            \foreach \from/\to in {ef/e,ef/f}
                \draw [ultra thick, color=green] (\from) -- (\to);
            
        \end{tikzpicture}
        \end{onlyenv}
    \end{center}
\end{frame}

\begin{frame}[b]
    \frametitle{Classification---the good, the bad, \& the ugly}
    \begin{description}
        \item<2->[Paraphyletic group] A group that consists of an ancestor and
            some, but not all, of its descendants. Need to add one clade or tip
            to get monophyly. An ``unnatural'' group. Bad!
    \end{description}
    \vspace{-0.25cm}
    \begin{center}
        \begin{onlyenv}<1-2>
        \begin{tikzpicture}
        [xscale=0.75,yscale=0.45,auto=left,every node/.style={circle}]%,fill=blue!20}]
            \node [inode, label={[label distance=-5pt]90:\sffamily\bfseries A}](a) at (1, 7) {};
            \node [inode, color=red, label={[label distance=-5pt]90:\sffamily\bfseries \textcolor{red}{B}}](b) at (3, 7) {};
            \node [inode, color=red, label={[label distance=-5pt]90:\sffamily\bfseries \textcolor{red}{C}}](c) at (5, 7) {};
            \node [inode, color=red, label={[label distance=-5pt]90:\sffamily\bfseries \textcolor{red}{D}}](d) at (7, 7) {};
            \node [inode, color=black, label={[label distance=-5pt]90:\sffamily\bfseries E}](e) at (9, 7) {};
            \node [inode, color=black, label={[label distance=-5pt]90:\sffamily\bfseries F}](f) at (11, 7) {};
            \node [inode, color=red](bc) at (4, 5)  {};
            \node [inode](ef) at (10, 5)  {};
            \node [inode, color=red](abc) at (3, 3)  {};
            \node [inode, color=red] (abcd) at (4, 1)  {};
            \node [inode](r) at (6, -3)  {};
          
            \foreach \from/\to in {bc/b,bc/c,abc/bc,abcd/abc,abcd/d}
                \draw [ultra thick, color=red] (\from) -- (\to);
            \foreach \from/\to in {r/abcd,r/ef,ef/e,ef/f,abc/a}
                \draw [ultra thick] (\from) -- (\to);
            
        \end{tikzpicture}
        \end{onlyenv}
    \end{center}

    \begin{center}
        \begin{onlyenv}<3>
        \begin{tikzpicture}
        [xscale=0.75,yscale=0.45,auto=left,every node/.style={circle}]%,fill=blue!20}]
            \node [inode, color=red, label={[label distance=-5pt]90:\sffamily\bfseries \textcolor{red}{A}}](a) at (1, 7) {};
            \node [inode, label={[label distance=-5pt]90:\sffamily\bfseries B}](b) at (3, 7) {};
            \node [inode, label={[label distance=-5pt]90:\sffamily\bfseries C}](c) at (5, 7) {};
            \node [inode, color=red, label={[label distance=-5pt]90:\sffamily\bfseries \textcolor{red}{D}}](d) at (7, 7) {};
            \node [inode, color=black, label={[label distance=-5pt]90:\sffamily\bfseries E}](e) at (9, 7) {};
            \node [inode, color=black, label={[label distance=-5pt]90:\sffamily\bfseries F}](f) at (11, 7) {};
            \node [inode](bc) at (4, 5)  {};
            \node [inode](ef) at (10, 5)  {};
            \node [inode, color=red](abc) at (3, 3)  {};
            \node [inode, color=red] (abcd) at (4, 1)  {};
            \node [inode](r) at (6, -3)  {};
          
            \foreach \from/\to in {abc/a,abcd/abc,abcd/d}
                \draw [ultra thick, color=red] (\from) -- (\to);
            \foreach \from/\to in {abc/bc,bc/b,bc/c,r/abcd,r/ef,ef/e,ef/f}
                \draw [ultra thick] (\from) -- (\to);
            
        \end{tikzpicture}
        \end{onlyenv}
    \end{center}
\end{frame}

\begin{frame}[b]
    \frametitle{Classification---the good, the bad, \& the ugly}
    \begin{description}
        \item<2->[Polyphyletic group] A group that consists of unrelated tips. Need
            to add more than one clade or tip to get monophyly. An
            ``unnatural'' group. Ugly!
    \end{description}
    \vspace{-0.25cm}
    \begin{center}
        \begin{onlyenv}<1-2>
        \begin{tikzpicture}
        [xscale=0.75,yscale=0.45,auto=left,every node/.style={circle}]%,fill=blue!20}]
            \node [inode, label={[label distance=-5pt]90:\sffamily\bfseries A}](a) at (1, 7) {};
            \node [inode, color=red, label={[label distance=-5pt]90:\sffamily\bfseries \textcolor{red}{B}}](b) at (3, 7) {};
            \node [inode, color=red, label={[label distance=-5pt]90:\sffamily\bfseries \textcolor{red}{C}}](c) at (5, 7) {};
            \node [inode, label={[label distance=-5pt]90:\sffamily\bfseries D}](d) at (7, 7) {};
            \node [inode, color=red, label={[label distance=-5pt]90:\sffamily\bfseries \textcolor{red}{E}}](e) at (9, 7) {};
            \node [inode, color=red, label={[label distance=-5pt]90:\sffamily\bfseries \textcolor{red}{F}}](f) at (11, 7) {};
            \node [inode, color=red](bc) at (4, 5)  {};
            \node [inode, color=red](ef) at (10, 5)  {};
            \node [inode](abc) at (3, 3)  {};
            \node [inode] (abcd) at (4, 1)  {};
            \node [inode](r) at (6, -3)  {};
          
            \foreach \from/\to in {bc/b,bc/c,ef/e,ef/f}
                \draw [ultra thick, color=red] (\from) -- (\to);
            \foreach \from/\to in {r/abcd,r/ef,abc/a,abc/bc,abcd/abc,abcd/d}
                \draw [ultra thick] (\from) -- (\to);
            
        \end{tikzpicture}
        \end{onlyenv}
    \end{center}
\end{frame}

% \begin{frame}[b]
%     \frametitle{Other terms}
%     \begin{description}
%         \item[Ingroup] The clade (monophyletic group) of taxa that is the
%             focus of a study (e.g., A, B, and C below).
%         \item[Outgroup] All other taxa outside of the ingroup clade (e.g., D,
%             E, and F below).
%     \end{description}
%     \vspace{-0.5cm}
%     \begin{center}
%         \begin{onlyenv}<1>
%         \begin{tikzpicture}
%         [xscale=0.75,yscale=0.45,auto=left,every node/.style={circle}]%,fill=blue!20}]
%             \node [inode, color=red, label={[label distance=-5pt]90:\sffamily\bfseries \textcolor{red}{A}}](a) at (1, 7) {};
%             \node [inode, color=red, label={[label distance=-5pt]90:\sffamily\bfseries \textcolor{red}{B}}](b) at (3, 7) {};
%             \node [inode, color=red, label={[label distance=-5pt]90:\sffamily\bfseries \textcolor{red}{C}}](c) at (5, 7) {};
%             \node [inode, label={[label distance=-5pt]90:\sffamily\bfseries D}](d) at (7, 7) {};
%             \node [inode, color=black, label={[label distance=-5pt]90:\sffamily\bfseries \textcolor{black}{E}}](e) at (9, 7) {};
%             \node [inode, color=black, label={[label distance=-5pt]90:\sffamily\bfseries \textcolor{black}{F}}](f) at (11, 7) {};
%             \node [inode, color=red](bc) at (4, 5)  {};
%             \node [inode, color=black](ef) at (10, 5)  {};
%             \node [inode, color=red](abc) at (3, 3)  {};
%             \node [inode] (abcd) at (4, 1)  {};
%             \node [inode](r) at (6, -3)  {};
          
%             \foreach \from/\to in {abc/a,abc/bc,bc/b,bc/c}
%                 \draw [ultra thick, color=red] (\from) -- (\to);
%             \foreach \from/\to in {r/abcd,r/ef,abcd/abc,abcd/d,ef/e,ef/f}
%                 \draw [ultra thick] (\from) -- (\to);
            
%         \end{tikzpicture}
%         \end{onlyenv}
%     \end{center}
% \end{frame}

\begin{frame}[b]
    \frametitle{Other terms}
    \begin{description}
        \small
        \item[Sister group] The next most closely related tip or clade; always
            reciprocal.
    \end{description}
    \vspace{-0.5cm}
    \begin{center}
        \begin{onlyenv}<1>
        \begin{tikzpicture}
        [xscale=0.75,yscale=0.45,auto=left,every node/.style={circle}]%,fill=blue!20}]
            \node [inode, color=black, label={[label distance=-5pt]90:\sffamily\bfseries \textcolor{black}{A}}](a) at (1, 7) {};
            \node [inode, color=black, label={[label distance=-5pt]90:\sffamily\bfseries \textcolor{black}{B}}](b) at (3, 7) {};
            \node [inode, color=black, label={[label distance=-5pt]90:\sffamily\bfseries \textcolor{black}{C}}](c) at (5, 7) {};
            \node [inode, label={[label distance=-5pt]90:\sffamily\bfseries D}](d) at (7, 7) {};
            \node [inode, color=black, label={[label distance=-5pt]90:\sffamily\bfseries \textcolor{black}{E}}](e) at (9, 7) {};
            \node [inode, color=black, label={[label distance=-5pt]90:\sffamily\bfseries \textcolor{black}{F}}](f) at (11, 7) {};
            \node [inode, color=black](bc) at (4, 5)  {};
            \node [inode, color=black](ef) at (10, 5)  {};
            \node [inode, color=black](abc) at (3, 3)  {};
            \node [inode] (abcd) at (4, 1)  {};
            \node [inode](r) at (6, -3)  {};
          
            \foreach \from/\to in {abc/a,abc/bc,bc/b,bc/c}
                \draw [ultra thick, color=black] (\from) -- (\to);
            \foreach \from/\to in {r/abcd,r/ef,abcd/abc,abcd/d,ef/e,ef/f}
                \draw [ultra thick] (\from) -- (\to);
            
        \end{tikzpicture}
        \end{onlyenv}
    \end{center}
\end{frame}

\begin{frame}
    \begin{clickerquestion}
        \item In reference to the rooted phylogeny below, which of the
            following statements are correct?
        \begin{clickeroptions}
            \item A and B are sister species.
            \item Species B, C, and D are monophyletic.
            \item Species A, B, and C are paraphyletic.
            \item \clickeranswer{Species D is more closely related to A than E.}
        \end{clickeroptions}
    \end{clickerquestion}

    \begin{center}
    \begin{tikzpicture}
    [xscale=0.55,yscale=0.4,auto=left,every node/.style={circle}]%,fill=blue!20}]
      \node [tnode](a) at (1, 7) {A};
      \node [tnode](b) at (3, 7) {B};
      \node [tnode](c) at (5, 7) {C};
      \node [tnode](d) at (7, 7) {D};
      \node [tnode](e) at (9, 7) {E};
      \node [tnode](f) at (11, 7) {F};
      \node [inode](bc) at (4, 5)  {};
      \node [inode](ef) at (10, 5)  {};
      \node [inode](abc) at (3, 3)  {};
      \node [inode] (abcd) at (4, 1)  {};
      \node [inode](r) at (6, -3)  {};
    
      \foreach \from/\to in {bc/b,bc/c,abc/bc,abc/a,abcd/abc,abcd/d,r/abcd,r/ef,ef/e,ef/f}
        \draw[ultra thick] (\from) -- (\to);
    
    \end{tikzpicture}
    \end{center}
\end{frame}

%%%%%%%%%%%%%%%%%%%%%%%%%%%%%%%%%%%%%%%%%%%%%%%%%%%
\section{Building trees tutorial}
%%%%%%%%%%%%%%%%%%%%%%%%%%%%%%%%%%%%%%%%%%%%%%%%%%%

\begin{frame}
    Phylogenetic trees worksheet

    \vspace{1cm}
    Work in teams of 3. Middle person as the scribe (does the writing). As you
    work through the questions, be sure to explain your logic to each other.

    \vspace{1cm}
    Please ask us for help, if you need it!
\end{frame}

\begin{frame}
    \begin{enumerate}%[Q 1:]
        \vspace{-1cm}
        \begin{columns}
            \small
            \column{0.7\textwidth}

            \column{0.3\textwidth}
            Sharks and rays

            \vspace{0.3cm}
            Lizards

            \vspace{0.3cm}
            Snakes

            \vspace{0.3cm}
            Mammals

            \vspace{0.3cm}
            Amphibians

            \vspace{0.3cm}
            Ray-finned fish
        \end{columns}

            \vspace{0.5cm}
        \item[Q 5.] Are lizards or sharks more closely related to amphibians?

            \vspace{0.5cm}
        \item[Q 6.] Are lizards or ray-finned fish more closely related to sharks?
        
    \end{enumerate}
\end{frame}

%%%%%%%%%%%%%%%%%%%%%%%%%%%%%%%%%%%%%%%%%%%%%%%%%%%
\section{The problem of homoplasy}
%%%%%%%%%%%%%%%%%%%%%%%%%%%%%%%%%%%%%%%%%%%%%%%%%%%

\begin{frame}
    \frametitle{Character terminology}
    \begin{description}
        \item[Homology] A character state that is shared among taxa due to
            inheritance from a common ancestor (shared by descent).
        \item[Homoplasy] A character state that is shared because of multiple
            (convergent) changes. Homo = ``same'' plasy = ``change.'' Diagnose
            polyphyletic groups.
    \end{description}
\end{frame}

\begin{frame}[t]
    \begin{center}
    \begin{tikzpicture}
    [xscale=0.65,yscale=0.25,auto=left,every node/.style={circle}]%,fill=blue!20}]
        \node [inode, color=red, label={[label distance=-5pt]90:\sffamily\bfseries \textcolor{red}{A}}](a) at (1, 7) {};
        \node [inode, color=blue, label={[label distance=-5pt]90:\sffamily\bfseries \textcolor{blue}{B}}](b) at (3, 7) {};
        \node [inode, color=blue, label={[label distance=-5pt]90:\sffamily\bfseries \textcolor{blue}{C}}](c) at (5, 7) {};
        \node [inode, color=blue, label={[label distance=-5pt]90:\sffamily\bfseries \textcolor{blue}{D}}](d) at (7, 7) {};
        \node [inode, color=red, label={[label distance=-5pt]90:\sffamily\bfseries \textcolor{red}{E}}](e) at (9, 7) {};
        \node [inode, color=red, label={[label distance=-5pt]90:\sffamily\bfseries \textcolor{red}{F}}](f) at (11, 7) {};
        \node [inode, color=blue](bc) at (4, 5)  {};
        \node [inode, color=red](ef) at (10, 5)  {};
        \node [inode, color=blue](abc) at (3, 3)  {};
        \node [inode, color=blue] (abcd) at (4, 1)  {};
        \node [inode, color=blue](r) at (6, -3)  {};
      
        \foreach \from/\to in {bc/b,bc/c,abc/bc,abcd/abc,abcd/d}
            \draw [ultra thick, color=blue] (\from) -- (\to);
        \draw [ultra thick, color=blue] (abc) -- (2, 5);
        \draw [ultra thick, color=red] (2, 5) -- (a);
        \draw [ultra thick, color=blue] (r) -- (5, -1);
        \draw [ultra thick, color=blue] (5, -1) -- (abcd);
        \draw [ultra thick, color=blue] (r) -- (8, 1);
        \draw [ultra thick, color=red] (8, 1) -- (ef);
        \foreach \from/\to in {ef/e,ef/f}
            \draw [ultra thick, color=red] (\from) -- (\to);
        
    \end{tikzpicture}
    \end{center}

    \begin{clickerquestion}
        \item \textcolor{blue}{Blue} = limbs; \textcolor{red}{Red} = limbless.
            Which is correct?
        \begin{clickeroptions}
            \item Limbs are homoplastic for B, C, \& D;
                lack of limbs is homoplastic for A, E, \& F;
                lack of limbs is homologous for E \& F.
            \item Species B, C, \& D are homologous;
                species A, E, \& F are homoplastic;
                species A \& F are homologous.
            \item \clickeranswer{Limbs are homologous for B, C, \& D;
                lack of limbs is homoplastic for A, E, \& F;
                lack of limbs is homologous for E \& F.}
            \item Limblessness is the ancestral character state.
        \end{clickeroptions}
    \end{clickerquestion}

\end{frame}

\begin{frame}
    \frametitle{Homology or homoplasy?}
    \begin{itemize}
        \item<1-> Body shape in dolphins and ichthyosaurs
        \item<2-> Vertebrae in dolphins and ichthyosaurs
    \end{itemize}
    \begin{figure}
        \includegraphics[height=2.5cm]{../images/ichthyosaur-nobu-tamura-cc-by-25.jpg}
        \caption{\tiny \href{http://creativecommons.org/licenses/by/2.5/}{CC BY 2.5} \href{http://spinops.blogspot.com/}{Nobu Tamura}}
    \end{figure}
    \vspace{-0.7cm}
    \begin{figure}
        \includegraphics[height=2.5cm]{../images/dolphin-noaa-cc-by-sa-30.jpg}
        \caption{\tiny \href{http://creativecommons.org/licenses/by-sa/3.0/}{CC BY-SA 3.0} NOAA}
    \end{figure}
\end{frame}

    
\begin{frame}
    \frametitle{Homology or homoplasy?}
    \begin{itemize}
        \item Hair in chimps and humans

            \hidebox{Homology}
        \item Hair loss in whales and humans

            \hidebox{Homoplasy}
        \item Flippers in penguins, seals, and turtles

            \hidebox{Homoplasy}
        \item Bones of the forelimb in penguins, seals, and turtles

            \hidebox{Homology}
        \item Camera eye in octopus and vertebrates

            \hidebox{Homoplasy}
        \item Multicellularity in octopus and vertebrates

            \hidebox{Homology}
    \end{itemize}
\end{frame}

\begin{frame}
    \begin{clickerquestion}
        \item Consider the wings of bats and birds; which of the following is
            correct?
        \begin{clickeroptions}
            \item Wings are homologous; limbs are homoplastic.
            \item \clickeranswer{Limbs are homologous; wings are homoplastic.}
            \item Both have highly modified hand and finger bones; these
                modifications are homologous.
        \end{clickeroptions}
    \end{clickerquestion}
\end{frame}

\begin{frame}
    \frametitle{What causes homoplasy?}
    \begin{enumerate}
        \item In morphological traits (e.g., wings), convergent evolution is
            due to \ldots

            \hidebox{natural selection for similar traits in similar
                environments}

        \item Multiple mutations also cause homoplasy in DNA data
            \begin{center}
            \begin{tikzpicture}
            [xscale=0.6,yscale=0.3,auto=left,every node/.style={circle}]%,fill=blue!20}]
              \node [tnode](a) at (7, 1) {A};
              \node [tnode](b) at (7, 3) {T};
              \node [tnode](c) at (7, 5) {T};
              \node [tnode](d) at (7, 7) {A};
              % \node [inode](ab) at (2, 5)  {};
              % \node [inode](abc) at (3, 3)  {};
              % \node [inode](r) at (4, 1)  {};
            
              \draw [very thick] (a) -- (5,1) -- (5,3) -- (b);
              \draw [very thick] (5,2) -- (3,2) -- (3,5) -- (c);
              \draw [very thick] (3,3.5) -- (1,3.5) -- (1, 7) -- (d);
              \draw [very thick] (1,5.25) -- (0,5.25);
            \end{tikzpicture}
            \end{center}

        \item Why is homoplasy problematic?

            \hidebox{If we use convergent character states as if they are
                shared due to ancestry, we will get the wrong tree}
        \item How do we know if a particular trait is due to homology?

            \hidebox{Look at related extant taxa, fossils, or development to
                see if the traits are shared due to ancestry or not.}
    \end{enumerate}
\end{frame}

\end{document}

