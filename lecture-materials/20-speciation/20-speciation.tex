\documentclass[table]{beamer}
% \documentclass[table,handout]{beamer}
% \setbeameroption{show notes}
% \setbeameroption{hide notes}
% \setbeameroption{show only notes}
\usepackage{varwidth}

\newif\ifhide
\newif\ifpost
\newif\ifhideclicker

\hidetrue
% \hideclickertrue
% \posttrue

\newcommand{\whiteout}[1]{\textcolor{white}{#1}}
\newcommand{\whiteoutbox}[1]{\fcolorbox{white}{white}{\parbox{\dimexpr \linewidth-2\fboxsep-2\fboxrule}{\whiteout{#1}}}}
\newcommand{\notebox}[1]{\fcolorbox{blue}{white}{\parbox{\dimexpr \linewidth-2\fboxsep-2\fboxrule}{#1}}}

\ifhide%
    \newcommand{\hmask}[1]{\phantom{\varwidth{\linewidth}#1\endvarwidth}}%
\else%
    \newcommand{\hmask}[1]{#1}%
\fi

\ifhide%
    \newcommand{\hignore}[1]{}%
\else%
    \newcommand{\hignore}[1]{#1}%
\fi

\ifpost%
    \newcommand{\nopost}[1]{}%
\else%
    \newcommand{\nopost}[1]{#1}%
\fi

\ifhide%
    \newcommand{\hidebox}[1]{\phantom{\varwidth{\linewidth}#1\endvarwidth}}%
\else%
    \newcommand{\hidebox}[1]{\fbox{\parbox{\linewidth}{#1}}}%
\fi

\ifhide%
    \newcommand{\wbox}[1]{\whiteoutbox{#1}}%
\else%
    \newcommand{\wbox}[1]{\notebox{#1}}%
\fi

% \ifhide%
%     \newcommand{\clickeranswer}[1]{#1}%
% \else%
%     \newcommand{\clickeranswer}[1]{\textbf{\textcolor{blue}{#1}}}%
% \fi

\ifhideclicker%
    \newcommand{\clickeranswer}[1]{#1}%
\else%
    \ifhide%
        \newcommand{\clickeranswer}[1]{#1}%
    \else%
        \newcommand{\clickeranswer}[1]{\textbf{\textcolor{blue}{#1}}}%
    \fi
\fi

\input{../utils/slide-preamble2.tex}
\newcommand{\highlight}[1]{\textcolor{violet}{\textit{\textbf{#1}}}}
\newcommand{\super}[1]{\ensuremath{^{\textrm{#1}}}}
\newcommand{\sub}[1]{\ensuremath{_{\textrm{#1}}}}
\newcommand{\dC}{\ensuremath{^\circ{\textrm{C}}}}
\newcommand{\tb}{\hspace{2em}}
\providecommand{\e}[1]{\ensuremath{\times 10^{#1}}}
\newcommand{\myHangIndent}{\hangindent=5mm}

\makeatletter
\newcommand*{\rom}[1]{\expandafter\@slowromancap\romannumeral #1@}
\makeatother

\newcommand{\blankslide}{{\setbeamercolor{background canvas}{bg=black}
\setbeamercolor{whitetext}{fg=white}
\begin{frame}<handout:0>[plain]
\end{frame}}}

\newcommand{\whiteslide}{
\begin{frame}<handout:0>[plain]
\end{frame}}

\newcommand{\f}[1]{\ensuremath{F_{#1}}}

\bibliography{../bib/references}
\input{../utils/title-info.tex}

\title[Speciation]{Speciation}
% \date{\today}
\date{April 30, 2015}

\begin{document}

\begin{noheadline}
\maketitle
\end{noheadline}

\nopost{
\begin{noheadline}
\begin{frame}[c]
    \vspace{-6mm}
    \begin{center} 
        \includegraphics[height=1.2\textheight]{../images/seating-chart.pdf}
    \end{center}
\end{frame}
\end{noheadline}
}

\clickerslide{
\begin{frame}
    \begin{clickerquestion}
        \item Consider the wings of bats and birds; which of the following is
            correct?
        \begin{clickeroptions}
            \item Wings are homologous; limbs are homoplastic.
            \item Both have highly modified hand and finger bones; these
                modifications are homologous.
            \item \clickeranswer{Many genes for limb formation are homologous,
                    but the different alleles/genes that give rise to wings in
                    each are homoplastic.}
            \item None of the above.
        \end{clickeroptions}
    \end{clickerquestion}
\end{frame}
}

\begin{noheadline}
\begin{frame}
\frametitle{Today's issues:}
\vspace{5mm}
% \tableofcontents[subsectionstyle=hide]
\tableofcontents
\end{frame}
\end{noheadline}

\section{How do biologists recognize species?}

\begin{noheadline}
\begin{frame}[t]
    \frametitle{How do biologists recognize species?}
    \begin{adjustwidth}{-1.5em}{-1.5em}
        A species is a population, or group of populations, in which
        evolutionary forces are acting independently.

        \vspace{3mm}
        Speciation is a splitting event:

        \nbox{Note: because species can exist at large spatial scales, and the
            process of speciation occurs over long temporal scales, it is often
            difficult to delimit species in practice. I.e., species are easy to
            define, but hard to identify.}
        \vspace{2cm}
        Species are genetically isolated from each other.
    \end{adjustwidth}
\end{frame}
\end{noheadline}

\begin{frame}[t]
    \begin{adjustwidth}{-1.5em}{-1.5em}
        Speciation occurs via:

        \begin{enumerate}
            \item Genetic isolation (lack of gene flow), followed by \ldots
            \item Genetic divergence (due to mutation, drift, and selection)
        \end{enumerate}

        \begin{itemize}
            \item Why does genetic isolation lead to ``evolutionary independent
                units?''
                
                \nbox{Once gene flow stops, the allele frequencies in each new
                    species are independent of one another. I.e., they are
                    ``free'' to evolve independently via mutation, drift, and
                    selection.}

                \vspace{5mm}
            \item Why does genetic divergence create synapomorphies?

                \nbox{Mutation will introduce new alleles (and traits) that are
                    unique to each new species (i.e., there is no gene flow to
                    ``share'' them). Thus, the descendants of each species will
                    accumulate unique alleles and traits.}

        \end{itemize}
    \end{adjustwidth}
\end{frame}

\subsection{Species criteria (``concepts'')}

\begin{frame}[t]
    \begin{adjustwidth}{-2em}{-1.5em}

    \vspace{-3mm}
    \textbf{Biological species concept:}
    \vspace{-3mm}
    \begin{table}%[htbp]
        \centering
        \begin{tabular}{ L{3.7cm} | L{3.7cm} | L{3.7cm} }
            Criterion for identifying species & Advantages & Disadvantages \\
            \hline
            \cmask{\mybullet Reproductive isolation (don't produce fertile offspring)} &
            \cmask{\mybullet Sound theoretically (lack of gene flow = evolutionarily
                independent))} &
            \cmask{\mybullet Doesn't apply to asexual species} \\
            \cmask{} &
            \cmask{} &
            \cmask{\mybullet Doesn't apply to fossils} \\
            \cmask{} &
            \cmask{} &
            \cmask{\mybullet Can be difficult to test} \\
            \cmask{} & \cmask{} & \cmask{} \\
            \cmask{} & \cmask{} & \cmask{} \\
            \cmask{} & \cmask{} & \cmask{} \\
            \cmask{} & \cmask{} & \cmask{} \\
            \cmask{} & \cmask{} & \cmask{} \\
            \cmask{} & \cmask{} & \cmask{} \\
            \cmask{} & \cmask{} & \cmask{} \\
            \cmask{} & \cmask{} & \cmask{} \\

        \end{tabular}
    \end{table}
    \end{adjustwidth}
\end{frame}

\begin{frame}[t]
    \begin{adjustwidth}{-2em}{-1.5em}

    \vspace{-3mm}
    \textbf{Morphospecies concept:}
    \vspace{-3mm}
    \begin{table}%[htbp]
        \centering
        \begin{tabular}{ L{3.7cm} | L{3.7cm} | L{3.7cm} }
            Criterion for identifying species & Advantages & Disadvantages \\
            \hline
            \cmask{\mybullet Distinct morphologically} &
            \cmask{\mybullet Widely applicable (sexual, asexual, fossils)} &
            \cmask{\mybullet Subjective (experts disagree what qualifies as
                ``distinct''} \\
            \cmask{} & \cmask{} & \cmask{} \\
            \cmask{} & \cmask{} & \cmask{} \\
        \end{tabular}
    \end{table}

    \vspace{2mm}
    \textbf{Phylogenetic species concept:}
    \vspace{-3mm}
    \begin{table}%[htbp]
        \centering
        \begin{tabular}{ L{3.7cm} | L{3.7cm} | L{3.7cm} }
            Criterion for identifying species & Advantages & Disadvantages \\
            \hline
            \cmask{\mybullet Smallest monophyletic groups} &
            \cmask{\mybullet Widely applicable (sexual, asexual, fossils)} &
            \cmask{\mybullet Difficult to apply; need to collect data and estimate trees} \\
            \cmask{} &
            \cmask{\mybullet Objective and testable} &
            \cmask{} \\
            \cmask{} & \cmask{} & \cmask{} \\
            \cmask{} & \cmask{} & \cmask{} \\
        \end{tabular}
    \end{table}
    \end{adjustwidth}
\end{frame}

\begin{frame}[t]
    \frametitle{An example of applying species concepts}
    \begin{adjustwidth}{-2em}{-1.5em}

        \begin{center} 
            \includegraphics[width=0.8\linewidth]{elephant-tree-2-tips.png}
        \end{center}

        \begin{itemize}[<+->]
            \item Under morphospecies concept, African elephants were
                considered 1 species.
            \item Biological species concept is difficult to test with
                elephants!
        \end{itemize}
    \end{adjustwidth}
\end{frame}

\begin{frame}[t]
    \frametitle{An example of applying species concepts}
    \begin{adjustwidth}{-2em}{-1.5em}

        \begin{center} 
            \includegraphics[width=0.8\linewidth]{elephant-tree-3-tips.png}
        \end{center}

        \begin{itemize}[<+->]
            \item DNA data revealed western and eastern populations had
                distinct alleles (synapomorphies) and were each monophyletic
                (statistical test rejected that they were not monophyletic).
            \item So, they are separate species according to the phylogenetic
                species concept!
            \item NOTE: If there was gene flow between the east and west
            populations, they would not be monophyletic (there would not be
            synapomorphies).
        \end{itemize}
    \end{adjustwidth}
\end{frame}

\section{How does speciation occur?}

\subsection{Allopatric speciation}

\subsubsection{Dispersal}

\begin{frame}
    \frametitle{}
\end{frame}

\subsubsection{Vicariance}

\subsection{Sympatric speciation}

\subsection[]{Mechanisms of isolation}

\subsection[]{Mechanisms of divergence}


\end{document}

\clickerslide{
\begin{frame}
    \begin{clickerquestion}
        \item 
        \begin{clickeroptions}
            \item 
            \item 
            \item 
            \item 
        \end{clickeroptions}
    \end{clickerquestion}
\end{frame}
}
