\documentclass[table]{beamer}
% \documentclass[table,handout]{beamer}
% \setbeameroption{show notes}
% \setbeameroption{hide notes}
% \setbeameroption{show only notes}
\usepackage{varwidth}

\newif\ifhide
\newif\ifpost
\newif\ifhideclicker

\hidetrue
% \hideclickertrue
% \posttrue

\newcommand{\whiteout}[1]{\textcolor{white}{#1}}
\newcommand{\whiteoutbox}[1]{\fcolorbox{white}{white}{\parbox{\dimexpr \linewidth-2\fboxsep-2\fboxrule}{\whiteout{#1}}}}
\newcommand{\notebox}[1]{\fcolorbox{blue}{white}{\parbox{\dimexpr \linewidth-2\fboxsep-2\fboxrule}{#1}}}

\ifhide%
    \newcommand{\hmask}[1]{\phantom{\varwidth{\linewidth}#1\endvarwidth}}%
\else%
    \newcommand{\hmask}[1]{#1}%
\fi

\ifhide%
    \newcommand{\hignore}[1]{}%
\else%
    \newcommand{\hignore}[1]{#1}%
\fi

\ifpost%
    \newcommand{\nopost}[1]{}%
\else%
    \newcommand{\nopost}[1]{#1}%
\fi

\ifhide%
    \newcommand{\hidebox}[1]{\phantom{\varwidth{\linewidth}#1\endvarwidth}}%
\else%
    \newcommand{\hidebox}[1]{\fbox{\parbox{\linewidth}{#1}}}%
\fi

\ifhide%
    \newcommand{\wbox}[1]{\whiteoutbox{#1}}%
\else%
    \newcommand{\wbox}[1]{\notebox{#1}}%
\fi

% \ifhide%
%     \newcommand{\clickeranswer}[1]{#1}%
% \else%
%     \newcommand{\clickeranswer}[1]{\textbf{\textcolor{blue}{#1}}}%
% \fi

\ifhideclicker%
    \newcommand{\clickeranswer}[1]{#1}%
\else%
    \ifhide%
        \newcommand{\clickeranswer}[1]{#1}%
    \else%
        \newcommand{\clickeranswer}[1]{\textbf{\textcolor{blue}{#1}}}%
    \fi
\fi

\input{../utils/slide-preamble2.tex}
\newcommand{\highlight}[1]{\textcolor{violet}{\textit{\textbf{#1}}}}
\newcommand{\super}[1]{\ensuremath{^{\textrm{#1}}}}
\newcommand{\sub}[1]{\ensuremath{_{\textrm{#1}}}}
\newcommand{\dC}{\ensuremath{^\circ{\textrm{C}}}}
\newcommand{\tb}{\hspace{2em}}
\providecommand{\e}[1]{\ensuremath{\times 10^{#1}}}
\newcommand{\myHangIndent}{\hangindent=5mm}

\makeatletter
\newcommand*{\rom}[1]{\expandafter\@slowromancap\romannumeral #1@}
\makeatother

\newcommand{\blankslide}{{\setbeamercolor{background canvas}{bg=black}
\setbeamercolor{whitetext}{fg=white}
\begin{frame}<handout:0>[plain]
\end{frame}}}

\newcommand{\whiteslide}{
\begin{frame}<handout:0>[plain]
\end{frame}}

\newcommand{\f}[1]{\ensuremath{F_{#1}}}

\bibliography{../bib/references}
\input{../utils/title-info.tex}

\title[Communities I: Disturbance \& Succession]{Communities I: Disturbance \&
    Succession}
% \date{\today}
\date{May 27, 2015}

% \setbeamertemplate{section in toc}[sections numbered]
% \setbeamertemplate{subsection in toc}[subsections numbered]

\begin{document}

\begin{noheadline}
\maketitle
\end{noheadline}

\nopost{
\begin{noheadline}
\begin{frame}[c]
    \vspace{-6mm}
    \begin{center} 
        \includegraphics[height=1.2\textheight]{../images/seating-chart-2.pdf}
    \end{center}
\end{frame}
\end{noheadline}
}

\clickerslide{
\begin{noheadline}
\begin{frame}
    \begin{clickerquestion}
        \item Pollination is a well-studied mutualism. For example, yucca moths
            lay their eggs in the seeds of yucca plants, and their larva eat
            some of the seeds; yucca moths are the only pollinator of yucca
            flowers (overall, both parties benefit).  However, some species of
            yucca moth lack the specialized mouthparts required to transfer
            pollen, but still lay their eggs in yucca seeds. How would you
            characterize this interaction? 

        \begin{clickeroptions}
            \item Commensalism (+/0)
            \item Competition
            \item Mutualism
            \item \clickeranswer{Parasitism}
        \end{clickeroptions}
    \end{clickerquestion}
\end{frame}
\end{noheadline}
}

\begin{noheadline}
\begin{frame}
\frametitle{Today's issues:}
\vspace{5mm}
% \tableofcontents[subsectionstyle=hide]
\tableofcontents
\end{frame}
\end{noheadline}

\section{Community structure}

\begin{frame}[t]
    \begin{adjustwidth}{-2em}{-1.5em}
        \textbf{Community} = the collection of species in a particular area.

        \vspace{4mm}
        These species interact via:
        
        \begin{itemize}
            \item Mutualism
            \item Competition
            \item Consumption
            \item Commensalism
        \end{itemize}

        \uncover<2->{
        Key idea with coevolution and arms races:

        \begin{itemize}
            \item Reciprocal adaptation =

            \nbox{Each species is an agent of selection on the species it
                interacts with (and vice versa). An allele in one species can
                change the selection on alleles of other species (and vice
                versa). Thus, adaptation in one species is contingent on the
                adaptations of its interacting species.}
        \end{itemize}
        }

        \uncover<3->{
        Big question: How do all these interactions and coevolution affect
        the structure of communities?
        }

    \end{adjustwidth}
\end{frame}

\begin{frame}[t]
    \begin{adjustwidth}{-2em}{-1.5em}
        Hypothesis 1 (Clements): Communities are highly structured

        \begin{itemize}
            \item The species present in an area are there because they ``have
                to be''; their niches comprise a highly integrated,
                interdependent unit.
        \end{itemize}

        \uncover<2->{
        \vspace{5mm}
        Hypothesis 2 (Gleason): Communities are random assemblages of species

        \begin{itemize}
            \item The species present in an area are there because they
                ``happen to be''; community composition is largely due to
                chance events (history).
        \end{itemize}
        }

        \uncover<3->{
        \vspace{5mm}
        In terms of statistical testing, which of these hypotheses is a better
        null?

        \nbox{Hypothesis 2; random outcomes (expectations) are easy to quantify
            using probability}
        }

    \end{adjustwidth}
\end{frame}

\clickerslide{
\begin{frame}
    \begin{clickerquestion}
        \item Geographic ranges are moving in response to global warming.
            Given these changes, what prediction does the ``niche hypothesis''
            (Clements) make?
        \begin{clickeroptions}
            \item Species should move independently; community structure should
                change dramatically.
            \item \clickeranswer{Species should move together; communities
                    should remain largely intact.}
            \item Alleles associated with temperature tolerance should
                increase in frequency.
            \item Some species will go extinct.
        \end{clickeroptions}
    \end{clickerquestion}
\end{frame}
}

\begin{frame}[t]
    \begin{adjustwidth}{-2em}{-1.5em}
        \begin{itemize}
            \item Mount St.\ Helens (post-eruption): Very harsh environment
                (resources extremely scarce).

            \item Tropical rainforest: Stable conditions and abundant
                resources.
        \end{itemize}
        
        In which environment are ``niche'' vs.\ ``random'' processes more
        important?

        \nbox{Niche processes (Clements' hypothesis) is more important on Mount
            St.\ Helens (the adaptations of species will be very important in
            determining what species can colonize such a harsh environment).
            Random processes (Gleason's hypothesis) is more important in
            tropical rainforests (which species colonize and grow in any given
            spot is largely due to chance).}

        % \vspace{8mm}
        Based on the evidence to date (and in the reading), which hypothesis
        (``niche'' or ``random'') is most consistent with observations?

        \nbox{Somewhere in the middle, but probably closer to ``random''
            (Gleason) hypothesis in general. However, this varies depending on
            the system and scale.}

    \end{adjustwidth}
\end{frame}

\section{How does disturbance affect communities?}

\begin{frame}[t]
    \begin{adjustwidth}{-2em}{-1.5em}
        How does disturbance affect communities?

        \vspace{2mm}
        \textbf{Disturbance} = any change that removes biomass.

        \vspace{2mm}
        Types of disturbance:
        \nbox{Fire, flood, wind, volcanism, logging, landslide, tree fall,
            etc.}

    \end{adjustwidth}

    \note[item]{What is biomass? The amount of organic material}
\end{frame}

\begin{frame}[t]
    \begin{adjustwidth}{-2em}{-1.5em}
        To understand disturbance, you need to analyze:
        \begin{enumerate}
            \item Type \cmask{What biomass and how removed?}
            \item Severity \cmask{How much biomass removed?}
            \item Frequency \cmask{How often does it occur?}
        \end{enumerate}

        \begin{uncoverenv}<2->
        Examples:
        \begin{itemize}
            \item Boreal forests burn to the ground every 100--300 years

                \vspace{4mm}
            \item Grasslands and savannas: fires every 2--3 years.

                \vspace{4mm}
            \item Freshwater systems: annual flooding.

                \vspace{4mm}
            \item Tropical and temperate rainforests: frequently, gaps in
                forest are created by fallen trees (windblown or diseased).
        \end{itemize}
        \end{uncoverenv}

        \begin{uncoverenv}<3->
        Humans have prevented these natural disturbances.  How would you mimic
        these disturbances?

        \nbox{\scriptsize Engineered floods, prescribed (controlled) burns,
            smart forestry practices to mimic natural tree-fall gaps}
        \end{uncoverenv}

    \end{adjustwidth}
\end{frame}

\begin{frame}[t]
    \begin{adjustwidth}{-2em}{-1.5em}
        Consider species from fire-prone ecosystems. How are the traits below
        adaptive?
        \begin{itemize}
            \item Lodgepole pines require full sun to thrive, and their cones
                are sealed shut with resin.
                
                \nbox{Fire melts resin, seeds fall and sprout. Only release
                    seeds when conditions are best (no competitors around; lots
                    of sun).}

            \item In most prairie (grassland) plants, the stem cells
                responsible for new growth are located below ground.

                \nbox{Plants can survive and regrow after a fire.}

            \item Some species in California chaparral have seeds that must be
                exposed to smoke/ash to germinate.

                \nbox{Fire makes space and removes competitors. Seeds only
                    germinate and grow when conditions are best (no competitors
                    around; lots of sun).}

            \item Burr oak trees have extraordinarily thick bark compared to
                other oak species of similar size.

                \nbox{Tissues are protected from fire and heat inside the thick
                    bark.}
        \end{itemize}

    \end{adjustwidth}
\end{frame}

\clickerslide{
\begin{frame}
    \begin{clickerquestion}
        \item Compare and contrast two types of disturbance that can occur in
            the same forest habitats along rivers: flooding and fires. 

        \begin{clickeroptions}
            \item \clickeranswer{Depending on severity, both can remove biomass
                    just on the forest floor or in the canopy (trees) as well.}
            \item Only fire adds nutrients.
            \item Most fires are about equal in severity; flooding varies
                widely in severity.
            \item Most floods are about equal in severity; fires vary widely in
                severity.
        \end{clickeroptions}
    \end{clickerquestion}
\end{frame}
}

\begin{frame}[t]
    \begin{adjustwidth}{-2em}{-1.5em}
        A case history: Forest fires in the Rocky Mountains

        \begin{itemize}[<+->]
            \item Historically, most regions burned every 60--100 years.

                \vspace{5mm}
            \item Most forests have now gone $>$120 years without a fire.

                \vspace{5mm}
            \item What are the likely consequences of this change in
                disturbance regime?

                \nbox{Build up of biomass (fuel) will lead to extremely hot and
                    extensive fires---these could devastate fores communities
                    (too hot for species' adaptations)}
        \end{itemize}
    \end{adjustwidth}
    \note[item]{One of the biggest ecological problems in West United States:
        Preventing disturbance regime in the Rocky Mountains (Smokey the
        bear).}
\end{frame}

\section{How do communities respond to disturbance?}

\begin{frame}[t]
    \begin{adjustwidth}{-2em}{-1.5em}
        How do communities respond to disturbance?

        \vspace{4mm}
        \textbf{Succession} refers to the sequence of species that occupy a
        site through time, after a disturbance.

        \begin{uncoverenv}<2->
        \vspace{4mm}
        To study how and why succession occurs at a particular location, we
        need to focus on:

        \begin{enumerate}
            \item Species traits
            \item Species interactions
            \item History of the site
        \end{enumerate}
        \end{uncoverenv}
    \end{adjustwidth}
\end{frame}

\begin{frame}[t]
    \begin{adjustwidth}{-2em}{-1.5em}
        Species traits:
        \begin{table}%[htbp]
            \centering
            \begin{tabular}{ L{5.2cm} | L{5.2cm} }
                \multicolumn{1}{c}{Early successional species} &
                \multicolumn{1}{c}{Late successional species} \\
                \hline
                Small seeds (good colonizers) &
                Large seeds (good competitors) \\[3ex]
                Thrive in harsh abiotic conditions (dry, low nutrients, high
                light, high variation in temperature and wind) &
                Require moisture, more organic nutrients in soil, less variable
                temperature/wind; can cope with low light \\[3ex]
                Rapid growth and reproduction; short-lived &
                Slow growth and reproduction; long-lived \\
            \end{tabular}
        \end{table}

        \uncover<2->{
        \vspace{4mm}
        Why do these life-history traits make sense?
        }
    \end{adjustwidth}
\end{frame}

\begin{frame}[t]
    \begin{adjustwidth}{-2em}{-1.5em}
        Species interactions; key observations:

        \begin{enumerate}
            \item Early successional species alter the physical environment in
                ways that favor (facilitate) late successional species.

                \nbox{Early successional species are good dispersers (they have
                    to get there!), and grow quickly and reproduce in harsh
                    conditions with very little nutrients/water. They can do
                    this by investing very little in biomass and survivorship,
                    and a lot in reproduction.  After enough of them colonize,
                    they create shade, hold in moisture, and add organic
                    nutrients to the soil---they facilitate conditions that are
                    good for late successional species.}

            \item Once basic resources are available (moisture/organic
                nutrients), late successional species are much better
                competitors than early successional species.
        \end{enumerate}
    \end{adjustwidth}
\end{frame}

\clickerslide{
\begin{frame}
    \begin{clickerquestion}
        \item Why aren't early successional species also good competitors? 

        \begin{clickeroptions}
            \item Many are---for example, weeds are everywhere. 
            \item They have narrow fundamental niches.
            \item \clickeranswer{There are fitness trade-offs.} 
            \item They have narrow realized niches. 
        \end{clickeroptions}
    \end{clickerquestion}

    \nbox{To have high fitness in harsh, post-disturbance environments with
        little nutrients, requires life-history strategies that lead to
        trade-offs for poor competitiveness (reproducing as much and as quickly
        as possible while investing as little as possible in
        biomass/survivorship).}

\end{frame}
}

\end{document}

\clickerslide{
\begin{frame}
    \begin{clickerquestion}
        \item 
        \begin{clickeroptions}
            \item 
            \item 
            \item 
            \item 
        \end{clickeroptions}
    \end{clickerquestion}
\end{frame}
}

\clickerpost{
{
\usebackgroundtemplate{\includegraphics[page=17,width=\paperwidth]{./24-Radiation-extinction.pdf}}
\begin{frame}[t,plain]
    \begin{adjustwidth}{-2em}{-1.5em}
        \cmask{Answer: 3}
    \end{adjustwidth}
\end{frame}
}
}

