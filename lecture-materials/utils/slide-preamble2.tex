\usepackage{beamerthemesplit}
% \usetheme{boxes}
\usetheme{Malmoe}
\usecolortheme{seahorse}
% \usecolortheme{seagull}
\usepackage{ifthen}
\usepackage{xspace}
\usepackage{multirow}
\usepackage{multicol}
\usepackage{booktabs}
\usepackage{xcolor}
\usepackage{wasysym}
\usepackage{comment}
\usepackage{hyperref}
\hypersetup{pdfborder={0 0 0}, colorlinks=true, urlcolor=blue, linkcolor=blue, citecolor=blue}
\usepackage{changepage}
\usepackage[compatibility=false]{caption}
\captionsetup[figure]{font=scriptsize, labelformat=empty, textformat=simple, justification=centering, skip=2pt}
\usepackage{tikz}
\usetikzlibrary{trees,calc,backgrounds}

\usepackage[bibstyle=joaks-slides,maxcitenames=3,mincitenames=1,backend=biber]{biblatex}

\newrobustcmd*{\shortfullcite}{\AtNextCite{\renewbibmacro{title}{}\renewbibmacro{in:}{}\renewbibmacro{number}{}}\fullcite}

\newrobustcmd*{\footlessfullcite}{\AtNextCite{\renewbibmacro{title}{}\renewbibmacro{in:}{}}\footfullcite}

% Make all footnotes smaller
% \renewcommand{\footnotesize}{\scriptsize}

\definecolor{myGray}{gray}{0.9}
\colorlet{rowred}{red!30!white}

\setbeamertemplate{blocks}[rounded][shadow=true]

\setbeamercolor{defaultcolor}{bg=structure!30!normal text.bg,fg=black}
\setbeamercolor{block body}{bg=structure!30!normal text.bg,fg=black}
\setbeamercolor{block title}{bg=structure!50!normal text.bg,fg=black}

\newenvironment<>{varblock}[2][\textwidth]{%
  \setlength{\textwidth}{#1}
  \begin{actionenv}#3%
    \def\insertblocktitle{#2}%
    \par%
    \usebeamertemplate{block begin}}
  {\par%
    \usebeamertemplate{block end}%
  \end{actionenv}}

\newenvironment{displaybox}[1][\textwidth]
{
    \centerline\bgroup\hfill
    \begin{beamerboxesrounded}[lower=defaultcolor,shadow=true,width=#1]{}
}
{
    \end{beamerboxesrounded}\hfill\egroup
}

\newenvironment{onlinebox}[1][4cm]
{
    \newbox\mybox
    \newdimen\myboxht
    \setbox\mybox\hbox\bgroup%
        \begin{beamerboxesrounded}[lower=defaultcolor,shadow=true,width=#1]{}
    \centering
}
{
    \end{beamerboxesrounded}\egroup
    \myboxht\ht\mybox
    \raisebox{-0.25\myboxht}{\usebox\mybox}\hspace{2pt}
}

\newenvironment{mydescription}{
    \begin{description}
        \setlength{\leftskip}{-1.5cm}}
    {\end{description}}

\newenvironment{myitemize}{
    \begin{itemize}
        \setlength{\leftskip}{-.3cm}}
    {\end{itemize}}

% footnote without a marker
\newcommand\barefootnote[1]{%
  \begingroup
  \renewcommand\thefootnote{}\footnote{#1}%
  \addtocounter{footnote}{-1}%
  \endgroup
}

% define formatting for footer
\newcommand{\myfootline}{%
    {\it
    \insertshorttitle
    \hspace*{\fill} 
    \insertshortauthor, \insertshortinstitute
    % \ifx\insertsubtitle\@empty\else, \insertshortsubtitle\fi
    \hspace*{\fill}
    \insertframenumber/\inserttotalframenumber}}

% set up footer
\setbeamertemplate{footline}{%
    \usebeamerfont{structure}
    \begin{beamercolorbox}[wd=\paperwidth,ht=2.25ex,dp=1ex]{frametitle}%
        % \Tiny\hspace*{4mm}\myfootline\hspace{4mm}
        \tiny\hspace*{4mm}\myfootline\hspace{4mm}
    \end{beamercolorbox}}

% remove navigation bar
\beamertemplatenavigationsymbolsempty

\makeatletter
    \newenvironment{noheadline}{
        \setbeamertemplate{headline}[default]
        \def\beamer@entrycode{\vspace*{-\headheight}}
    }{}
\makeatother

\newcounter{clickerQuestionCounter}
\ifhideclicker%
\newenvironment{clickerquestion}
{ \stepcounter{clickerQuestionCounter}
  \begin{enumerate}[Q \arabic{clickerQuestionCounter}:]\color{white} }
{ \end{enumerate} }
\else%
\newenvironment{clickerquestion}
{ \stepcounter{clickerQuestionCounter}
  \begin{enumerate}[Q \arabic{clickerQuestionCounter}:] }
{ \end{enumerate} }
\fi

\ifhideclicker%
\newenvironment{clickeroptions}
{ \begin{enumerate}[\begingroup\color{white} 1)\endgroup]\color{white} }
{ \end{enumerate} }
\else%
\newenvironment{clickeroptions}
{ \begin{enumerate}[\begingroup\color{red} 1)\endgroup] }
{ \end{enumerate} }
\fi


\tikzstyle{centered} = [align=center, text centered, font=\sffamily\bfseries]
\tikzstyle{skip} = [centered, inner sep=0pt, fill]
\tikzstyle{empty} = [centered, inner sep=0pt]
\tikzstyle{inode} = [centered, circle, minimum width=4pt, fill=black, inner sep=0pt]
\tikzstyle{tnode} = [centered, circle, inner sep=1pt]
\tikzset{
  % edge styles
  level distance=10mm,
  mate/.style={edge from parent/.style={draw,distance=3pt}},
  mleft/.style={grow=left, level distance=10mm, edge from parent path={(\tikzparentnode.west)--(\tikzchildnode.east)}},
  mright/.style={grow=right, level distance=10mm, edge from parent path={(\tikzparentnode.east)--(\tikzchildnode.west)}},
  % node styles
  male/.style={rectangle,minimum size=4mm,fill=gray!80},
  female/.style={circle,minimum size=4mm,fill=gray!80},
  amale/.style={male,fill=red},
  afemale/.style={female,fill=red},
}
