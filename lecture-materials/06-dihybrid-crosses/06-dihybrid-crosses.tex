\documentclass[table]{beamer}
% \documentclass[table,handout]{beamer}
% \setbeameroption{show notes}
% \setbeameroption{hide notes}
% \setbeameroption{show only notes}
\usepackage{varwidth}

\newif\ifhide
\newif\ifpost
\newif\ifhideclicker

\hidetrue
% \hideclickertrue
% \posttrue

\newcommand{\whiteout}[1]{\textcolor{white}{#1}}
\newcommand{\whiteoutbox}[1]{\fcolorbox{white}{white}{\parbox{\dimexpr \linewidth-2\fboxsep-2\fboxrule}{\whiteout{#1}}}}
\newcommand{\notebox}[1]{\fcolorbox{blue}{white}{\parbox{\dimexpr \linewidth-2\fboxsep-2\fboxrule}{#1}}}

\ifhide%
    \newcommand{\hmask}[1]{\phantom{\varwidth{\linewidth}#1\endvarwidth}}%
\else%
    \newcommand{\hmask}[1]{#1}%
\fi

\ifhide%
    \newcommand{\hignore}[1]{}%
\else%
    \newcommand{\hignore}[1]{#1}%
\fi

\ifpost%
    \newcommand{\nopost}[1]{}%
\else%
    \newcommand{\nopost}[1]{#1}%
\fi

\ifhide%
    \newcommand{\hidebox}[1]{\phantom{\varwidth{\linewidth}#1\endvarwidth}}%
\else%
    \newcommand{\hidebox}[1]{\fbox{\parbox{\linewidth}{#1}}}%
\fi

\ifhide%
    \newcommand{\wbox}[1]{\whiteoutbox{#1}}%
\else%
    \newcommand{\wbox}[1]{\notebox{#1}}%
\fi

% \ifhide%
%     \newcommand{\clickeranswer}[1]{#1}%
% \else%
%     \newcommand{\clickeranswer}[1]{\textbf{\textcolor{blue}{#1}}}%
% \fi

\ifhideclicker%
    \newcommand{\clickeranswer}[1]{#1}%
\else%
    \ifhide%
        \newcommand{\clickeranswer}[1]{#1}%
    \else%
        \newcommand{\clickeranswer}[1]{\textbf{\textcolor{blue}{#1}}}%
    \fi
\fi

\input{../utils/slide-preamble2.tex}
\newcommand{\highlight}[1]{\textcolor{violet}{\textit{\textbf{#1}}}}
\newcommand{\super}[1]{\ensuremath{^{\textrm{#1}}}}
\newcommand{\sub}[1]{\ensuremath{_{\textrm{#1}}}}
\newcommand{\dC}{\ensuremath{^\circ{\textrm{C}}}}
\newcommand{\tb}{\hspace{2em}}
\providecommand{\e}[1]{\ensuremath{\times 10^{#1}}}
\newcommand{\myHangIndent}{\hangindent=5mm}

\makeatletter
\newcommand*{\rom}[1]{\expandafter\@slowromancap\romannumeral #1@}
\makeatother

\newcommand{\blankslide}{{\setbeamercolor{background canvas}{bg=black}
\setbeamercolor{whitetext}{fg=white}
\begin{frame}<handout:0>[plain]
\end{frame}}}

\newcommand{\whiteslide}{
\begin{frame}<handout:0>[plain]
\end{frame}}

\newcommand{\f}[1]{\ensuremath{F_{#1}}}

\bibliography{../bib/references}
\input{../utils/title-info.tex}

\title[Dihybrid Crosses; Mitosis \& Meiosis]{Dihybrid Crosses; Mitosis \& Meiosis}
% \date{\today}
\date{April 7, 2015}

\begin{document}

\begin{noheadline}
\maketitle
\end{noheadline}

\nopost{
\begin{noheadline}
\begin{frame}[c]
    \vspace{-6mm}
    \begin{center} 
        \includegraphics[height=1.3\textheight]{../images/seating-chart.pdf}
    \end{center}
\end{frame}
\end{noheadline}
}


\begin{noheadline}
\begin{frame}
\frametitle{Today's issues:}
% \tableofcontents[subsectionstyle=hide]
\begin{enumerate}
    \item How are alleles from different genes transmitted to offspring
        (together, or independently of each other?
    \item What is the physical basis of Mendel's rules?
        \begin{enumerate}
            \item Mitosis: How cells divide during asexual reproduction and
                during growth
            \item Meiosis: How cells divide prior to formation of eggs and
                sperm (sexual reproduction)
        \end{enumerate}
\end{enumerate}
\end{frame}
\end{noheadline}

\section{Transmission of alleles from different genes}

\begin{noheadline}
\begin{frame}[t]
    \frametitle{Interpreting a dihybrid cross}
    \vspace{-5mm}
    \begin{clickerquestion}
        \item Predict results of this cross, \uppercase{assuming the alleles of
                different genes do \highlight{not} stay together}
    \end{clickerquestion}

    \vspace{-2mm}
    \begin{center}
        $PPTT \,\textrm{\male} \times pptt \,\textrm{\female}$
    \end{center}
    \vspace{-3mm}
    Gamete genotypes: \hmask{\highlight{All $PT$ for male, and all $pt$ for female}} \\
    \f{1} genotypes: \hmask{\highlight{All $PpTt$}} \\
    % What happens when an \f{1} offspring self-fertilizes? \\
    % \f{1} parental genotypes: \hmask{\highlight{$PT//pt \times PT//pt$}} \\
    \f{1} gamete genotypes: \hmask{\highlight{$\frac{1}{4}PT: \frac{1}{4}Pt : \frac{1}{4}pT : \frac{1}{4}pt$}} \\
    Punnett square (\f{1} $\times$ \f{1}):

    \vspace{-2mm}
    \begin{table}%[htbp]
        \centering
        \begin{tabular}{ l | l l l l}
            & \hmask{\highlight{$PT$}} & \hmask{\highlight{$Pt$}} & \hmask{\highlight{$pT$}} & \hmask{\highlight{$pt$}} \\
            \hline
            \hmask{\highlight{$PT$}} & \hmask{\highlight{$PPTT$}} & \hmask{\highlight{$PPTt$}} & \hmask{\highlight{$PpTT$}} & \hmask{\highlight{$PpTt$}} \\
            \hmask{\highlight{$Pt$}} & \hmask{\highlight{$PPTt$}} & \hmask{\highlight{$PPtt$}} & \hmask{\highlight{$PpTt$}} & \hmask{\highlight{$Pptt$}} \\
            \hmask{\highlight{$pT$}} & \hmask{\highlight{$PpTT$}} & \hmask{\highlight{$PpTt$}} & \hmask{\highlight{$ppTT$}} & \hmask{\highlight{$ppTt$}} \\
            \hmask{\highlight{$pt$}} & \hmask{\highlight{$PpTt$}} & \hmask{\highlight{$Pptt$}} & \hmask{\highlight{$ppTt$}} & \hmask{\highlight{$pptt$}} \\
        \end{tabular}
    \end{table}
    \vspace{-2mm}

    Ratio of \f{2} phenotypes:
    \begin{clickeroptions}
        \item 3 purple/tall: 1 white/dwarf
        \item 1 purple/tall: 1 white/dwarf
        \item 9 purple/tall: 1 purple/dwarf: 1 white/tall: 1 white/dwarf
        \item \clickeranswer{9 purple/tall: 3 purple/dwarf: 3 white/tall: 1 white/dwarf}
    \end{clickeroptions}
\end{frame}
\end{noheadline}

\begin{frame}[t]
    \frametitle{Mendel's actual results}
    \vspace{-6mm}
    \begin{center}
        $YYRR  \times yyrr$
    \end{center}
    \vspace{-4mm}
    \begin{table}%[htbp]
        \centering
        \begin{tabular}{l l}
            $Y$ = yellow & $R$ = round \\
            $y$ = green & $r$ = wrinkled \\
        \end{tabular}
    \end{table}

    \vspace{-2mm}
    \uncover<2->{
    All \f{1}s had yellow and round seeds ($YyRr$) \\
    }

    \vspace{2mm}
    \uncover<3->{
    When \f{1}s self-fertilized, phenotypes of \f{2} offspring were:
    \vspace{-3mm}
    \begin{table}%[htbp]
        \centering
        \begin{tabular}{l l}
            yellow-round & 315 \\
            green-round & 108 \\
            yellow-wrinkled & 101 \\
            green-wrinkled & 32 \\
            TOTAL & 556 \\
        \end{tabular}
    \end{table}
    }

    \vspace{-3mm}
    \uncover<4->{
    \begin{clickerquestion}
        \item What are the \highlight{frequencies} of the observed phenotypes?
            What hypothesis does this experiment support?
        \begin{clickeroptions}
            \item 9 y-r: 3 g-r: 3 y-w: 1 g-w; independent assortment
            \item 9 y-r: 3 g-r: 3 y-w: 1 g-w; dependent assortment
            \item \clickeranswer{0.57 y-r, 0.19 g-r, 0.18 y-w, 0.06 g-w; independent assortment}
            \item 0.57 y-r, 0.19 g-r, 0.18 y-w, 0.06 g-w; dependent assortment
        \end{clickeroptions}
    \end{clickerquestion}
    }
\end{frame}

\begin{frame}
    \frametitle{Mendel's actual results}
    \begin{itemize}[<+->]
        \item \highlight{Note:} Other combinations of two traits behaved the
            same way.
        \item Mendel's conclusion: The segregation of alleles of different
            genes occurs independently of each other.
            \begin{itemize}
                \item \textbf{This is the principle of independent assortment}
            \end{itemize}
        \item \highlight{Note:} The principle of independent assortment is
            misleading.
    \end{itemize}

    \note[item]{Do the ``sometimes principle'' and Cookie Monster}
\end{frame}


\section{The physical basis of Mendel's rules}

\begin{frame}
    \begin{itemize}
        \item To understand the physical basis of Mendel's rules and how
            genetic variation is generated in populations, we need to examine
            two types of cell division that were first described in the late
            1800s. 
        \item I. Mitosis: The process responsible for asexual reproduction and
            growth in multicellular organisms. 

        \item How does it happen? 
        \item What are chromosomes (``colored bodies'')?
    \end{itemize}
\end{frame}

\begin{frame}
    \begin{itemize}
        \item Chromosomes come in different types, distinguished by size and
            shape (morphology). Draw 5:
        \vspace{2.5cm}
        \item In many organisms, there are 2 of each type of chromosome
            present. Draw the second of each type (above).
        \item Pairs of each type are said to be homologous (they are homologs).
            Circle the 5 pairs of homologs.
    \end{itemize}
\end{frame}

\begin{frame}
    \frametitle{Key notation:}
    \begin{itemize}
        \item Use $n$ to indicate the \highlight{number of different types} of
            chromosomes found in a species
        \item Use a numeral before the $n$ to indicate \highlight{the number of
                each type} present
        \item Which is the ploidy? \nbox{number before $n$; e.g., $\highlight{2}n=46$}
        \item Which is the haploid number? \nbox{$n$}; e.g., $2\highlight{n}=46$, $\highlight{n=23}$}
        \item What's the difference between a chromosome and a chromatid?
    \end{itemize}
\end{frame}


\subsection{Mitosis}

\subsection{Meiosis}


\end{document}

\begin{noheadline}
\begin{frame}
    \begin{clickerquestion}
        \item 
        \begin{clickeroptions}
            \item 
            \item 
            \item 
            \item 
        \end{clickeroptions}
    \end{clickerquestion}
\end{frame}
\end{noheadline}
