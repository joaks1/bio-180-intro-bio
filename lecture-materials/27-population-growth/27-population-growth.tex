\documentclass[table]{beamer}
% \documentclass[table,handout]{beamer}
% \setbeameroption{show notes}
% \setbeameroption{hide notes}
% \setbeameroption{show only notes}
\usepackage{varwidth}

\newif\ifhide
\newif\ifpost
\newif\ifhideclicker

\hidetrue
% \hideclickertrue
% \posttrue

\newcommand{\whiteout}[1]{\textcolor{white}{#1}}
\newcommand{\whiteoutbox}[1]{\fcolorbox{white}{white}{\parbox{\dimexpr \linewidth-2\fboxsep-2\fboxrule}{\whiteout{#1}}}}
\newcommand{\notebox}[1]{\fcolorbox{blue}{white}{\parbox{\dimexpr \linewidth-2\fboxsep-2\fboxrule}{#1}}}

\ifhide%
    \newcommand{\hmask}[1]{\phantom{\varwidth{\linewidth}#1\endvarwidth}}%
\else%
    \newcommand{\hmask}[1]{#1}%
\fi

\ifhide%
    \newcommand{\hignore}[1]{}%
\else%
    \newcommand{\hignore}[1]{#1}%
\fi

\ifpost%
    \newcommand{\nopost}[1]{}%
\else%
    \newcommand{\nopost}[1]{#1}%
\fi

\ifhide%
    \newcommand{\hidebox}[1]{\phantom{\varwidth{\linewidth}#1\endvarwidth}}%
\else%
    \newcommand{\hidebox}[1]{\fbox{\parbox{\linewidth}{#1}}}%
\fi

\ifhide%
    \newcommand{\wbox}[1]{\whiteoutbox{#1}}%
\else%
    \newcommand{\wbox}[1]{\notebox{#1}}%
\fi

% \ifhide%
%     \newcommand{\clickeranswer}[1]{#1}%
% \else%
%     \newcommand{\clickeranswer}[1]{\textbf{\textcolor{blue}{#1}}}%
% \fi

\ifhideclicker%
    \newcommand{\clickeranswer}[1]{#1}%
\else%
    \ifhide%
        \newcommand{\clickeranswer}[1]{#1}%
    \else%
        \newcommand{\clickeranswer}[1]{\textbf{\textcolor{blue}{#1}}}%
    \fi
\fi

\input{../utils/slide-preamble2.tex}
\newcommand{\highlight}[1]{\textcolor{violet}{\textit{\textbf{#1}}}}
\newcommand{\super}[1]{\ensuremath{^{\textrm{#1}}}}
\newcommand{\sub}[1]{\ensuremath{_{\textrm{#1}}}}
\newcommand{\dC}{\ensuremath{^\circ{\textrm{C}}}}
\newcommand{\tb}{\hspace{2em}}
\providecommand{\e}[1]{\ensuremath{\times 10^{#1}}}
\newcommand{\myHangIndent}{\hangindent=5mm}

\makeatletter
\newcommand*{\rom}[1]{\expandafter\@slowromancap\romannumeral #1@}
\makeatother

\newcommand{\blankslide}{{\setbeamercolor{background canvas}{bg=black}
\setbeamercolor{whitetext}{fg=white}
\begin{frame}<handout:0>[plain]
\end{frame}}}

\newcommand{\whiteslide}{
\begin{frame}<handout:0>[plain]
\end{frame}}

\newcommand{\f}[1]{\ensuremath{F_{#1}}}

\bibliography{../bib/references}
\input{../utils/title-info.tex}

\title[Population growth]{Population growth}
% \date{\today}
\date{May 13, 2015}


% \setbeamertemplate{section in toc}[sections numbered]
% \setbeamertemplate{subsection in toc}[subsections numbered]

\begin{document}

\begin{noheadline}
\maketitle
\end{noheadline}


\nopost{
\begin{noheadline}
\begin{frame}[c]
    \vspace{-6mm}
    \begin{center} 
        \includegraphics[height=1.2\textheight]{../images/seating-chart-2.pdf}
    \end{center}
\end{frame}
\end{noheadline}
}

\clickerslide{
\begin{frame}
    \begin{clickerquestion}
        \item In the birds called phalaropes, females are much more
            brightly colored than males. They fight over males and mate with
            several different males each breeding season.  Which of the
            following best explains these observations? 

        \begin{clickeroptions}
            \item \clickeranswer{Males do all parental care (incubate eggs,
                    feed young).}
            \item Females are larger than males. 
            \item Phalaropes often swim in a circle while feeding, forming a
                whirlpool that brings prey to the surface. 
            \item Outside of the breeding season, phalaropes are pelagic (they
                live out at sea). 
            \item They breed at high latitudes; females migrate south before
                males. 
        \end{clickeroptions}
    \end{clickerquestion}
\end{frame}
}

\begin{noheadline}
\begin{frame}
\frametitle{Today's issues:}
\vspace{5mm}
% \tableofcontents[subsectionstyle=hide]
\tableofcontents
\end{frame}
\end{noheadline}

\section{What is population ecology?}

\begin{frame}
    \frametitle{What is population ecology?}
    \begin{adjustwidth}{-2em}{-1.5em}
        Members of the same population live in the same area at the
        same time

        % CHECK THIS!!!!
        \begin{itemize}
            \item Interbreed on a regular basis

                \vspace{5mm}
            \item Are exposed to a similar environment and selection pressures

                \vspace{5mm}
            \item Interact via intra-specific competition
        \end{itemize}
    \end{adjustwidth}
\end{frame}

\section[Basic models of population growth]{What are the basic models used to describe population growth?}

\begin{frame}
    \begin{adjustwidth}{-2em}{-1.5em}
        \vspace{-1mm}
        What are the basic models used to describe population growth?

        \vspace{2mm}
        Data on Whooping Cranes of Wood Buffalo National Park and Aransas NWR

        \vspace{-1.1cm}
        \begin{figure}
        \begin{center}
            \includegraphics[width=0.75\linewidth]{crane-growth.pdf}
        \end{center}
        \end{figure}

        \vspace{-8mm}
        \barefootnote{\tiny Data from COSEWIC 2010 and Whooping Crane
                Conservation Society}
    \end{adjustwidth}
\end{frame}
\section{Case studies in population growth}

\subsection{Density dependence}

\subsection{Population cycles}

\end{document}

\clickerslide{
\begin{frame}
    \begin{clickerquestion}
        \item 
        \begin{clickeroptions}
            \item 
            \item 
            \item 
            \item 
        \end{clickeroptions}
    \end{clickerquestion}
\end{frame}
}

\clickerpost{
{
\usebackgroundtemplate{\includegraphics[page=17,width=\paperwidth]{./24-Radiation-extinction.pdf}}
\begin{frame}[t,plain]
    \begin{adjustwidth}{-2em}{-1.5em}
        \cmask{Answer: 3}
    \end{adjustwidth}
\end{frame}
}
}

