\documentclass[table]{beamer}
% \documentclass[table,handout]{beamer}
% \setbeameroption{show notes}
% \setbeameroption{hide notes}
% \setbeameroption{show only notes}
\usepackage{varwidth}

\newif\ifhide
\newif\ifpost
\newif\ifhideclicker

\hidetrue
% \hideclickertrue
% \posttrue

\newcommand{\whiteout}[1]{\textcolor{white}{#1}}
\newcommand{\whiteoutbox}[1]{\fcolorbox{white}{white}{\parbox{\dimexpr \linewidth-2\fboxsep-2\fboxrule}{\whiteout{#1}}}}
\newcommand{\notebox}[1]{\fcolorbox{blue}{white}{\parbox{\dimexpr \linewidth-2\fboxsep-2\fboxrule}{#1}}}

\ifhide%
    \newcommand{\hmask}[1]{\phantom{\varwidth{\linewidth}#1\endvarwidth}}%
\else%
    \newcommand{\hmask}[1]{#1}%
\fi

\ifhide%
    \newcommand{\hignore}[1]{}%
\else%
    \newcommand{\hignore}[1]{#1}%
\fi

\ifpost%
    \newcommand{\nopost}[1]{}%
\else%
    \newcommand{\nopost}[1]{#1}%
\fi

\ifhide%
    \newcommand{\hidebox}[1]{\phantom{\varwidth{\linewidth}#1\endvarwidth}}%
\else%
    \newcommand{\hidebox}[1]{\fbox{\parbox{\linewidth}{#1}}}%
\fi

\ifhide%
    \newcommand{\wbox}[1]{\whiteoutbox{#1}}%
\else%
    \newcommand{\wbox}[1]{\notebox{#1}}%
\fi

% \ifhide%
%     \newcommand{\clickeranswer}[1]{#1}%
% \else%
%     \newcommand{\clickeranswer}[1]{\textbf{\textcolor{blue}{#1}}}%
% \fi

\ifhideclicker%
    \newcommand{\clickeranswer}[1]{#1}%
\else%
    \ifhide%
        \newcommand{\clickeranswer}[1]{#1}%
    \else%
        \newcommand{\clickeranswer}[1]{\textbf{\textcolor{blue}{#1}}}%
    \fi
\fi

\input{../utils/slide-preamble2.tex}
\newcommand{\highlight}[1]{\textcolor{violet}{\textit{\textbf{#1}}}}
\newcommand{\super}[1]{\ensuremath{^{\textrm{#1}}}}
\newcommand{\sub}[1]{\ensuremath{_{\textrm{#1}}}}
\newcommand{\dC}{\ensuremath{^\circ{\textrm{C}}}}
\newcommand{\tb}{\hspace{2em}}
\providecommand{\e}[1]{\ensuremath{\times 10^{#1}}}
\newcommand{\myHangIndent}{\hangindent=5mm}

\makeatletter
\newcommand*{\rom}[1]{\expandafter\@slowromancap\romannumeral #1@}
\makeatother

\newcommand{\blankslide}{{\setbeamercolor{background canvas}{bg=black}
\setbeamercolor{whitetext}{fg=white}
\begin{frame}<handout:0>[plain]
\end{frame}}}

\newcommand{\whiteslide}{
\begin{frame}<handout:0>[plain]
\end{frame}}

\newcommand{\f}[1]{\ensuremath{F_{#1}}}

\bibliography{../bib/references}
\input{../utils/title-info.tex}

\title[Population growth]{Population growth}
% \date{\today}
\date{May 13, 2015}


% \setbeamertemplate{section in toc}[sections numbered]
% \setbeamertemplate{subsection in toc}[subsections numbered]

\begin{document}

\begin{noheadline}
\maketitle
\end{noheadline}


\nopost{
\begin{noheadline}
\begin{frame}[c]
    \vspace{-6mm}
    \begin{center} 
        \includegraphics[height=1.2\textheight]{../images/seating-chart-2.pdf}
    \end{center}
\end{frame}
\end{noheadline}
}

\clickerslide{
\begin{frame}
    \begin{clickerquestion}
        \item In the birds called phalaropes, females are much more
            brightly colored than males. They fight over males and mate with
            several different males each breeding season.  Which of the
            following best explains these observations? 

        \begin{clickeroptions}
            \item \clickeranswer{Males do all parental care (incubate eggs,
                    feed young).}
            \item Females are larger than males. 
            \item Phalaropes often swim in a circle while feeding, forming a
                whirlpool that brings prey to the surface. 
            \item Outside of the breeding season, phalaropes are pelagic (they
                live out at sea). 
            \item They breed at high latitudes; females migrate south before
                males. 
        \end{clickeroptions}
    \end{clickerquestion}
\end{frame}
}

\begin{noheadline}
\begin{frame}
\frametitle{Today's issues:}
\vspace{5mm}
% \tableofcontents[subsectionstyle=hide]
\tableofcontents
\end{frame}
\end{noheadline}

\section{What is population ecology?}

\begin{frame}
    \frametitle{What is population ecology?}
    \begin{adjustwidth}{-2em}{-1.5em}
        Members of the same population live in the same area at the
        same time

        % CHECK THIS!!!!
        \begin{itemize}
            \item Interbreed on a regular basis

                \vspace{5mm}
            \item Are exposed to a similar environment and selection pressures

                \vspace{5mm}
            \item Interact via intra-specific competition
        \end{itemize}
    \end{adjustwidth}
\end{frame}

\section[Basic models of population growth]{What are the basic models used to describe population growth?}

\begin{frame}
    \begin{adjustwidth}{-2em}{-1.5em}
        \vspace{-1mm}
        What are the basic models used to describe population growth?

        \vspace{2mm}
        Data on Whooping Cranes of Wood Buffalo National Park and Aransas NWR

        \vspace{-1.1cm}
        \begin{figure}
        \begin{center}
            \includegraphics[width=0.75\linewidth]{crane-growth.pdf}
        \end{center}
        \end{figure}

        \vspace{-8mm}
        \barefootnote{\tiny Data from COSEWIC 2010 and Whooping Crane
                Conservation Society}
    \end{adjustwidth}
\end{frame}

\begin{frame}[t]
    \begin{adjustwidth}{-2em}{-1.5em}
        How fast is the whooping crane population growing?

        \vspace{2mm}
        Data from 3 wild populations:

        \begin{table}%[htbp]
            \centering
            \begin{tabular}{ L{4.7cm} C{2cm} C{2cm} }
                & 2013 & 2014 \\
                \cline{2-3}
                Wood Buffalo-TX & 300 & 304 \\
                Louisiana resident & 34 & 30 \\
                Wisconsin-Florida & 110 & 114 \\
            \end{tabular}
        \end{table}

        \vspace{3mm}
        Growth rate over 1 year:

        \begin{table}%[htbp]
            \centering
            \begin{tabular}{ L{4.7cm} C{2cm} C{2cm} }
                & rate & \% change \\
                \cline{2-3}
                Wood Buffalo--TX &
                \cmask{\scriptsize $\frac{304}{300}=1.013$} &
                \cmask{\scriptsize $1.3\%$ increase} \\[3ex]
                Louisiana resident &
                \cmask{\scriptsize $\frac{30}{34}=0.882$} &
                \cmask{\scriptsize $11.8\%$ decrease} \\[3ex]
                Wisconsin--Florida &
                \cmask{\scriptsize $\frac{114}{110}=1.036$} &
                \cmask{\scriptsize $3.6\%$ increase} \\
            \end{tabular}
        \end{table}
    \end{adjustwidth}
\end{frame}

\begin{frame}[t]
    \begin{adjustwidth}{-2em}{-1.5em}
        Generalizing these ideas \ldots

        \vspace{2mm}
        Let:
        \begin{description}
            \item[\popsize{0}] = population size at start
            \item[\popsize{1}] = population size one breeding interval later
        \end{description}

        \uncover<2->{
        \vspace{2mm}
        Then,
        \[
            \frac{\popsize{1}}{\popsize{0}} = \popgrowthratediscrete{}
        \]

        \vspace{2mm}
        \begin{center}
        \popgrowthratediscrete{} is the finite rate of increase
        \end{center}
        }

        \uncover<3->{
        \vspace{2mm}
        In the Wood Buffalo--TX population, from 2013--2014,
        $\popgrowthratediscrete{} = \frac{304}{300} = 1.013$
        }

        \uncover<4->{
        \vspace{2mm}
        What is \popgrowthratediscrete{} if $\popsize{1} = 280$?

        \nbox{$\popgrowthratediscrete{} = \frac{\popsize{1}}{\popsize{0}} =
            \frac{280}{300} = 0.933$}

        \vspace{2mm}
        What is \popgrowthratediscrete{} if $\popsize{1} = 315$?

        \nbox{$\popgrowthratediscrete{} = \frac{\popsize{1}}{\popsize{0}} =
            \frac{315}{300} = 1.05$}
        }
    \end{adjustwidth}
\end{frame}

\begin{frame}[t]
    \begin{adjustwidth}{-2em}{-1.5em}
        \vspace{-4mm}
        \[
            \frac{\popsize{1}}{\popsize{0}} = \popgrowthratediscrete{}
        \]

        \vspace{-3mm}
        \[
            \popsize{1} = \popsize{0} \popgrowthratediscrete{}
        \]

        \uncover<2->{
        \begin{center}
            In general,
            $\popsize{\ptime} = \popsize{0} \popgrowthratediscrete{}^{\ptime}$
        \end{center}
        }

        \uncover<3->{
        \popgrowthratediscrete{} is an aggregate growth rate, computed
        over discrete intervals. Why is \ptime an exponent?

        \nbox{The population size changes by (is multiplied by)
            \popgrowthratediscrete{} each interval (generation). Thus, you must
            multiply by \popgrowthratediscrete{} for each generation.}
        }

        \uncover<4->{
        How can we express population growth on a per capita (per individual)
        basis, computed at any time (i.e., treating time as continuous)?
        }

        \begin{itemize}
            \item<5-> Use \popgrowthrate{} to symbolize the per capita growth
                rate

            \vspace{3mm}
            \item<6-> \popgrowthrate{} is the instantaneous rate of increase,
                like interest compounded continuously.
        \end{itemize}

    \end{adjustwidth}
\end{frame}

\begin{frame}[t]
    \begin{adjustwidth}{-2em}{-1.5em}
        \vspace{-6mm}
        \[
            \popgrowthratediscrete{} = e^{\popgrowthrate{}}
        \]
        (this is a fundamental relationship between finite rates and
        instantaneous rates)
        \uncover<2->{
        \[
            \ln(\popgrowthratediscrete{}) = \popgrowthrate{}
        \]
        }

        \uncover<3->{
        \begin{table}%[htbp]
            \centering
            \begin{tabular}{ L{4.7cm} C{2cm} C{2.5cm} }
                & \popgrowthratediscrete{} & \popgrowthrate{} \\
                \cline{2-3}
                Wood Buffalo--TX &
                1.013 &
                \cmask{\scriptsize $\ln(1.013) = 0.013$} \\[3ex]
                Louisiana resident &
                0.882 &
                \cmask{\scriptsize $\ln(0.882) = -0.126$} \\[3ex]
                Wisconsin--Florida &
                1.036 &
                \cmask{\scriptsize $\ln(1.036) =  0.035$} \\
            \end{tabular}
        \end{table}
        }

        \uncover<4->{
        \vspace{2mm}
        \popgrowthratediscrete{} describes discrete growth, \popgrowthrate{}
        describes continuous growth

        \vspace{2mm}
        \popgrowthrate{} hurts my brain, so why?!?
        
        \nbox{Applicable to all populations at all times, and is independent of
            generation time}
        }

    \end{adjustwidth}
\end{frame}

\begin{frame}[t]
    \begin{adjustwidth}{-2em}{-1.5em}
        \vspace{-4mm}
        \[
            \popsize{\ptime} = \popsize{0} \popgrowthratediscrete{}^{\ptime}
        \]
        \[
            \popgrowthratediscrete{} = e^{\popgrowthrate{}}
        \]
        \vspace{-4mm}
        \uncover<2->{
        \[
            \popsize{\ptime} = \popsize{0} e^{\popgrowthrate{}\ptime}
        \]
        }

        \uncover<3->{
        Consider whooping cranes again \ldots

        \vspace{2mm}
        If 22 individuals were alive in 1942, and 600 in 2014, what is
        \popgrowthrate{}?

        \nbox{$600 = 22 e^{\popgrowthrate{}(72years)}$}
        \nbox{$\frac{600}{22} = e^{\popgrowthrate{}(72years)}$}
        \nbox{$\ln(\frac{600}{22}) = \popgrowthrate{}(72years)$}
        \nbox{$\frac{\ln(\frac{600}{22})}{72} = \popgrowthrate{} = 0.0459$}
        }

    \end{adjustwidth}
\end{frame}

\clickerslide{
\begin{frame}
    \begin{clickerquestion}
    \item If this \popgrowthrate{} (0.0459) were sustained, how many MORE years
        will it take to reach the goal of a world population of 1000 (HINT:
        $\popsize{0} = 600$?

        \begin{clickeroptions}
            \item $\approx$108 years
            \item $\approx$35 years
            \item $\approx$24 years
            \item \clickeranswer{$\approx$11 years}
        \end{clickeroptions}
    \end{clickerquestion}

        \nbox{$1000 = 600 e^{0.0459\ptime}$}
        \nbox{$\frac{1000}{600} = e^{0.0459\ptime}$}
        \nbox{$\log(\frac{1000}{600}) = 0.0459\ptime$}
        \nbox{$\frac{\log(\frac{1000}{600})}{0.0459} = \ptime = 11.13$}
\end{frame}
}

\begin{frame}[t]
    \begin{adjustwidth}{-2em}{-1.5em}
        Exponential versus logistic (density-dependent) growth

        \begin{center}
        \includegraphics[height=0.5\textheight]{exp-growth-curves.png}
        \includegraphics[height=0.5\textheight]{seal-pop-growth.png}
        \end{center}

        What is the difference here?

        \nbox{In exponential growth, \popgrowthrate{} does not change; i.e.,
            \popgrowthrate{} is independent of population size (growth is
            density-independent). In logistic growth, \popgrowthrate{} is
            dependent (changes with) population size (growth is
            density-dependent)}

    \end{adjustwidth}
\end{frame}

\clickerslide{
\begin{frame}
    \begin{clickerquestion}
        \item The whooping crane population is growing exponentially. Which of
            the following is NOT supported by this observation?

        \begin{clickeroptions}
            \item \clickeranswer{Disease or lack of food and space will limit
                    \popgrowthrate{} once the population reaches 1000.}
            \item They have not yet reached carrying capacity.
            \item \popgrowthrate{} has been (relatively) constant.
            \item Growth has been density independent.
        \end{clickeroptions}
    \end{clickerquestion}
\end{frame}
}

\section{Case studies in population growth}

\subsection{Density dependence}

\begin{frame}[t]
    \begin{adjustwidth}{-2em}{-1.5em}
        \vspace{-3mm}
        Case study: Density-dependent growth

        \begin{center}
            \includegraphics[width=0.73\linewidth]{daphnia-pop-growth.png}
        \end{center}

        What is wrong with this picture?

        \nbox{The logistic curve is a poor fit to the data; there's a big
            overshoot (hump) in population size beyond predicted carrying
            capacity.}
    \end{adjustwidth}
\end{frame}

\begin{frame}[t]
    \begin{adjustwidth}{-2em}{-1.5em}
        What density-dependent factor(s) constrain growth?

        \nbox{Competition for resources (food, nesting sites, space, etc.)}
        \nbox{Predation rate---easier to capture at higher densities}
        \nbox{Disease---higher transmission rates at higher densities}

        \vspace{2.7cm}
        How to test these hypotheses?

        \nbox{Experimentally remove the hypothesized constraint from some
            populatoins, and not in other populations. If it was constraining
            population growth, you should see an increase in population growth
            rate in the populations from which the constraint was removed.}

    \end{adjustwidth}
\end{frame}

\clickerslide{
\begin{frame}
    \begin{clickerquestion}
        \item The \textit{Daphnia} in the experiment had a lifespan of about 85
            days, and fed on a self-sustaining population of algae (none
            added). Which of the following is the most likely hypothesis to
            explain the ``overshoot'' in the growth curve of \textit{Daphnia
                magna}? 
 
        \begin{clickeroptions}
            \item At their peak density, the \textit{Daphnia} ate up all the
                algae. 
            \item \clickeranswer{At high density, the \textit{Daphnia} changed
                    the environment in a way that lowered resource-producing
                    capacity.}
            \item At densities above about 100 individuals/50mL,
                \textit{Daphnia} stop reproducing. 
            \item At high density, a disease outbreak killed about a third of
                the \textit{Daphnia} in less than a week. 
        \end{clickeroptions}
    \end{clickerquestion}

    \nbox{(1)---If so, population would go to zero (we don't see that).
          (2)---Correct.
          (3)---If no reproduction, the population would decline whenever
          \popsize{} is above 100 individuals/50mL (we don't see that).
          (4)---the decline took much longer than a week.}
    \note[item]{What changed about the environment is unknown, but what about a
        case like Kalbab Plateau deer, or elk in Yellowstone prior to the
        introduction of wolves?}
\end{frame}
}

\subsection{Population cycles}

\begin{frame}[t]
    \begin{adjustwidth}{-2em}{-1.5em}
        Population cycles

        \vspace{2mm}
        E.g., Red grouse in England; 4-year cycles in abundance.

        \uncover<2->{
        \vspace{3mm}
        Hypothesis: The population crashes every four years, because there is
        rapid transmission of a parasitic nematode worm at high population
        densities.
        }

        \uncover<3->{
        \vspace{3mm}
        Experiment: In ``crash years'' catch birds at the roost (at night)
        and administer an anti-worm medication.
        }
    \end{adjustwidth}
\end{frame}

\begin{frame}[t]
    \begin{adjustwidth}{-2em}{-1.5em}
        Interpret these graphs:

        \vspace{1mm}
        \includegraphics[width=1.02\linewidth]{pop-cycles.png}

        \vspace{1mm}
        \nbox{Supports the parasite-transmission hypothesis; removing the worms
            eliminated the cycling.}

    \end{adjustwidth}
\end{frame}

\end{document}

\clickerslide{
\begin{frame}
    \begin{clickerquestion}
        \item 
        \begin{clickeroptions}
            \item 
            \item 
            \item 
            \item 
        \end{clickeroptions}
    \end{clickerquestion}
\end{frame}
}

\clickerpost{
{
\usebackgroundtemplate{\includegraphics[page=17,width=\paperwidth]{./24-Radiation-extinction.pdf}}
\begin{frame}[t,plain]
    \begin{adjustwidth}{-2em}{-1.5em}
        \cmask{Answer: 3}
    \end{adjustwidth}
\end{frame}
}
}

