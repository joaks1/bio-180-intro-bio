\documentclass[table]{beamer}
% \documentclass[table,handout]{beamer}
% \setbeameroption{show notes}
% \setbeameroption{hide notes}
% \setbeameroption{show only notes}
\usepackage{varwidth}

\newif\ifhide
\newif\ifpost
\newif\ifhideclicker

\hidetrue
% \hideclickertrue
% \posttrue

\newcommand{\whiteout}[1]{\textcolor{white}{#1}}
\newcommand{\whiteoutbox}[1]{\fcolorbox{white}{white}{\parbox{\dimexpr \linewidth-2\fboxsep-2\fboxrule}{\whiteout{#1}}}}
\newcommand{\notebox}[1]{\fcolorbox{blue}{white}{\parbox{\dimexpr \linewidth-2\fboxsep-2\fboxrule}{#1}}}

\ifhide%
    \newcommand{\hmask}[1]{\phantom{\varwidth{\linewidth}#1\endvarwidth}}%
\else%
    \newcommand{\hmask}[1]{#1}%
\fi

\ifhide%
    \newcommand{\hignore}[1]{}%
\else%
    \newcommand{\hignore}[1]{#1}%
\fi

\ifpost%
    \newcommand{\nopost}[1]{}%
\else%
    \newcommand{\nopost}[1]{#1}%
\fi

\ifhide%
    \newcommand{\hidebox}[1]{\phantom{\varwidth{\linewidth}#1\endvarwidth}}%
\else%
    \newcommand{\hidebox}[1]{\fbox{\parbox{\linewidth}{#1}}}%
\fi

\ifhide%
    \newcommand{\wbox}[1]{\whiteoutbox{#1}}%
\else%
    \newcommand{\wbox}[1]{\notebox{#1}}%
\fi

% \ifhide%
%     \newcommand{\clickeranswer}[1]{#1}%
% \else%
%     \newcommand{\clickeranswer}[1]{\textbf{\textcolor{blue}{#1}}}%
% \fi

\ifhideclicker%
    \newcommand{\clickeranswer}[1]{#1}%
\else%
    \ifhide%
        \newcommand{\clickeranswer}[1]{#1}%
    \else%
        \newcommand{\clickeranswer}[1]{\textbf{\textcolor{blue}{#1}}}%
    \fi
\fi

\input{../utils/slide-preamble2.tex}
\newcommand{\highlight}[1]{\textcolor{violet}{\textit{\textbf{#1}}}}
\newcommand{\super}[1]{\ensuremath{^{\textrm{#1}}}}
\newcommand{\sub}[1]{\ensuremath{_{\textrm{#1}}}}
\newcommand{\dC}{\ensuremath{^\circ{\textrm{C}}}}
\newcommand{\tb}{\hspace{2em}}
\providecommand{\e}[1]{\ensuremath{\times 10^{#1}}}
\newcommand{\myHangIndent}{\hangindent=5mm}

\makeatletter
\newcommand*{\rom}[1]{\expandafter\@slowromancap\romannumeral #1@}
\makeatother

\newcommand{\blankslide}{{\setbeamercolor{background canvas}{bg=black}
\setbeamercolor{whitetext}{fg=white}
\begin{frame}<handout:0>[plain]
\end{frame}}}

\newcommand{\whiteslide}{
\begin{frame}<handout:0>[plain]
\end{frame}}

\newcommand{\f}[1]{\ensuremath{F_{#1}}}

\bibliography{../bib/references}
\input{../utils/title-info.tex}

\title[Mendel: Monohybrid crosses]{Mendel: Monohybrid crosses}
% \date{\today}
\date{April 6, 2015}

\begin{document}

\begin{noheadline}
\maketitle
\end{noheadline}

% \nopost{
% \begin{noheadline}
% \begin{frame}[c]
%     \begin{center} 
%         \includegraphics[height=1.1\textheight]{../images/seating-chart-section-B.pdf}
%     \end{center}
% \end{frame}
% \end{noheadline}
% }

\begin{noheadline}
\begin{frame}
    \begin{clickerquestion}
        \item If a Caucasian moves to the subtropics or tropics, her skin may
            respond to the increased ultraviolet light by tanning. Which of the
            following statements regarding this pattern is correct?
        \begin{clickeroptions}
            \item When a person's skin tans, they are adapting.
            \item \clickeranswer{When a person's skin tans, they are
                    acclimating.}
            \item The trait of being able to tan is an acclimation.
            \item \clickeranswer{The trait of being able to tan is an
                    adaptation.}
        \end{clickeroptions}
    \end{clickerquestion}
\end{frame}
\end{noheadline}

\begin{noheadline}
\begin{frame}
    \begin{clickerquestion}
        \item Which of the following best characterizes evolution by natural
            selection?
        \begin{clickeroptions}
            \item The strongest individuals in a population survive best and
                reproduce the most. Weak individuals are eliminated. 
            \item Survival of the fittest.
            \item \clickeranswer{Any population that exhibits heritable
                    variation and differential reproductive success will
                    evolve.}
            \item Adaptations are for the good of the species---they help the
                species survive.
        \end{clickeroptions}
    \end{clickerquestion}
\end{frame}
\end{noheadline}

\begin{noheadline}
\begin{frame}
\frametitle{Today's issues:}
\tableofcontents
\end{frame}
\end{noheadline}

\section{Darwin's dilemma}

\begin{frame}
    \frametitle{Darwin's dilemma: the problem of variation}

    \uncover<1->{
    The fact of evolution was accepted in the 1870s--1880s.
    }
    
    \bigskip

    \uncover<1->{
    Natural selction an evolutionary process was controversial until 1930s.
    }

    \begin{enumerate}
        \item<1-> Selection will exhaust variation; Why would this ``stop''
            evolution?
            \nbox{\scriptsize If certain traits are favored generation after
                generation, then soon only those certain traits will exist. No
                more variation---no more evolution (even if the environment
                changes)}
            % \vspace{2mm}
        \item<1-> Blending inheritance will eliminate new, advantageous
            variants; why would this ``stop'' evolution?
            \nbox{\scriptsize A new advantageous trait will be rare. It would
                blend after mating and not be pure in the next generation---it
                would be ``watered down.'' This would happen again and again,
                until the trait would disappear for all intents and purposes.}
    \end{enumerate}
\end{frame}

\section{Mendel's solution}

\begin{frame}
    \frametitle{Mendel's solution}

    \begin{itemize}[<+->]
        \item Mendel was trying to understand the basic patterns of inheritance

        \item Two contrasting hypotheses:
            \begin{enumerate}
                \item Blending inheritance
                    \nbox{The genetic material from each parent mixes
                        together---blending to form intermediate phenotypes}

                \item Inheritance of acquired characters
                    \nbox{Traits of individuals change through use or exposure
                        to an environment; the changed form of the trait is is
                        passed on to offspring}
            \end{enumerate}

        \item What are these?
    \end{itemize}
\end{frame}

\section{Analyzing monohybrid crosses}

\subsection{Garden peas as a model organism}

\begin{frame}
    \frametitle{Garden peas as a model organism}

    \begin{itemize}[<+->]
        \item Mendel studied 7 characteristics, and could grow individuals with
            2 distinct phenotypes of each

        \item More specifically, the 2 phenotypes existed in pure breeding (or
            true breeding) lines
            \begin{itemize}
                \item E.g., his tall individuals produced all tall offspring
                    when mated to themselves or other tall individuals
            \end{itemize}
    \end{itemize}
\end{frame}

\begin{frame}
    \begin{clickerquestion}
        \item Why did Mendel bother to analyze 7 different traits? Why not just
            one?
        \begin{clickeroptions}
            \item It is crucial to replicate experiments---meaning, repeat them,
                to test the hypothesis that the original results weren't due to
                unusual subjects or conditions.
            \item In experiments, large sample sizes are better than smaller
                sample sizes.
            \item This was the number of traits that had pure lines available.
            \item \clickeranswer{To test the hypothesis that his results were
                    due to analyzing a weird trait.}
        \end{clickeroptions}
    \end{clickerquestion}
\end{frame}

\begin{frame}
    \begin{clickerquestion}
        \item What was important about using pure lines?
        \begin{clickeroptions}
            \item \clickeranswer{Whatever the hereditary ``stuff'' was, Mendel
                    knew that pure lines had only one version.}
            \item They are not affected by differences in environmental
                conditions (e.g., water, nutrients).
            \item They can be crossed in a controlled way, by snipping anthers
                (pollen-producing organs) and performing hand-pollination.
            \item They are readily available, have short generation times, and
                are easy to grow (good model system).
        \end{clickeroptions}
    \end{clickerquestion}
\end{frame}

\subsection{Analyzing a monohybrid cross}

\begin{frame}
    \frametitle{Analyzing a monohybrid cross---tall vs dwarf growth habit}

    \begin{itemize}[<+->]
        \item Mendel's  protocal:
            \begin{enumerate}
                \item Cross pure-line parentals to yield $F_1$ offspring
                \item Allow $F_1$s to self-fertilize to produce $F_2$ offspring
            \end{enumerate}

        \item Mendel's results:
            \begin{enumerate}
                \item Phenotypes of $F_1$s: All tall
                \item Phenotypes of $F_2$s: Some tall, some dwarf
            \end{enumerate}
    \end{itemize}

    \note[item]{Two HUGE innovations here:}
    \note[item]<1>{He looked at $F_2$s}
    \note[item]<2>{He counted (i.e., observed $\approx$3:1 ratio}
\end{frame}

\begin{frame}
    \begin{itemize}[<+->]
        \item Mendel claimed that blending inheritance is not correct. Which
            observation(s) supports this conclusion?
            \nbox{All tall (not intermediate) individuals in the \f{1}s
                \textbf{and} dwarfed individuals in the \f{2}s---they didn't
                blend!}

        \item Mendel could claim that inheritance of acquired characters is not
            correct. Which observation(s) supports this conclusion?
            \nbox{Dwarfed individuals in the $F_2$s---couldn't have been
                acquired, as no $F_1$s were dwarfed}

        \item The other six traits gave similar results. Why was this
            important?
            \nbox{Evidence that the pattern is general---not limited to height}

        \item Reciprocal crosses gave the same results. Why was this important?
            \nbox{Evidence that the pattern is general---they are not specific
                to the gender of parent}
    \end{itemize}
\end{frame}

\begin{frame}
    \frametitle{Terminology}
    \begin{description}
        \item[Gene] \uncover<2->{Material that influences the phenotype for a
                particular trait, and is passed on to offspring.}
        \item[Allele] \uncover<3->{\ldots}
    \end{description}
\end{frame}

\subsection{Mendel's model}

\begin{frame}
    \frametitle{Mendel's model---A series of hypotheses to explain his results}
    What evidence did he have for each part of his model?

    \begin{enumerate}[<+->]
        \item Inheritance is particulate---blending doesn't occur
            \nbox{Evidence: Integrity of dwarf allele (reappears in \f{2}s)}
        \item Each individual pea plant has two alleles of each gene
            \nbox{Evidence: If \f{1}s only had tall alleles, it would not be
                possible for some of \f{2}s to be dwarfed}
        \item Individuals can be homozygous or heterozygous
            \nbox{Evidence: the pure lines had only tall ($TT$) or only dwarf
                ($tt$) alleles; \f{1}s had to have one of each}
    \end{enumerate}
\end{frame}

\begin{frame}
    \frametitle{Mendel's model---A series of hypotheses to explain his results}
    \begin{enumerate}[<+->]
        \item Some alleles are dominant to others; some alleles are recessive
            \begin{enumerate}[NOTE]
                \item Dominance and recessiveness are defined \highlight{ONLY}
                    in terms of the appearance of heterozygotes.
                \item Most alleles are not dominant or recessive
                \item Genotypes and phenotypes are distinct
            \end{enumerate}
    \end{enumerate}

    \note[item]{Dominance has a much different definition in biology than
        everyday English. Dominance has nothing to do with frequency or
        fitness. E.g., Huntington's disease alleles}
    \note[item]{Genotype = listing of alleles present}
    \note[item]{Introduce notation for peas: $PP$, $Pp$, $pp$}
\end{frame}

\begin{frame}
    \frametitle{Mendel's model---A series of hypotheses to explain his results}
    \begin{enumerate}[<+->]
        \item During gamete formation in a parent, pairs of alleles segregate
            (separate) and go into different gametes (each gamete contains one
            allele of each gene)
            \begin{itemize}
                \item \textbf{This is the principle of segregation}
                \item What are the \highlight{gamete} genotypes produced
                    by parents with the following genotypes?
                    \begin{itemize}
                        \item $PP$ \hmask{$P \quad / \quad P$}
                        \item $Pp$ \hmask{$P \quad / \quad p$}
                        \item $A_{1}A_{2}$ \hmask{$A_{1} \quad / \quad A_{2}$}
                        \item $X^{R}X^{r}$ \hmask{$X^{R} \quad / \quad X^{r}$}
                            \nbox{NOTE: all are in equal proportions (i.e., 1:1)}
                    \end{itemize}
            \end{itemize}
    \end{enumerate}
\end{frame}

\begin{frame}
    \frametitle{Mendel's model---A series of hypotheses to explain his results}
    \begin{enumerate}[<+->]
        \item Male and female gametes fuse to form a zygote
            \begin{itemize}
                \item Each offspring has two alleles
                \item One allele from each parent
            \end{itemize}
    \end{enumerate}
\end{frame}

\subsection{Does Mendel's model work?}

\begin{frame}
    \frametitle{Interpreting a monohybrid cross}
\end{frame}


\end{document}

\begin{noheadline}
\begin{frame}
    \begin{clickerquestion}
        \item 
        \begin{clickeroptions}
            \item 
            \item 
            \item 
            \item 
        \end{clickeroptions}
    \end{clickerquestion}
\end{frame}
\end{noheadline}
