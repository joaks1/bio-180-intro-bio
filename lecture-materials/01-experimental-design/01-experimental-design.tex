\documentclass[table]{beamer}
% \documentclass[table,handout]{beamer}
% \setbeameroption{show notes}
% \setbeameroption{hide notes}
% \setbeameroption{show only notes}
\usepackage{varwidth}

\newif\ifhide
\newif\ifpost
\newif\ifhideclicker

\hidetrue
% \hideclickertrue
% \posttrue

\newcommand{\whiteout}[1]{\textcolor{white}{#1}}
\newcommand{\whiteoutbox}[1]{\fcolorbox{white}{white}{\parbox{\dimexpr \linewidth-2\fboxsep-2\fboxrule}{\whiteout{#1}}}}
\newcommand{\notebox}[1]{\fcolorbox{blue}{white}{\parbox{\dimexpr \linewidth-2\fboxsep-2\fboxrule}{#1}}}

\ifhide%
    \newcommand{\hmask}[1]{\phantom{\varwidth{\linewidth}#1\endvarwidth}}%
\else%
    \newcommand{\hmask}[1]{#1}%
\fi

\ifhide%
    \newcommand{\hignore}[1]{}%
\else%
    \newcommand{\hignore}[1]{#1}%
\fi

\ifpost%
    \newcommand{\nopost}[1]{}%
\else%
    \newcommand{\nopost}[1]{#1}%
\fi

\ifhide%
    \newcommand{\hidebox}[1]{\phantom{\varwidth{\linewidth}#1\endvarwidth}}%
\else%
    \newcommand{\hidebox}[1]{\fbox{\parbox{\linewidth}{#1}}}%
\fi

\ifhide%
    \newcommand{\wbox}[1]{\whiteoutbox{#1}}%
\else%
    \newcommand{\wbox}[1]{\notebox{#1}}%
\fi

% \ifhide%
%     \newcommand{\clickeranswer}[1]{#1}%
% \else%
%     \newcommand{\clickeranswer}[1]{\textbf{\textcolor{blue}{#1}}}%
% \fi

\ifhideclicker%
    \newcommand{\clickeranswer}[1]{#1}%
\else%
    \ifhide%
        \newcommand{\clickeranswer}[1]{#1}%
    \else%
        \newcommand{\clickeranswer}[1]{\textbf{\textcolor{blue}{#1}}}%
    \fi
\fi

\input{../utils/slide-preamble2.tex}
\newcommand{\highlight}[1]{\textcolor{violet}{\textit{\textbf{#1}}}}
\newcommand{\super}[1]{\ensuremath{^{\textrm{#1}}}}
\newcommand{\sub}[1]{\ensuremath{_{\textrm{#1}}}}
\newcommand{\dC}{\ensuremath{^\circ{\textrm{C}}}}
\newcommand{\tb}{\hspace{2em}}
\providecommand{\e}[1]{\ensuremath{\times 10^{#1}}}
\newcommand{\myHangIndent}{\hangindent=5mm}

\makeatletter
\newcommand*{\rom}[1]{\expandafter\@slowromancap\romannumeral #1@}
\makeatother

\newcommand{\blankslide}{{\setbeamercolor{background canvas}{bg=black}
\setbeamercolor{whitetext}{fg=white}
\begin{frame}<handout:0>[plain]
\end{frame}}}

\newcommand{\whiteslide}{
\begin{frame}<handout:0>[plain]
\end{frame}}

\newcommand{\f}[1]{\ensuremath{F_{#1}}}

% \bibliography{../bib/references}
\bibliography{references}
\input{../utils/title-info.tex}

\title[Intro \& Experimental Design]{Course Introduction \& Experimental Design}
% \date{\today}
\date{March 30, 2015}

\begin{document}

\begin{noheadline}
\maketitle
\end{noheadline}

% \nopost{
% \begin{noheadline}
% \begin{frame}[c]
%     \vspace{-3mm}
%     \begin{center} 
%         \includegraphics[height=1.1\textheight]{../images/seating-chart.pdf}
%     \end{center}
% \end{frame}
% \end{noheadline}
% }

% The people
\begin{noheadline}
\begin{frame}
\frametitle{Welcome to Biology 180!}

    \begin{table}%[htbp]
        \centering
        \begin{tabular}{ l l }
            \textbf{Instructor} & \\
            Jamie Oaks & \href{mailto:joaks1@uw.edu}{joaks1@uw.edu} \\[1.5ex]
            \textbf{Staff} & \\
            John Parks, Course Coordinator & \href{mailto:jwparks@uw.edu}{jwparks@uw.edu} \\
            Celese Spencer, Field Trips & \href{mailto:celese@uw.edu}{celese@uw.edu} \\[1.5ex]
            \textbf{Teaching Assistants} \\
        \end{tabular}
    \end{table}

\end{frame}
\end{noheadline}

\begin{noheadline}
\begin{frame}
    \begin{adjustwidth}{-2em}{-2em}
\includegraphics<1| handout:0>[page=1,width=\paperwidth]{./johns-slides.pdf}
\includegraphics<2| handout:0>[page=2,width=\paperwidth]{./johns-slides.pdf}
\includegraphics<3| handout:0>[page=3,width=\paperwidth]{./johns-slides.pdf}
\includegraphics<4| handout:1>[page=4,width=\paperwidth]{./johns-slides.pdf}
\includegraphics<5| handout:0>[page=5,width=\paperwidth]{./johns-slides.pdf}
\includegraphics<6| handout:2>[page=6,width=\paperwidth]{./johns-slides.pdf}
\includegraphics<7| handout:0>[page=7,width=\paperwidth]{./johns-slides.pdf}
\includegraphics<8| handout:0>[page=8,width=\paperwidth]{./johns-slides.pdf}
\includegraphics<9| handout:3>[page=9,width=\paperwidth]{./johns-slides.pdf}
    \end{adjustwidth}
\end{frame}
\end{noheadline}

\blankslide

\begin{noheadline}
\begin{frame}
\frametitle{Welcome to Biology 180!}
\tableofcontents[subsectionstyle=hide]
\end{frame}
\end{noheadline}

\section{What are the course goals?}

\begin{noheadline}
\begin{frame}[t]
    \frametitle{What are the course goals?}

    \vspace{-5mm}
    \begin{center}
    \begin{tikzpicture}
        \node[anchor=south west,inner sep=0] (image) at (0,0) {\includegraphics[width=0.95\textwidth]{pnas-pages.png}};
        \begin{scope}[x={(image.south east)},y={(image.north west)}]
            % \draw[red,ultra thick,rounded corners] (0.62,0.65) rectangle (0.78,0.75);
            \onslide<2->{\draw[->,red,ultra thick] (0.52, 0.52) -- (0.52, 0.46);}
            \onslide<3->{\draw[->,red,ultra thick] (0.71, 0.72) -- (0.71, 0.66);}
            \onslide<4->{\draw[->,red,ultra thick] (0.785, 0.85) -- (0.785, 0.79);}
        \end{scope}
    \end{tikzpicture}
    \end{center}

    \vspace{-3mm}
    \uncover<5->{\textbf{What conclusions can you draw from this graph?} \\}
\end{frame}
\end{noheadline}

\begin{noheadline}
\begin{frame}
    \begin{adjustwidth}{-2em}{-2em}
    \vspace{-2cm}
\includegraphics<1| handout:0>[page=28,width=\paperwidth]{./adams-slides.pdf}
\includegraphics<2| handout:0>[page=29,width=\paperwidth]{./adams-slides.pdf}
\includegraphics<3| handout:0>[page=30,width=\paperwidth]{./adams-slides.pdf}
\includegraphics<4| handout:0>[page=31,width=\paperwidth]{./adams-slides.pdf}
\includegraphics<5| handout:0>[page=32,width=\paperwidth]{./adams-slides.pdf}
\includegraphics<6| handout:0>[page=33,width=\paperwidth]{./adams-slides.pdf}
\includegraphics<7| handout:0>[page=34,width=\paperwidth]{./adams-slides.pdf}
\includegraphics<8| handout:0>[page=35,width=\paperwidth]{./adams-slides.pdf}
\includegraphics<9| handout:1>[page=36,width=\paperwidth]{./adams-slides.pdf}
    \end{adjustwidth}
\end{frame}
\end{noheadline}


\begin{noheadline}
\begin{frame}
    \frametitle{What are the course goals?}
    \begin{adjustwidth}{-1em}{-1em}
    \begin{center}
        \textbf{Our job: Prepare you to succeed in Biology 200, upper level
            courses \ldots and possibly a career related to biology.} \\

        \vspace{5mm}
        \uncover<2->{Criteria for Medical School Recommendations:}
    \end{center}
    \end{adjustwidth}

    \begin{adjustwidth}{-2em}{-2em}
    \begin{multicols}{2}
        \begin{itemize}
                \small
            \item<3-> \highlight{Motivation} for training in research
            \item<4-> Intellectual potential \& \highlight{curiosity}
            \item<5-> Ability to \highlight{analyze/problem-solve}
            \item<6-> \highlight{Creativity} and imagination
            \item<7-> \highlight{Oral} communication skills
            \item<8-> \highlight{Written} communication skills
            \item<9-> Ability to \highlight{work with others}
            \item<10-> \highlight{Maturity}
            \item<11-> \highlight{Emotional stability}
            \item<12-> Industrious \& \highlight{persistent}
            \item<13-> Planning \& \highlight{organizational skills}
            \item<14-> Ethics \& \highlight{integrity}
        \end{itemize}
    \end{multicols}

    \begin{uncoverenv}<15->
    \begin{center}
        \textbf{\#1 goal: Teach you how to think like a biologist.}
    \end{center}
    \end{uncoverenv}
    \end{adjustwidth}
\end{frame}
\end{noheadline}


\section{How does this course work?}

% THERE WILL BE A SEATING CHART; WE WILL E-MAIL TO YOU AFTER LECTURE TODAY

\begin{noheadline}
\begin{frame}
    \frametitle{How does this class work?}

    \begin{adjustwidth}{-1.5em}{-1em}
    \begin{columns}

        \column{0.4\linewidth}

        \begin{itemize}
            \item<1-> Cell phones are not allowed in Bio 180
            \item<2-> Why?
            \begin{itemize}
                \item<3-> Professional development
                \item<4-> Student performance
            \end{itemize}
        \end{itemize}

        \column{0.6\linewidth}

        \begin{uncoverenv}<4->
        \begin{figure}
            \begin{center}
            \includegraphics[width=1\textwidth]{cell-phone-data.png}
            \caption{\tiny \shortfullcite{Duncan2012}}
            \end{center}
        \end{figure}
        \end{uncoverenv}

    \end{columns}
    \end{adjustwidth}
\end{frame}
\end{noheadline}

\begin{noheadline}
\begin{frame}[t]
    \frametitle{How does this class work?}

        \vspace{-5mm}
        \begin{itemize}
            \item<1-> Reading quizzes
                \vspace{14mm}
            \item<2-> Clickers
                \vspace{14mm}
            \item<3-> Practic exams
                \vspace{14mm}
            \item<4-> Exams
        \end{itemize}
\end{frame}
\end{noheadline}

\begin{noheadline}
\begin{frame}
    \frametitle{How does this class work?}

    \begin{table}%[htbp]
        \centering
        \begin{tabular}{ l | c c }
            & Old-school & High-structure \\
            Grade & (2002-2003) & (2007-) \\
            \hline
            $< 1.5$ & 17\% & 3.4\% \\
            $\geq 3.5$ & 14.5\% & 24.3\% \\
        \end{tabular}
    \end{table}

\end{frame}
\end{noheadline}

% slide with clicker qs, practice exams, exams

\section{How does science work?}

\begin{noheadline}
\begin{frame}
    \frametitle{Experimental Design Module}

    \begin{columns}

        \column{0.7\linewidth}

        \vspace{-1cm}
        \begin{minipage}[c][\textheight][c]{\linewidth}
        \begin{itemize}
            % \item Interpreting data and designing experiments
            \item Work in groups of 3; middle person is the scribe
            \item Be sure your names are legible
            \item We're here to answer questions
            \item Give the worksheet to any TA before leaving
        \end{itemize}
        \end{minipage}

        \column{0.3\linewidth}

        \vspace{-1cm}
        \begin{figure}
            \begin{center}
            \includegraphics[width=1.25\textwidth]{../images/xkcd-cartoon.png}
            \caption{\tiny \href{https://xkcd.com/242/}{https://xkcd.com/242/}}
            \end{center}
        \end{figure}
    \end{columns}
\end{frame}
\end{noheadline}

\end{document}


